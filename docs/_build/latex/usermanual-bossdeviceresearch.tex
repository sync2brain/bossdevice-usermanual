%% Generated by Sphinx.
\def\sphinxdocclass{report}
\documentclass[letterpaper,10pt,english]{sphinxmanual}
\ifdefined\pdfpxdimen
   \let\sphinxpxdimen\pdfpxdimen\else\newdimen\sphinxpxdimen
\fi \sphinxpxdimen=.75bp\relax

\PassOptionsToPackage{warn}{textcomp}
\usepackage[utf8]{inputenc}
\ifdefined\DeclareUnicodeCharacter
% support both utf8 and utf8x syntaxes
  \ifdefined\DeclareUnicodeCharacterAsOptional
    \def\sphinxDUC#1{\DeclareUnicodeCharacter{"#1}}
  \else
    \let\sphinxDUC\DeclareUnicodeCharacter
  \fi
  \sphinxDUC{00A0}{\nobreakspace}
  \sphinxDUC{2500}{\sphinxunichar{2500}}
  \sphinxDUC{2502}{\sphinxunichar{2502}}
  \sphinxDUC{2514}{\sphinxunichar{2514}}
  \sphinxDUC{251C}{\sphinxunichar{251C}}
  \sphinxDUC{2572}{\textbackslash}
\fi
\usepackage{cmap}
\usepackage[T1]{fontenc}
\usepackage{amsmath,amssymb,amstext}
\usepackage{babel}



\usepackage{times}
\expandafter\ifx\csname T@LGR\endcsname\relax
\else
% LGR was declared as font encoding
  \substitutefont{LGR}{\rmdefault}{cmr}
  \substitutefont{LGR}{\sfdefault}{cmss}
  \substitutefont{LGR}{\ttdefault}{cmtt}
\fi
\expandafter\ifx\csname T@X2\endcsname\relax
  \expandafter\ifx\csname T@T2A\endcsname\relax
  \else
  % T2A was declared as font encoding
    \substitutefont{T2A}{\rmdefault}{cmr}
    \substitutefont{T2A}{\sfdefault}{cmss}
    \substitutefont{T2A}{\ttdefault}{cmtt}
  \fi
\else
% X2 was declared as font encoding
  \substitutefont{X2}{\rmdefault}{cmr}
  \substitutefont{X2}{\sfdefault}{cmss}
  \substitutefont{X2}{\ttdefault}{cmtt}
\fi


\usepackage[Bjarne]{fncychap}
\usepackage{sphinx}

\fvset{fontsize=\small}
\usepackage{geometry}


% Include hyperref last.
\usepackage{hyperref}
% Fix anchor placement for figures with captions.
\usepackage{hypcap}% it must be loaded after hyperref.
% Set up styles of URL: it should be placed after hyperref.
\urlstyle{same}

\addto\captionsenglish{\renewcommand{\contentsname}{Table of Contents}}

\usepackage{sphinxmessages}




\title{User Manual \sphinxhyphen{} bossdevice research}
\date{Jul 10, 2021}
\release{1.0.0}
\author{sync2brain GmbH}
\newcommand{\sphinxlogo}{\vbox{}}
\renewcommand{\releasename}{Release}
\makeindex
\begin{document}

\pagestyle{empty}
\sphinxmaketitle
\pagestyle{plain}
\sphinxtableofcontents
\pagestyle{normal}
\phantomsection\label{\detokenize{index::doc}}



\chapter{Introduction to bossdevice research}
\label{\detokenize{1_introduction_to_bossdevice_research:introduction-to-bossdevice-research}}\label{\detokenize{1_introduction_to_bossdevice_research::doc}}
The bossdevice research is a real\sphinxhyphen{}time digital signal processor consisting of hardware and software algorithms. It is designed to read\sphinxhyphen{}in a real\sphinxhyphen{}time raw data stream from a biosignal amplifier (electroencephalography, EEG), to continuously analyze the data and to detect patterns in this data based on oscillations in different frequencies. When such a pattern is detected, the device indicates this through a standard output port. This enables a stimulation device (such as a sound generator) to be triggered in response to a specific biosignal pattern occurring. The device can be programmed by the user to detect different patterns.

\begin{sphinxadmonition}{important}{Important:}
The bossdevice research is not a medical device. It may not be used outside of research and it may not be used in trials involving patients. It is not intended as an accessory to a medical device or to control a medical device. It may only be connected to a stimulation device if the stimulation device provides an input port for the purpose of receiving information regarding the desired timing of stimulation. Whether or not a stimulus is then generated in response to a signal from the bossdevice is determined by the stimulation device.
\end{sphinxadmonition}

\begin{figure}[htbp]
\centering
\capstart

\noindent\sphinxincludegraphics{{Fig1_bossdeviceandneurone}.png}
\caption{The bossdevice research placed along with Bittim NeurOne biosignal amplifier.}\label{\detokenize{1_introduction_to_bossdevice_research:id1}}\end{figure}


\chapter{Setup bossdevice research}
\label{\detokenize{2_setup_bossdevice_research:setup-bossdevice-research}}\label{\detokenize{2_setup_bossdevice_research::doc}}
The bossdevice plugs into a compatible EEG amplifier on the input side and generates a fuzzy logic based binary signal on the output side. It reads a real\sphinxhyphen{}time biosignal data stream and generates trigger signal when a pre\sphinxhyphen{}defined EEG\sphinxhyphen{}state is detected. The bossdevice is programmable from Matlab and enables complex experiments to be fully automated. Setting up the device involves several steps and the next sections descibre their details.


\section{Connecting with Biosignal Amplifier}
\label{\detokenize{2_setup_bossdevice_research:connecting-with-biosignal-amplifier}}
The input of the bossdevice i.e. the EEG data is provided by connecting the Biosignal Amplifier (EEG device e.g. Bittium’s NeurOne or Brain Product’s actiCHamp) with the bossdevice using a standard LAN Cable. Search the “REAL TIME OUT” LAN port at the back plate of your EEG device and connect it to the “Biosignal” LAN port available at the bossdevice research. You can also see the “Biosignal” port on the bossdevice research in the figure shown below.

\begin{figure}[htbp]
\centering
\capstart

\noindent\sphinxincludegraphics{{Figure2_biosignallanport}.png}
\caption{Ports available on the bossdevice (see the Biosignal LAN port).}\label{\detokenize{2_setup_bossdevice_research:id1}}\end{figure}


\section{Connecting with Control PC}
\label{\detokenize{2_setup_bossdevice_research:connecting-with-control-pc}}
In order to program the bossdevice research, it needs to be connected to any standard PC hosting a Matlab and other dependencies as described in the “Downloads \& Dependencies” chapter. The “Control PC” LAN port at the backplate of the bossdevice research is connected to any standard LAN/Ethernet port on the backplate of Control PC. Then in order to open the network IP address, follow these steps:
\begin{enumerate}
\sphinxsetlistlabels{\arabic}{enumi}{enumii}{}{.}%
\item {} 
On Microsoft Windows 7/8/10 Click Start Menu \textgreater{} Control Panel \textgreater{} Network and Sharing Centre or Network and Internet \textgreater{} Network and Sharing Centre.

\item {} 
Click Change adapter settings.

\item {} 
Right\sphinxhyphen{}click on Local Area Connection corresponding to LAN/Ethernet port where the “Control PC” LAN cable is connected and click Properties.

\end{enumerate}

\begin{figure}[htbp]
\centering
\capstart

\noindent\sphinxincludegraphics{{Fig3_controlpc1}.png}
\caption{Step 3: Local Area Connection setup of Control PC and bossdevice research}\label{\detokenize{2_setup_bossdevice_research:id2}}\end{figure}
\begin{enumerate}
\sphinxsetlistlabels{\arabic}{enumi}{enumii}{}{.}%
\setcounter{enumi}{3}
\item {} 
Select Internet Protocol Version 4 (TCP/IPv4) and click Properties.

\end{enumerate}

\begin{figure}[htbp]
\centering
\capstart

\noindent\sphinxincludegraphics{{Figure4_controlpc2}.png}
\caption{Step 4: Local Area Connection setup of Control PC and bossdevice research}\label{\detokenize{2_setup_bossdevice_research:id3}}\end{figure}

5. Select Use the following IP address. Then enter the following IP address, Subnet mask, Default gateway, and DNS server.
.. code\sphinxhyphen{}block:

\begin{sphinxVerbatim}[commandchars=\\\{\}]
\PYG{n}{IP} \PYG{n}{Address}\PYG{p}{:}          \PYG{l+m+mf}{10.10}\PYG{o}{.}\PYG{l+m+mf}{10.100}
\PYG{n}{Subnet} \PYG{n}{mask}\PYG{p}{:}         \PYG{l+m+mf}{255.255}\PYG{o}{.}\PYG{l+m+mf}{255.0}
\PYG{n}{Default} \PYG{n}{gateway}\PYG{p}{:}     \PYG{l+m+mf}{10.10}\PYG{o}{.}\PYG{l+m+mf}{11.88}
\PYG{n}{DNS} \PYG{n}{server}\PYG{p}{:}          \PYG{l+m+mf}{10.10}\PYG{o}{.}\PYG{l+m+mf}{11.68}
\end{sphinxVerbatim}

\begin{figure}[htbp]
\centering
\capstart

\noindent\sphinxincludegraphics{{Figure5_controlpc3}.png}
\caption{Step 5: Local Area Connection setup of Control PC and bossdevice research}\label{\detokenize{2_setup_bossdevice_research:id4}}\end{figure}
\begin{enumerate}
\sphinxsetlistlabels{\arabic}{enumi}{enumii}{}{.}%
\setcounter{enumi}{5}
\item {} 
Click OK and OK on your way out of the wizard.

\end{enumerate}

Similar approach can also be used on Linux/MAC OS to setup the requried IP network for communciation between the bossdevice research and any control PC.


\section{Connecting with Synchronizing Devices}
\label{\detokenize{2_setup_bossdevice_research:connecting-with-synchronizing-devices}}
The bossdevice research can send up to 4 TTL trigger signals to connected synchronizing devices from its output port. Search the “TRIGGER OUT 1\sphinxhyphen{}4” BNC ports at back plate of the bossdevice research and connect it to the BNC ports of your intended device.

\begin{figure}[htbp]
\centering
\capstart

\noindent\sphinxincludegraphics{{Figure6_ttlports}.png}
\caption{Ports available on the bossdevice (see the BNC output ports 1\sphinxhyphen{}4 named as TRIGGER OUT 1 till 4).}\label{\detokenize{2_setup_bossdevice_research:id5}}\end{figure}

In addition to the TTL output the 8 bit trigger LPT marker ports can also be connected to the bossdevice research for a numerical input or can also be output from bossdevice research to write synchronizing markers to the raw EEG data or for any other purpose.

\begin{figure}[htbp]
\centering
\capstart

\noindent\sphinxincludegraphics{{Figure7_markersport}.png}
\caption{Ports available on the bossdevice (see the Marker 1 and Marker 2 8 bit LPT ports).}\label{\detokenize{2_setup_bossdevice_research:id6}}\end{figure}


\section{Connecting with Monitor}
\label{\detokenize{2_setup_bossdevice_research:connecting-with-monitor}}
Monitor setup is optional but recommended and it shows the useful information and analytics of the bossdevice research time of execution and errors (if any). The Display Port from backplate of the bossdevice research may be connected to the monitor.

\begin{figure}[htbp]
\centering
\capstart

\noindent\sphinxincludegraphics{{Figure2_biosignallanport}.png}
\caption{Ports available on the bossdevice (see the Display port (DP) for connection to monitor).}\label{\detokenize{2_setup_bossdevice_research:id7}}\end{figure}


\section{Power Supply}
\label{\detokenize{2_setup_bossdevice_research:power-supply}}
The power supply provided along with the bossdevice research can be plugged into the “DC IN 12V” port at the back plate of the bossdevice research as shown in figure below.

\begin{figure}[htbp]
\centering
\capstart

\noindent\sphinxincludegraphics{{Figure2_biosignallanport}.png}
\caption{Ports available on the bossdevice (see the 12V DC power supply connection port).}\label{\detokenize{2_setup_bossdevice_research:id8}}\end{figure}


\section{Switching ON and OFF}
\label{\detokenize{2_setup_bossdevice_research:switching-on-and-off}}
Since the device is all setup, the device can be switched on using the push button at on of the sides of bossdevice. When switched off, the LED present on the button is not illuminated while when it is switched on, a blue LED is illuminated.


\chapter{Setup biosignal Amplifier}
\label{\detokenize{3_setup_biosignal_amplifier:setup-biosignal-amplifier}}\label{\detokenize{3_setup_biosignal_amplifier::doc}}

\section{Bittium NeurOne}
\label{\detokenize{3_setup_biosignal_amplifier:bittium-neurone}}
\begin{sphinxadmonition}{important}{Important:}
bossdevice research accepts a 5kHz sampling rate of the biosignal, therefore the sampling rate from the real time UDP out of NeurOne must be 5kHz.
\end{sphinxadmonition}

\begin{sphinxadmonition}{important}{Important:}
In your NeurOne protocol, keep all the EEG channels on the top and other Aux/bipolar/EMG/ECG channels at the bottom.
\end{sphinxadmonition}

\begin{sphinxadmonition}{important}{Important:}
If two 32 channels preamplifiers are used, do not use the Aux/bipolar inputs/EMG/ECG channels numberd from 33\sphinxhyphen{}40, but only the last 8 of the 2nd amplifier i.e. 73 to 80. Similar pattern applies to other combinition of preamplifiers.
\end{sphinxadmonition}

\begin{sphinxadmonition}{important}{Important:}
bossdevice research can take maximum of 128 EEG channels only, and 8 Aux/Bipoloar/EMG/ECG channels.
\end{sphinxadmonition}


\subsection{Software Version}
\label{\detokenize{3_setup_biosignal_amplifier:software-version}}
The NeurOne software version should be v1.5.1 or higher.

\begin{figure}[htbp]
\centering

\noindent\sphinxincludegraphics{{Figure8_N1}.png}
\end{figure}


\subsection{Firmware Version}
\label{\detokenize{3_setup_biosignal_amplifier:firmware-version}}
The firmware version should be 1.4.30 or higher.


\subsection{Real Time Out Channels}
\label{\detokenize{3_setup_biosignal_amplifier:real-time-out-channels}}
The Target UDP Port should be changed to 25000 from default of 50000, and the rest of details as shown in the figure below.

\begin{figure}[htbp]
\centering

\noindent\sphinxincludegraphics{{Figure9_N1}.png}
\end{figure}


\chapter{Matlab API Documentation}
\label{\detokenize{4_api_documentation:matlab-api-documentation}}\label{\detokenize{4_api_documentation::doc}}
The bossdevice API is a set of low\sphinxhyphen{}level functions that allows user to control the BOSS device in a user\sphinxhyphen{}friendly programmable manner. The combined execution of such low level functions allows the bossdevice to perform certain high level tasks that could be amongst but not limited to generating various number of trigger outputs, timed in a certain fashion, triggering at a certain Phase of an ongoing EEG Oscillation and guided by the Oscillation Amplitude Thresholds.

\begin{sphinxadmonition}{important}{Important:}
Please downloaded Matlab 2021a, Simulink Real\sphinxhyphen{}Time and Simulink Coder Adds\sphinxhyphen{}On from Matlab toolboxes, download the repository from master branch as detailed on “Downloads \& Dependencies” chapter of the user manual. Add the bossdevice firmware and repository to the search path of Matlab 2021a. Observe from command line that if bossdevice firmware has been loaded from intended path, otherwise many copies of API are available on your device and Matlab 2021a is confused about choosing one therefore it is recommended to change the current directory of Matlab 2021a to the correct path of bossdevice latest API.
\end{sphinxadmonition}


\section{Initialization}
\label{\detokenize{4_api_documentation:initialization}}

\subsection{Description}
\label{\detokenize{4_api_documentation:description}}
BOSS Device API can be initialized as a MATLAB Class object and provides access to its methods and functions to generate various trigger outputs and define multiple brain states upon which the predefined triggers to be released.


\subsection{Syntax}
\label{\detokenize{4_api_documentation:syntax}}
\begin{sphinxVerbatim}[commandchars=\\\{\}]
\PYG{p}{[}\PYG{n}{MATLAB\PYGZus{}class\PYGZus{}object}\PYG{p}{]} \PYG{p}{=} \PYG{n}{bossdevice}\PYG{p}{;}
\end{sphinxVerbatim}


\subsection{Example}
\label{\detokenize{4_api_documentation:example}}
\begin{sphinxVerbatim}[commandchars=\\\{\}]
\PYG{n}{bd}\PYG{p}{=}\PYG{n}{bossdevice}\PYG{p}{;} \PYG{c}{\PYGZpc{} \PYGZdq{}bd\PYGZdq{} class object is created in MATLAB workspace}
\end{sphinxVerbatim}


\bigskip\hrule\bigskip



\section{sendPulse}
\label{\detokenize{4_api_documentation:sendpulse}}

\subsection{Description}
\label{\detokenize{4_api_documentation:id1}}
sendPulse method of the bossdevice class allows you to generate 1 pulse at a specified Output port (out of 4) of the bossdevice.


\subsection{Syntax}
\label{\detokenize{4_api_documentation:id2}}
\begin{sphinxVerbatim}[commandchars=\\\{\}]
\PYG{n}{obj}\PYG{p}{.}\PYG{n}{sendPulse}\PYG{p}{(}\PYG{p}{[}\PYG{n}{bossdevice\PYGZus{}port\PYGZus{}number}\PYG{p}{]}\PYG{p}{)} \PYG{c}{\PYGZpc{} [bossdevice\PYGZus{}port\PYGZus{}number] is non\PYGZhy{}zero positive integer less than or equal to 4}
\end{sphinxVerbatim}


\subsection{Example}
\label{\detokenize{4_api_documentation:id3}}
\begin{sphinxVerbatim}[commandchars=\\\{\}]
\PYG{n}{bd}\PYG{p}{.}\PYG{n}{sendPulse}\PYG{p}{(}\PYG{l+m+mi}{1}\PYG{p}{)} \PYG{c}{\PYGZpc{} delivers a single pulse at the Output port 1}
\PYG{n}{bd}\PYG{p}{.}\PYG{n}{sendPulse}\PYG{p}{(}\PYG{l+m+mi}{3}\PYG{p}{)} \PYG{c}{\PYGZpc{} delivers a single pulse at the Output port 3}
\end{sphinxVerbatim}


\bigskip\hrule\bigskip



\section{manualTrigger}
\label{\detokenize{4_api_documentation:manualtrigger}}

\subsection{Description}
\label{\detokenize{4_api_documentation:id4}}
manualTrigger method of bossdevice allows you to deliver a pre\sphinxhyphen{}built sequence of up to 1500 triggers where each trigger is timed relative to the onset of manualTrigger command, delivers on a specified Output port and also delivers an 8\sphinxhyphen{}bit marker assigned to each trigger as well.
The pre\sphinxhyphen{}built sequence is generated using another command (see below: configure\_time\_port\_marker) however the bossdevice starts delivering the pulses configured before when manualTrigger method is executed.


\subsection{Syntax}
\label{\detokenize{4_api_documentation:id5}}
\begin{sphinxVerbatim}[commandchars=\\\{\}]
\PYG{n}{obj}\PYG{p}{.}\PYG{n}{manualTrigger} \PYG{c}{\PYGZpc{} start the previously configured trigger sequences in [time port marker] style.}
\end{sphinxVerbatim}


\subsection{Example}
\label{\detokenize{4_api_documentation:id6}}
\begin{sphinxVerbatim}[commandchars=\\\{\}]
\PYG{n}{bd}\PYG{p}{.}\PYG{n}{manualTrigger}
\end{sphinxVerbatim}


\bigskip\hrule\bigskip



\section{configure\_time\_port\_marker}
\label{\detokenize{4_api_documentation:configure-time-port-marker}}

\subsection{Description}
\label{\detokenize{4_api_documentation:id7}}
configure\_time\_port\_marker method of bossdevice allows you to prepare a sequence of triggers in {[}time port marker{]} vector style. Maximum of 1500 triggers can be loaded onto this configuration method. In order to prepare a single trigger in {[}time port marker{]} style, just give input argument as {[}0 1 1{]} vector.
Each trigger in this vector has three elements , first index being time, second index being the port number of bossdevice to deliver the output pulse and the last index being the 8\sphinxhyphen{}bit event marker to be generated and written to the Biosignal processor for respective trigger.


\subsection{Syntax}
\label{\detokenize{4_api_documentation:id8}}
\begin{sphinxVerbatim}[commandchars=\\\{\}]
\PYG{n}{obj}\PYG{p}{.}\PYG{n}{configure\PYGZus{}time\PYGZus{}port\PYGZus{}marker}\PYG{p}{(}\PYG{p}{[}\PYG{n}{trigger\PYGZus{}sequence\PYGZus{}vector}\PYG{p}{]}
\PYG{c}{\PYGZpc{} [trigger\PYGZus{}sequence vector] is an Nx3 vector , whereas N is number of triggers and the three elements being the time onset, port number and event marker}
\end{sphinxVerbatim}


\subsection{Example}
\label{\detokenize{4_api_documentation:id9}}
\begin{sphinxVerbatim}[commandchars=\\\{\}]
\PYG{n}{bd}\PYG{p}{.}\PYG{n}{configure\PYGZus{}time\PYGZus{}port\PYGZus{}marker}\PYG{p}{(}\PYG{p}{[}\PYG{l+m+mi}{2} \PYG{l+m+mi}{3} \PYG{l+m+mi}{144}\PYG{p}{]}\PYG{p}{)} \PYG{c}{\PYGZpc{}configures a trigger sequence in which the pulse will be delivered 2 seconds after the manualTrigger command execution, on the 3rd port and event writes 144 marker on the Biosignal processor}
\PYG{n}{bd}\PYG{p}{.}\PYG{n}{configure\PYGZus{}time\PYGZus{}port\PYGZus{}marker}\PYG{p}{(}\PYG{p}{[}\PYG{l+m+mi}{3} \PYG{l+m+mi}{3} \PYG{l+m+mi}{144}\PYG{p}{;}\PYG{n}{4} \PYG{l+s}{2} \PYG{l+s}{128])}  \PYG{l+s}{\PYGZpc{}configures} \PYG{l+s}{2} \PYG{l+s}{trigger} \PYG{l+s}{sequence} \PYG{l+s}{in} \PYG{l+s}{which} \PYG{l+s}{the} \PYG{l+s}{first} \PYG{l+s}{pulse} \PYG{l+s}{will} \PYG{l+s}{be} \PYG{l+s}{delivered} \PYG{l+s}{3} \PYG{l+s}{seconds} \PYG{l+s}{after} \PYG{l+s}{the} \PYG{l+s}{manualTrigger} \PYG{l+s}{command} \PYG{l+s}{execution,} \PYG{l+s}{on} \PYG{l+s}{the} \PYG{l+s}{3rd} \PYG{l+s}{port} \PYG{l+s}{and} \PYG{l+s}{will} \PYG{l+s}{writes} \PYG{l+s}{144} \PYG{l+s}{marker} \PYG{l+s}{on} \PYG{l+s}{the} \PYG{l+s}{Biosignal} \PYG{l+s}{processor,} \PYG{l+s}{and} \PYG{l+s}{then} \PYG{l+s}{the} \PYG{l+s}{second} \PYG{l+s}{pulse} \PYG{l+s}{will} \PYG{l+s}{be} \PYG{l+s}{delivered} \PYG{l+s}{4} \PYG{l+s}{seconds} \PYG{l+s}{after} \PYG{l+s}{the} \PYG{l+s}{manualTrigger} \PYG{l+s}{command} \PYG{l+s}{execution,} \PYG{l+s}{on} \PYG{l+s}{the} \PYG{l+s}{2nd} \PYG{l+s}{port} \PYG{l+s}{and} \PYG{l+s}{writes} \PYG{l+s}{128} \PYG{l+s}{event} \PYG{l+s}{marker} \PYG{l+s}{on} \PYG{l+s}{the} \PYG{l+s}{Biosignal} \PYG{l+s}{processor,}
\end{sphinxVerbatim}


\bigskip\hrule\bigskip



\section{stop}
\label{\detokenize{4_api_documentation:stop}}

\subsection{Description}
\label{\detokenize{4_api_documentation:id10}}
Quite often stopping of the trigger sequence is required before the sequence ends. the stop method allows you to stop further generation and delivery of any sequence loaded onto the trigger generator of the bossdevice.
The last configured sequence can again be started by using “manualTrigger” method after the stop command.


\subsection{Syntax}
\label{\detokenize{4_api_documentation:id11}}
\begin{sphinxVerbatim}[commandchars=\\\{\}]
\PYG{n}{obj}\PYG{p}{.}\PYG{n}{stop} \PYG{c}{\PYGZpc{} stops the generation of output triggers from bossdevice however the class object remain intact.}
\end{sphinxVerbatim}


\subsection{Example}
\label{\detokenize{4_api_documentation:id12}}
\begin{sphinxVerbatim}[commandchars=\\\{\}]
\PYG{n}{bd}\PYG{p}{.}\PYG{n}{stop}
\end{sphinxVerbatim}


\bigskip\hrule\bigskip



\section{eeg\_channels}
\label{\detokenize{4_api_documentation:eeg-channels}}

\subsection{Description}
\label{\detokenize{4_api_documentation:id13}}\begin{quote}

bossdevice can work on maximum of 128 EEG channels, however the minimum number of channels required to work with EEG associated features of bossdevice have been kept flexible and can be defined as in number of EEG cannnels being streamed from your Biosignal Processor.
\end{quote}


\subsection{Syntax}
\label{\detokenize{4_api_documentation:id14}}
\begin{sphinxVerbatim}[commandchars=\\\{\}]
\PYG{n}{obj}\PYG{p}{.}\PYG{n}{eeg\PYGZus{}channels}\PYG{p}{=}\PYG{p}{[}\PYG{n}{number\PYGZus{}of\PYGZus{}streaming\PYGZus{}eeg\PYGZus{}channels}\PYG{p}{]} \PYG{c}{\PYGZpc{} [number\PYGZus{}of\PYGZus{}streaming\PYGZus{}eeg\PYGZus{}channels] is an integer from 1 up till 128.}
\end{sphinxVerbatim}


\subsection{Example}
\label{\detokenize{4_api_documentation:id15}}
\begin{sphinxVerbatim}[commandchars=\\\{\}]
\PYG{n}{bd}\PYG{p}{.}\PYG{n}{eeg\PYGZus{}channels}\PYG{p}{=}\PYG{l+m+mi}{32}\PYG{p}{;} \PYG{c}{\PYGZpc{} informing bossbox that 32 EEG channels are being streamed frm your Biosignal Processor}
\PYG{n}{bd}\PYG{p}{.}\PYG{n}{eeg\PYGZus{}channels}\PYG{p}{=}\PYG{l+m+mi}{64}\PYG{p}{;} \PYG{c}{\PYGZpc{} informing bossbox that 64 EEG channels are being streamed frm your Biosignal Processor}
\PYG{n}{bd}\PYG{p}{.}\PYG{n}{eeg\PYGZus{}channels}\PYG{p}{=}\PYG{l+m+mi}{128}\PYG{p}{;} \PYG{c}{\PYGZpc{} informing bossbox that 128 EEG channels are being streamed frm your Biosignal Processor}
\PYG{n}{bd}\PYG{p}{.}\PYG{n}{eeg\PYGZus{}channels}\PYG{p}{=}\PYG{l+m+mi}{5}\PYG{p}{;} \PYG{c}{\PYGZpc{} informing bossbox that 5 EEG channels are being streamed frm your Biosignal Processor}
\end{sphinxVerbatim}


\bigskip\hrule\bigskip



\section{aux\_channels}
\label{\detokenize{4_api_documentation:aux-channels}}

\subsection{Description}
\label{\detokenize{4_api_documentation:id16}}\begin{quote}

bossdevice can work on maximum of 8 Auxiliary/Biopolar/EMG channels, however the minimum number of channels required to work with Auxiliary Channels associated features of bossdevice have been kept flexible and can be defined as in number of Aux cannnels being streamed from your Biosignal Processor.
\end{quote}

\begin{sphinxadmonition}{important}{Important:}
Auxiliary channels should always be the last 8 or N channels in your Protocol of Biosignal Processor otherwise bossdevice will assume them to be EEG channels
\end{sphinxadmonition}


\subsection{Syntax}
\label{\detokenize{4_api_documentation:id17}}
\begin{sphinxVerbatim}[commandchars=\\\{\}]
\PYG{n}{obj}\PYG{p}{.}\PYG{n}{aux\PYGZus{}channels}\PYG{p}{=}\PYG{p}{[}\PYG{n}{number\PYGZus{}of\PYGZus{}streaming\PYGZus{}aux\PYGZus{}channels}\PYG{p}{]} \PYG{c}{\PYGZpc{} [number\PYGZus{}of\PYGZus{}streaming\PYGZus{}eeg\PYGZus{}channels] is an integer from 1 up till 8.}
\end{sphinxVerbatim}


\subsection{Example}
\label{\detokenize{4_api_documentation:id18}}
\begin{sphinxVerbatim}[commandchars=\\\{\}]
\PYG{n}{bd}\PYG{p}{.}\PYG{n}{eeg\PYGZus{}channels}\PYG{p}{=}\PYG{l+m+mi}{8}\PYG{p}{;} \PYG{c}{\PYGZpc{} informing bossbox that 8 aux/Emg/Bipolar channels are being streamed frm your Biosignal Processor}
\PYG{n}{bd}\PYG{p}{.}\PYG{n}{eeg\PYGZus{}channels}\PYG{p}{=}\PYG{l+m+mi}{2}\PYG{p}{;} \PYG{c}{\PYGZpc{} informing bossbox that 2 aux/EMG/Bipolar channels are being}
\end{sphinxVerbatim}


\bigskip\hrule\bigskip



\section{aux\_channels}
\label{\detokenize{4_api_documentation:id19}}

\subsection{Description}
\label{\detokenize{4_api_documentation:id20}}\begin{quote}

Spatial filtering of the signals is an important step before commencing real\sphinxhyphen{}time brain states detection. The bossdevice allows to specify two different spatially filtered channels and can detect brain states (based on the target phase \& amplitude) for both of these spatial filtered channels in parallel. If any one of them is specified then the other one is ignored and assigned 0 weights. The index of eeg\_channels being streamed from the Biosignal processor e.g. ActiCHamp or NeurOne corrosponds to the index of the weights matrix explained below.
\end{quote}


\subsection{Syntax}
\label{\detokenize{4_api_documentation:id21}}
\begin{sphinxVerbatim}[commandchars=\\\{\}]
\PYG{n}{obj}\PYG{p}{.}\PYG{n}{spatial\PYGZus{}filter\PYGZus{}weights}\PYG{p}{=}\PYG{p}{[}\PYG{n}{weights\PYGZus{}for\PYGZus{}both\PYGZus{}channels}\PYG{p}{]}
\PYG{c}{\PYGZpc{}[weights\PYGZus{}for\PYGZus{}both\PYGZus{}channels] is a Nx2 matrix where N is Number of EEG channels specified before (see eeg\PYGZus{}channels) and the first column is first spatially filtered channel, similarly the second column is the second channel. Each element of the column vector is a \PYGZdq{}normalized weight\PYGZdq{} w.r.t. particular column such that the sum of all weights of a particular channel is zero.}
\end{sphinxVerbatim}


\subsection{Example}
\label{\detokenize{4_api_documentation:id22}}
\begin{sphinxVerbatim}[commandchars=\\\{\}]
\PYG{c}{\PYGZpc{} assume that there are 5 EEG channels}
\PYG{n}{bd}\PYG{p}{.}\PYG{n}{spatial\PYGZus{}filter\PYGZus{}weights}\PYG{p}{=}\PYG{p}{[}\PYG{l+m+mi}{0}\PYG{p}{;} \PYG{o}{\PYGZhy{}}\PYG{l+m+mf}{0.5}\PYG{p}{;} \PYG{l+m+mi}{1}\PYG{p}{;} \PYG{o}{\PYGZhy{}}\PYG{l+m+mf}{0.5}\PYG{p}{;} \PYG{l+m+mi}{0}\PYG{p}{]}\PYG{p}{;} \PYG{c}{\PYGZpc{} the channel 1 will be assigned the given weights and all the eeg\PYGZus{}channels for 2nd spatially filtered channel would have 0 weights.}
\PYG{c}{\PYGZpc{} assume that there are 3 EEG channels}
\PYG{n}{bd}\PYG{p}{.}\PYG{n}{spatial\PYGZus{}filter\PYGZus{}weights}\PYG{p}{=}\PYG{p}{[}\PYG{l+m+mi}{0} \PYG{o}{\PYGZhy{}}\PYG{l+m+mf}{0.5}\PYG{p}{;} \PYG{n}{1} \PYG{l+s}{1}\PYG{p}{;} \PYG{n}{0} \PYG{l+s}{\PYGZhy{}0.5]}\PYG{p}{;} \PYG{c}{\PYGZpc{} the channel 1 will be assigned the weights given on first column and channel 2 will be assigned the weights given on 2nd column.}
\end{sphinxVerbatim}


\bigskip\hrule\bigskip



\section{phase\_target}
\label{\detokenize{4_api_documentation:phase-target}}

\subsection{Description}
\label{\detokenize{4_api_documentation:id23}}\begin{quote}

bossdevice contains 3 built in Oscillatory phase prediction models each for Theta (4\sphinxhyphen{}8 Hz), Alpha (8\sphinxhyphen{}14Hz) and Beta (14\sphinxhyphen{}30Hz) frequency bands. Real\sphinxhyphen{}time phase detection can be performed for maximum of 2 different, pre\sphinxhyphen{}specified spatially filtered channels in parallel. This method allows to define a target phase in radians for a particular frequency band.
\end{quote}


\subsection{Syntax}
\label{\detokenize{4_api_documentation:id24}}
\begin{sphinxVerbatim}[commandchars=\\\{\}]
\PYG{n}{obj}\PYG{p}{.}\PYG{n}{alpha}\PYG{p}{.}\PYG{n}{phase\PYGZus{}target}\PYG{p}{(}\PYG{p}{[}\PYG{n}{spatial\PYGZus{}filter\PYGZus{}channel\PYGZus{}number}\PYG{p}{]}\PYG{p}{)}\PYG{p}{=}\PYG{n}{phase\PYGZus{}angle\PYGZus{}in\PYGZus{}radians} \PYG{c}{\PYGZpc{} sets target phase of Alpha Oscillation}
\PYG{n}{obj}\PYG{p}{.}\PYG{n}{theta}\PYG{p}{.}\PYG{n}{phase\PYGZus{}target}\PYG{p}{(}\PYG{p}{[}\PYG{n}{spatial\PYGZus{}filter\PYGZus{}channel\PYGZus{}number}\PYG{p}{]}\PYG{p}{)}\PYG{p}{=}\PYG{n}{phase\PYGZus{}angle\PYGZus{}in\PYGZus{}radians} \PYG{c}{\PYGZpc{} sets target phase of Theta Oscillation}
\PYG{n}{obj}\PYG{p}{.}\PYG{n+nb}{beta}\PYG{p}{.}\PYG{n}{phase\PYGZus{}target}\PYG{p}{(}\PYG{p}{[}\PYG{n}{spatial\PYGZus{}filter\PYGZus{}channel\PYGZus{}number}\PYG{p}{]}\PYG{p}{)}\PYG{p}{=}\PYG{n}{phase\PYGZus{}angle\PYGZus{}in\PYGZus{}radians} \PYG{c}{\PYGZpc{} sets target phase of Beta Oscillation}
\PYG{c}{\PYGZpc{} [spatial\PYGZus{}filter\PYGZus{}channel\PYGZus{}number] is an integer having value 1 or 2.}
\PYG{c}{\PYGZpc{} [phase\PYGZus{}angle\PYGZus{}in\PYGZus{}radians] is 0 for Peak, pi for Trough, \PYGZhy{}pi/2 for Rising Flank \PYGZam{} +pi/2 for Falling Flank}
\end{sphinxVerbatim}

\begin{figure}[htbp]
\centering
\capstart

\noindent\sphinxincludegraphics{{Figure10_phases}.png}
\caption{Radian values of several target phases in bossdevice research}\label{\detokenize{4_api_documentation:id51}}\end{figure}


\subsection{Example}
\label{\detokenize{4_api_documentation:id25}}
\begin{sphinxVerbatim}[commandchars=\\\{\}]
\PYG{n}{bd}\PYG{p}{.}\PYG{n}{alpha}\PYG{p}{.}\PYG{n}{phase\PYGZus{}target}\PYG{p}{(}\PYG{l+m+mi}{1}\PYG{p}{)}\PYG{p}{=}\PYG{l+m+mi}{0}\PYG{p}{;} \PYG{c}{\PYGZpc{} bossdevice is loaded to detect peak (0 radians) from first spatially filtered channel  with the assumption that the Oscillation is in Alpha frequency band}
\PYG{n}{bd}\PYG{p}{.}\PYG{n+nb}{beta}\PYG{p}{.}\PYG{n}{phase\PYGZus{}target}\PYG{p}{(}\PYG{l+m+mi}{2}\PYG{p}{)}\PYG{p}{=}\PYG{n+nb}{pi}\PYG{p}{;} \PYG{c}{\PYGZpc{} bossdevice is loaded to detect trough(pi radians) from second spatially filtered channel with the assumption that the Oscillation is in Beta frequency band}
\end{sphinxVerbatim}


\bigskip\hrule\bigskip



\section{phase\_plusminus}
\label{\detokenize{4_api_documentation:phase-plusminus}}

\subsection{Description}
\label{\detokenize{4_api_documentation:id26}}\begin{quote}

Defining absolute target phase angles in order to detect a brain state is often prone to error mainly due to the resolution of data obtained after sampling rate transition. In order to overcome this digitization resolution error another parameter has to be defined such that the vicinities of the target phase shall be made clear to the detection algorithm. For an instance, while detecting a 0 radians phase, the phase vector would probably look like this {[}\sphinxhyphen{}0.001324 \sphinxhyphen{}0.00234 0.00243 0.004324{]}, and since none of them are mathematically equivalent to zero therefore in order to not allow to skip such Oscillatory Peak events and to increase the accuracy of the phase detection, a tolerance value is to be provided.
\end{quote}


\subsection{Syntax}
\label{\detokenize{4_api_documentation:id27}}
\begin{sphinxVerbatim}[commandchars=\\\{\}]
\PYG{n}{obj}\PYG{p}{.}\PYG{n}{alpha}\PYG{p}{.}\PYG{n}{phase\PYGZus{}plusminus}\PYG{p}{(}\PYG{p}{[}\PYG{n}{spatial\PYGZus{}filter\PYGZus{}channel\PYGZus{}number}\PYG{p}{]}\PYG{p}{)}\PYG{p}{=}\PYG{n}{phase\PYGZus{}tolerance\PYGZus{}in\PYGZus{}radians} \PYG{c}{\PYGZpc{} sets target phase tolerance of Alpha Oscillation}
\PYG{n}{obj}\PYG{p}{.}\PYG{n}{theta}\PYG{p}{.}\PYG{n}{phase\PYGZus{}plusminus}\PYG{p}{(}\PYG{p}{[}\PYG{n}{spatial\PYGZus{}filter\PYGZus{}channel\PYGZus{}number}\PYG{p}{]}\PYG{p}{)}\PYG{p}{=}\PYG{n}{phase\PYGZus{}tolerance\PYGZus{}in\PYGZus{}radians} \PYG{c}{\PYGZpc{} sets target phase tolerance of Theta Oscillation}
\PYG{n}{obj}\PYG{p}{.}\PYG{n+nb}{beta}\PYG{p}{.}\PYG{n}{phase\PYGZus{}plusminus}\PYG{p}{(}\PYG{p}{[}\PYG{n}{spatial\PYGZus{}filter\PYGZus{}channel\PYGZus{}number}\PYG{p}{]}\PYG{p}{)}\PYG{p}{=}\PYG{n}{phase\PYGZus{}tolerance\PYGZus{}in\PYGZus{}radians} \PYG{c}{\PYGZpc{} sets target phase of Beta Oscillation}
\PYG{c}{\PYGZpc{} [spatial\PYGZus{}filter\PYGZus{}channel\PYGZus{}number] is an integer having value 1 or 2.}
\PYG{c}{\PYGZpc{}Note: While targeting a random phase, the tolerance could go as high as possible i.e. pi radians or Nxpi radians e.g. 2pi etc}
\end{sphinxVerbatim}


\subsection{Example}
\label{\detokenize{4_api_documentation:id28}}
\begin{sphinxVerbatim}[commandchars=\\\{\}]
\PYG{n}{bd}\PYG{p}{.}\PYG{n}{alpha}\PYG{p}{.}\PYG{n}{phase\PYGZus{}plusminus}\PYG{p}{(}\PYG{l+m+mi}{1}\PYG{p}{)}\PYG{p}{=}\PYG{n+nb}{pi}\PYG{o}{/}\PYG{l+m+mi}{40}\PYG{p}{;} \PYG{c}{\PYGZpc{} bossdevice is loaded to detect the setted target phase with a tolerance of pi/40 radians from first spatially filtered channel with the assumption that the Oscillation is in Alpha frequency band}
\PYG{n}{bd}\PYG{p}{.}\PYG{n+nb}{beta}\PYG{p}{.}\PYG{n}{phase\PYGZus{}plusminus}\PYG{p}{(}\PYG{l+m+mi}{2}\PYG{p}{)}\PYG{p}{=}\PYG{n+nb}{pi}\PYG{o}{/}\PYG{l+m+mi}{70}\PYG{p}{;} \PYG{c}{\PYGZpc{} ossdevice is loaded to detect the setted target phase with a tolerance of pi/70 radians from second spatially filtered channel with the assumption that the Oscillation is in Beta frequency band}
\end{sphinxVerbatim}


\bigskip\hrule\bigskip



\section{amplitude\_min}
\label{\detokenize{4_api_documentation:amplitude-min}}

\subsection{Description}
\label{\detokenize{4_api_documentation:id29}}\begin{quote}

Defining minimum amplitude threshold in order to match a specific brain state is important. The bossdevice allows to define minimum amplitude threshold in micro volts that must be reached along with other brain state associated conditions such as maximum amplitude, phase target, and phase tolerance in order to be able to declare the brain state as detected and trigger further sequence of events. bossdevice contains 3 built in Oscillatory minimum amplitude estimation models each for Theta (4\sphinxhyphen{}8 Hz), Alpha (8\sphinxhyphen{}14Hz) and Beta (14\sphinxhyphen{}30Hz) frequency bands. Real\sphinxhyphen{}time minimum amplitude detection can be performed for maximum of 2 different, pre\sphinxhyphen{}specified spatially filtered channels in parallel.
\end{quote}


\subsection{Syntax}
\label{\detokenize{4_api_documentation:id30}}
\begin{sphinxVerbatim}[commandchars=\\\{\}]
\PYG{n}{obj}\PYG{p}{.}\PYG{n}{alpha}\PYG{p}{.}\PYG{n}{amplitude\PYGZus{}min}\PYG{p}{(}\PYG{p}{[}\PYG{n}{spatial\PYGZus{}filter\PYGZus{}channel\PYGZus{}number}\PYG{p}{]}\PYG{p}{)}\PYG{p}{=}\PYG{n}{min\PYGZus{}amplitude\PYGZus{}microV} \PYG{c}{\PYGZpc{} sets minimum amplitude threshold of Alpha Oscillation}
\PYG{n}{obj}\PYG{p}{.}\PYG{n}{theta}\PYG{p}{.}\PYG{n}{amplitude\PYGZus{}min}\PYG{p}{(}\PYG{p}{[}\PYG{n}{spatial\PYGZus{}filter\PYGZus{}channel\PYGZus{}number}\PYG{p}{]}\PYG{p}{)}\PYG{p}{=}\PYG{n}{min\PYGZus{}amplitude\PYGZus{}microV} \PYG{c}{\PYGZpc{} sets minimum amplitude threshold of Theta Oscillation}
\PYG{n}{obj}\PYG{p}{.}\PYG{n+nb}{beta}\PYG{p}{.}\PYG{n}{amplitude\PYGZus{}min}\PYG{p}{(}\PYG{p}{[}\PYG{n}{spatial\PYGZus{}filter\PYGZus{}channel\PYGZus{}number}\PYG{p}{]}\PYG{p}{)}\PYG{p}{=}\PYG{n}{min\PYGZus{}amplitude\PYGZus{}microV} \PYG{c}{\PYGZpc{} sets minimum amplitude threshold of Beta Oscillation}
\PYG{c}{\PYGZpc{} [spatial\PYGZus{}filter\PYGZus{}channel\PYGZus{}number] is an integer having value 1 or 2.}
\end{sphinxVerbatim}


\subsection{Example}
\label{\detokenize{4_api_documentation:id31}}
\begin{sphinxVerbatim}[commandchars=\\\{\}]
\PYG{n}{bd}\PYG{p}{.}\PYG{n}{alpha}\PYG{p}{.}\PYG{n}{amplitude\PYGZus{}min}\PYG{p}{(}\PYG{l+m+mi}{1}\PYG{p}{)}\PYG{p}{=}\PYG{l+m+mi}{1000}\PYG{p}{;} \PYG{c}{\PYGZpc{} bossdevice is loaded to monitor the specified minimum amplitude threshold from first spatially filtered channel with the assumption that the Oscillation is in Alpha frequency band}
\PYG{n}{bd}\PYG{p}{.}\PYG{n+nb}{beta}\PYG{p}{.}\PYG{n}{amplitude\PYGZus{}min}\PYG{p}{(}\PYG{l+m+mi}{2}\PYG{p}{)}\PYG{p}{=}\PYG{l+m+mf}{1e5}\PYG{p}{;} \PYG{c}{\PYGZpc{} bossdevice is loaded to monitor the specified minimum amplitude threshold from second spatially filtered channel with the assumption that the Oscillation is in Beta frequency band}
\end{sphinxVerbatim}


\bigskip\hrule\bigskip



\section{amplitude\_max}
\label{\detokenize{4_api_documentation:amplitude-max}}

\subsection{Description}
\label{\detokenize{4_api_documentation:id32}}\begin{quote}

Defining maximum amplitude threshold in order to match a specific brain state is important. The bossdevice allows to define maximum amplitude threshold in micro volts that must be not be reached in order to be able to declare the brain state as detected and trigger further sequence of events. The bossdevice contains 3 built in Oscillatory maximum amplitude estimation models each for Theta (4\sphinxhyphen{}8 Hz), Alpha (8\sphinxhyphen{}14Hz) and Beta (14\sphinxhyphen{}30Hz) frequency bands. Real\sphinxhyphen{}time maximum amplitude detection can be performed for maximum of 2 different, pre\sphinxhyphen{}specified spatially filtered channels in parallel.
\end{quote}


\subsection{Syntax}
\label{\detokenize{4_api_documentation:id33}}
\begin{sphinxVerbatim}[commandchars=\\\{\}]
\PYG{n}{obj}\PYG{p}{.}\PYG{n}{alpha}\PYG{p}{.}\PYG{n}{amplitude\PYGZus{}max}\PYG{p}{(}\PYG{p}{[}\PYG{n}{spatial\PYGZus{}filter\PYGZus{}channel\PYGZus{}number}\PYG{p}{]}\PYG{p}{)}\PYG{p}{=}\PYG{n}{max\PYGZus{}amplitude\PYGZus{}microV} \PYG{c}{\PYGZpc{} sets maximum amplitude threshold of Alpha Oscillation}
\PYG{n}{obj}\PYG{p}{.}\PYG{n}{theta}\PYG{p}{.}\PYG{n}{amplitude\PYGZus{}max}\PYG{p}{(}\PYG{p}{[}\PYG{n}{spatial\PYGZus{}filter\PYGZus{}channel\PYGZus{}number}\PYG{p}{]}\PYG{p}{)}\PYG{p}{=}\PYG{n}{max\PYGZus{}amplitude\PYGZus{}microV} \PYG{c}{\PYGZpc{} sets maximum amplitude threshold of Theta Oscillation}
\PYG{n}{obj}\PYG{p}{.}\PYG{n+nb}{beta}\PYG{p}{.}\PYG{n}{amplitude\PYGZus{}max}\PYG{p}{(}\PYG{p}{[}\PYG{n}{spatial\PYGZus{}filter\PYGZus{}channel\PYGZus{}number}\PYG{p}{]}\PYG{p}{)}\PYG{p}{=}\PYG{n}{max\PYGZus{}amplitude\PYGZus{}microV} \PYG{c}{\PYGZpc{} sets maximum amplitude threshold of Beta Oscillation}
\PYG{c}{\PYGZpc{} [spatial\PYGZus{}filter\PYGZus{}channel\PYGZus{}number] is an integer having value 1 or 2.}
\end{sphinxVerbatim}


\subsection{Example}
\label{\detokenize{4_api_documentation:id34}}
\begin{sphinxVerbatim}[commandchars=\\\{\}]
\PYG{n}{bd}\PYG{p}{.}\PYG{n}{alpha}\PYG{p}{.}\PYG{n}{amplitude\PYGZus{}max}\PYG{p}{(}\PYG{l+m+mi}{1}\PYG{p}{)}\PYG{p}{=}\PYG{l+m+mi}{1000}\PYG{p}{;} \PYG{c}{\PYGZpc{} bossdevice is loaded to monitor the specified maximum amplitude threshold from first spatially filtered channel with the assumption that the Oscillation is in Alpha frequency band}
\PYG{n}{bd}\PYG{p}{.}\PYG{n+nb}{beta}\PYG{p}{.}\PYG{n}{amplitude\PYGZus{}max}\PYG{p}{(}\PYG{l+m+mi}{2}\PYG{p}{)}\PYG{p}{=}\PYG{l+m+mf}{1e5}\PYG{p}{;} \PYG{c}{\PYGZpc{} bossdevice is loaded to monitor the specified maximum amplitude threshold from second spatially filtered channel with the assumption that the Oscillation is in Beta frequency band}
\end{sphinxVerbatim}


\bigskip\hrule\bigskip



\section{lpf\_fir\_coeffs}
\label{\detokenize{4_api_documentation:lpf-fir-coeffs}}

\subsection{Description}
\label{\detokenize{4_api_documentation:id35}}\begin{quote}

Spatially filtered channel’s signals described in the “spatial\_filter\_weights” function are passed through a low pass FIR filter for anti\sphinxhyphen{}aliasing. The coefficients of this filter can be specified using this function provided that the filter has an order of 100 at maximum.
\end{quote}


\subsection{Syntax}
\label{\detokenize{4_api_documentation:id36}}
\begin{sphinxVerbatim}[commandchars=\\\{\}]
\PYG{n}{obj}\PYG{p}{.}\PYG{n}{alpha}\PYG{p}{.}\PYG{n}{lpf\PYGZus{}fir\PYGZus{}coeffs} \PYG{p}{=} \PYG{p}{[}\PYG{n}{filter\PYGZus{}coefficients}\PYG{p}{]}\PYG{p}{;} \PYG{c}{\PYGZpc{}for Alpha}
\PYG{n}{obj}\PYG{p}{.}\PYG{n}{theta}\PYG{p}{.}\PYG{n}{lpf\PYGZus{}fir\PYGZus{}coeffs} \PYG{p}{=} \PYG{p}{[}\PYG{n}{filter\PYGZus{}coefficients}\PYG{p}{]}\PYG{p}{;} \PYG{c}{\PYGZpc{}for Theta}
\PYG{n}{obj}\PYG{p}{.}\PYG{n+nb}{beta}\PYG{p}{.}\PYG{n}{lpf\PYGZus{}fir\PYGZus{}coeffs} \PYG{p}{=} \PYG{p}{[}\PYG{n}{filter\PYGZus{}coefficients}\PYG{p}{]}\PYG{p}{;} \PYG{c}{\PYGZpc{}for Beta}
\end{sphinxVerbatim}


\subsection{Sampling Rates}
\label{\detokenize{4_api_documentation:sampling-rates}}
\begin{sphinxVerbatim}[commandchars=\\\{\}]
\PYG{n}{Oscillation}         \PYG{l+s}{Sampling} \PYG{l+s}{Rate}
\PYG{n}{Alpha} \PYG{p}{(}\PYG{l+m+mi}{8}\PYG{o}{\PYGZhy{}}\PYG{l+m+mi}{14} \PYG{n}{Hz}\PYG{p}{)}         \PYG{l+m+mi}{500} \PYG{n}{Hz}
\PYG{n}{Theta} \PYG{p}{(}\PYG{l+m+mi}{4}\PYG{o}{\PYGZhy{}}\PYG{l+m+mi}{8} \PYG{n}{Hz}\PYG{p}{)}          \PYG{l+m+mi}{250} \PYG{n}{Hz}
\PYG{n}{Beta}\PYG{p}{(}\PYG{l+m+mi}{14}\PYG{o}{\PYGZhy{}}\PYG{l+m+mi}{30} \PYG{n}{Hz}\PYG{p}{)}          \PYG{l+m+mi}{1000} \PYG{n}{Hz}
\end{sphinxVerbatim}


\subsection{Example}
\label{\detokenize{4_api_documentation:id37}}
\begin{sphinxVerbatim}[commandchars=\\\{\}]
\PYG{n}{bd}\PYG{p}{.}\PYG{n}{alpha}\PYG{p}{.}\PYG{n}{lpf\PYGZus{}fir\PYGZus{}coeffs} \PYG{p}{=}  \PYG{p}{[}\PYG{n}{filter\PYGZus{}coefficients}\PYG{p}{]}\PYG{p}{;}
\PYG{n}{bd}\PYG{p}{.}\PYG{n+nb}{beta}\PYG{p}{.}\PYG{n}{lpf\PYGZus{}fir\PYGZus{}coeffs} \PYG{p}{=}  \PYG{p}{[}\PYG{n}{filter\PYGZus{}coefficients}\PYG{p}{]}\PYG{p}{;}
\end{sphinxVerbatim}


\bigskip\hrule\bigskip



\section{bpf\_fir\_coeffs}
\label{\detokenize{4_api_documentation:bpf-fir-coeffs}}

\subsection{Description}
\label{\detokenize{4_api_documentation:id38}}\begin{quote}

Low pass filtered channel’s signal described in the “lpf\_fir\_coeffs” function is then pass through an FIR Band Pass filter. The coefficients of this filter can be specified using this function provided that the filter has an order of 100 at maximum.
\end{quote}


\subsection{Syntax}
\label{\detokenize{4_api_documentation:id39}}
\begin{sphinxVerbatim}[commandchars=\\\{\}]
\PYG{n}{obj}\PYG{p}{.}\PYG{n}{alpha}\PYG{p}{.}\PYG{n}{bpf\PYGZus{}fir\PYGZus{}coeffs} \PYG{p}{=} \PYG{p}{[}\PYG{n}{filter\PYGZus{}coefficients}\PYG{p}{]}\PYG{p}{;} \PYG{c}{\PYGZpc{}for Alpha}
\PYG{n}{obj}\PYG{p}{.}\PYG{n}{theta}\PYG{p}{.}\PYG{n}{bpf\PYGZus{}fir\PYGZus{}coeffs} \PYG{p}{=} \PYG{p}{[}\PYG{n}{filter\PYGZus{}coefficients}\PYG{p}{]}\PYG{p}{;} \PYG{c}{\PYGZpc{}for Theta}
\PYG{n}{obj}\PYG{p}{.}\PYG{n+nb}{beta}\PYG{p}{.}\PYG{n}{bpf\PYGZus{}fir\PYGZus{}coeffs} \PYG{p}{=} \PYG{p}{[}\PYG{n}{filter\PYGZus{}coefficients}\PYG{p}{]}\PYG{p}{;} \PYG{c}{\PYGZpc{}for Beta}
\end{sphinxVerbatim}


\subsection{Sampling Rates}
\label{\detokenize{4_api_documentation:id40}}
\begin{sphinxVerbatim}[commandchars=\\\{\}]
\PYG{n}{Oscillation}         \PYG{l+s}{Sampling} \PYG{l+s}{Rate}
\PYG{n}{Alpha} \PYG{p}{(}\PYG{l+m+mi}{8}\PYG{o}{\PYGZhy{}}\PYG{l+m+mi}{14} \PYG{n}{Hz}\PYG{p}{)}         \PYG{l+m+mi}{500} \PYG{n}{Hz}
\PYG{n}{Theta} \PYG{p}{(}\PYG{l+m+mi}{4}\PYG{o}{\PYGZhy{}}\PYG{l+m+mi}{8} \PYG{n}{Hz}\PYG{p}{)}          \PYG{l+m+mi}{250} \PYG{n}{Hz}
\PYG{n}{Beta}\PYG{p}{(}\PYG{l+m+mi}{14}\PYG{o}{\PYGZhy{}}\PYG{l+m+mi}{30} \PYG{n}{Hz}\PYG{p}{)}          \PYG{l+m+mi}{1000} \PYG{n}{Hz}
\end{sphinxVerbatim}


\subsection{Example}
\label{\detokenize{4_api_documentation:id41}}
\begin{sphinxVerbatim}[commandchars=\\\{\}]
\PYG{n}{bd}\PYG{p}{.}\PYG{n}{alpha}\PYG{p}{.}\PYG{n}{bpf\PYGZus{}fir\PYGZus{}coeffs} \PYG{p}{=} \PYG{n}{firls}\PYG{p}{(}\PYG{l+m+mi}{80}\PYG{p}{,} \PYG{p}{[}\PYG{l+m+mi}{0} \PYG{p}{(}\PYG{l+m+mi}{11} \PYG{o}{+} \PYG{p}{[}\PYG{o}{\PYGZhy{}}\PYG{l+m+mi}{5} \PYG{o}{\PYGZhy{}}\PYG{l+m+mi}{2} \PYG{o}{+}\PYG{l+m+mi}{2} \PYG{o}{+}\PYG{l+m+mi}{5}\PYG{p}{]}\PYG{p}{)} \PYG{p}{(}\PYG{l+m+mi}{500}\PYG{o}{/}\PYG{l+m+mi}{2}\PYG{p}{)}\PYG{p}{]}\PYG{o}{/}\PYG{p}{(}\PYG{l+m+mi}{500}\PYG{o}{/}\PYG{l+m+mi}{2}\PYG{p}{)}\PYG{p}{,} \PYG{p}{[}\PYG{l+m+mi}{0} \PYG{l+m+mi}{0} \PYG{l+m+mi}{1} \PYG{l+m+mi}{1} \PYG{l+m+mi}{0} \PYG{l+m+mi}{0}\PYG{p}{]}\PYG{p}{,} \PYG{p}{[}\PYG{l+m+mi}{1} \PYG{l+m+mi}{1} \PYG{l+m+mi}{1}\PYG{p}{]} \PYG{p}{)}\PYG{p}{;} \PYG{c}{\PYGZpc{}creates FIR Band Pass filter of order \PYGZdq{}80\PYGZdq{} , around peak frequency of 11 Hz and assigned it to Alpha model of bossdevice}
\PYG{n}{bd}\PYG{p}{.}\PYG{n+nb}{beta}\PYG{p}{.}\PYG{n}{bpf\PYGZus{}fir\PYGZus{}coeffs} \PYG{p}{=} \PYG{n}{firls}\PYG{p}{(}\PYG{l+m+mi}{100}\PYG{p}{,} \PYG{p}{[}\PYG{l+m+mi}{0} \PYG{p}{(}\PYG{l+m+mi}{19} \PYG{o}{+} \PYG{p}{[}\PYG{o}{\PYGZhy{}}\PYG{l+m+mi}{5} \PYG{o}{\PYGZhy{}}\PYG{l+m+mi}{2} \PYG{o}{+}\PYG{l+m+mi}{2} \PYG{o}{+}\PYG{l+m+mi}{5}\PYG{p}{]}\PYG{p}{)} \PYG{p}{(}\PYG{l+m+mi}{1000}\PYG{o}{/}\PYG{l+m+mi}{2}\PYG{p}{)}\PYG{p}{]}\PYG{o}{/}\PYG{p}{(}\PYG{l+m+mi}{1000}\PYG{o}{/}\PYG{l+m+mi}{2}\PYG{p}{)}\PYG{p}{,} \PYG{p}{[}\PYG{l+m+mi}{0} \PYG{l+m+mi}{0} \PYG{l+m+mi}{1} \PYG{l+m+mi}{1} \PYG{l+m+mi}{0} \PYG{l+m+mi}{0}\PYG{p}{]}\PYG{p}{,} \PYG{p}{[}\PYG{l+m+mi}{1} \PYG{l+m+mi}{1} \PYG{l+m+mi}{1}\PYG{p}{]} \PYG{p}{)}\PYG{p}{;} \PYG{c}{\PYGZpc{}creates FIR Band Pass filter of order \PYGZdq{}100\PYGZdq{} , around peak frequency of 19 Hz and assigned it to Beta model of bossdevice}
\end{sphinxVerbatim}


\bigskip\hrule\bigskip



\section{arm}
\label{\detokenize{4_api_documentation:arm}}

\subsection{Description}
\label{\detokenize{4_api_documentation:id42}}
Arming bossdevice would allow bossdevice to actively search for the Brain states already set into it and generate the trigger sequence configured using “configure\_time\_port\_marker” upon every instance of detection.


\subsection{Syntax}
\label{\detokenize{4_api_documentation:id43}}
\begin{sphinxVerbatim}[commandchars=\\\{\}]
\PYG{n}{obj}\PYG{p}{.}\PYG{n}{armed}\PYG{p}{;}
\end{sphinxVerbatim}


\subsection{Example}
\label{\detokenize{4_api_documentation:id44}}
\begin{sphinxVerbatim}[commandchars=\\\{\}]
\PYG{n}{bd}\PYG{p}{.}\PYG{n}{armed}\PYG{p}{;} \PYG{c}{\PYGZpc{} will set bossdevice to detect the specified brain states and trigger sequence}
\end{sphinxVerbatim}


\bigskip\hrule\bigskip



\section{disarm}
\label{\detokenize{4_api_documentation:disarm}}

\subsection{Description}
\label{\detokenize{4_api_documentation:id45}}
Disarming bossdevice would allow bossdevice to stop search for the Brain states already by now “Armed’ into it and immediately stops the generation of the trigger sequence configured using “configure\_time\_port\_marker” .


\subsection{Syntax}
\label{\detokenize{4_api_documentation:id46}}
\begin{sphinxVerbatim}[commandchars=\\\{\}]
\PYG{n}{obj}\PYG{p}{.}\PYG{n}{disarm}\PYG{p}{;}
\end{sphinxVerbatim}


\subsection{Example}
\label{\detokenize{4_api_documentation:id47}}
\begin{sphinxVerbatim}[commandchars=\\\{\}]
\PYG{n}{bd}\PYG{p}{.}\PYG{n}{disarm}\PYG{p}{;} \PYG{c}{\PYGZpc{} will stop bossdevice to detect the specified brain states and stop ongoing trigger sequence}
\end{sphinxVerbatim}


\bigskip\hrule\bigskip



\section{sample\_and\_hold\_period}
\label{\detokenize{4_api_documentation:sample-and-hold-period}}

\subsection{Description}
\label{\detokenize{4_api_documentation:id48}}
Upon an artifactual trigger event such as Trans\sphinxhyphen{}cranial Magnetic Stimulation (TMS) pulse artifact etc., the data gets distorted and is accumulated with a lot of noise, in order to hold the samples for a specified period of time, sample and hold period method can be helpful. This is also helpful to keep the visual output of the bossdevice clean and tidy.


\subsection{Syntax}
\label{\detokenize{4_api_documentation:id49}}
\begin{sphinxVerbatim}[commandchars=\\\{\}]
\PYG{n}{obj}\PYG{p}{.}\PYG{n}{sample\PYGZus{}and\PYGZus{}hold\PYGZus{}period} \PYG{p}{=} \PYG{p}{[}\PYG{n}{no\PYGZus{}of\PYGZus{}sample\PYGZus{}to\PYGZus{}hold}\PYG{p}{]}\PYG{p}{;} \PYG{c}{\PYGZpc{} no\PYGZus{}of\PYGZus{}sample\PYGZus{}to\PYGZus{}hold are calculated at the rate of 5KHz}
\end{sphinxVerbatim}


\subsection{Example}
\label{\detokenize{4_api_documentation:id50}}
\begin{sphinxVerbatim}[commandchars=\\\{\}]
\PYG{n}{bd}\PYG{p}{.}\PYG{n}{sample\PYGZus{}and\PYGZus{}hold\PYGZus{}period} \PYG{p}{=} \PYG{l+m+mi}{500}\PYG{p}{;} \PYG{c}{\PYGZpc{} hold the samples for 100 ms}
\end{sphinxVerbatim}


\bigskip\hrule\bigskip



\chapter{Demo Scripts}
\label{\detokenize{5_demo_scripts:demo-scripts}}\label{\detokenize{5_demo_scripts::doc}}

\section{Open Loop Jittered Stimulation}
\label{\detokenize{5_demo_scripts:open-loop-jittered-stimulation}}
This demo script uses bossdevice research and 2 different approaches to generate jittered open loop stimulus

Resources: 1) bossdevice Switched On, 2) bossdevice Open Source MATLAB API, 3) The stimulator is Switched On, External Trigger mode is turned on and the Stimulator is Enabled

\begin{sphinxVerbatim}[commandchars=\\\{\}]
\PYG{n}{NumberOfTrials}\PYG{p}{=}\PYG{l+m+mi}{10}\PYG{p}{;}
 \PYG{n}{ITI}\PYG{p}{=}\PYG{p}{[}\PYG{l+m+mi}{4} \PYG{l+m+mi}{6}\PYG{p}{]}\PYG{p}{;} \PYG{c}{\PYGZpc{}ITI is seconds \PYGZhy{} a random number between these two values}
 \PYG{c}{\PYGZpc{}\PYGZpc{} Initializing bossdevice research API}
 \PYG{n}{bd}\PYG{p}{=}\PYG{n}{bossdevice}\PYG{p}{;}
 \PYG{c}{\PYGZpc{}\PYGZpc{} Approach 1 \PYGZhy{} For Loop Based Open Loop Stimulation}
 \PYG{n}{bd}\PYG{p}{.}\PYG{n}{configure\PYGZus{}time\PYGZus{}port\PYGZus{}marker}\PYG{p}{(}\PYG{p}{[}\PYG{l+m+mi}{0} \PYG{l+m+mi}{1} \PYG{l+m+mi}{1}\PYG{p}{]}\PYG{p}{)}\PYG{p}{;} \PYG{c}{\PYGZpc{}Configuring Trigger Sequence in [Time Port Marker] format}
 \PYG{n}{for} \PYG{l+s}{TrialNumber=1:NumberOfTrials}
     \PYG{n}{bd}\PYG{p}{.}\PYG{n}{manualTrigger}
     \PYG{n+nb}{disp}\PYG{p}{(}\PYG{p}{[}\PYG{l+s}{\PYGZsq{}}\PYG{l+s}{Triggered Trial \PYGZsh{}\PYGZsq{}} \PYG{n}{num2str}\PYG{p}{(}\PYG{n}{TrialNumber}\PYG{p}{)}\PYG{p}{]}\PYG{p}{)}
     \PYG{n}{min\PYGZus{}inter\PYGZus{}trig\PYGZus{}interval}\PYG{p}{=} \PYG{n}{ITI}\PYG{p}{(}\PYG{l+m+mi}{1}\PYG{p}{)}\PYG{o}{+} \PYG{p}{(}\PYG{n}{ITI}\PYG{p}{(}\PYG{l+m+mi}{2}\PYG{p}{)}\PYG{o}{\PYGZhy{}}\PYG{n}{ITI}\PYG{p}{(}\PYG{l+m+mi}{1}\PYG{p}{)}\PYG{p}{)}\PYG{o}{.*}\PYG{n+nb}{rand}\PYG{p}{(}\PYG{l+m+mi}{1}\PYG{p}{,}\PYG{l+m+mi}{1}\PYG{p}{)}\PYG{p}{;} \PYG{c}{\PYGZpc{}Assigning New Random ITI for this Trial to the bossdevice research}
     \PYG{n}{pause}\PYG{p}{(}\PYG{n}{min\PYGZus{}inter\PYGZus{}trig\PYGZus{}interval}\PYG{p}{)} \PYG{c}{\PYGZpc{}Wait for next trial start}
 \PYG{n}{end}
 \PYG{l+s}{disp(}\PYG{l+s}{\PYGZsq{}Trials Completed\PYGZsq{}}\PYG{l+s}{)}
 \PYG{c}{\PYGZpc{}\PYGZpc{} Approach 2 \PYGZhy{} bossdevice research Sequence Generator Based Open Loop Stimulation}
 \PYG{n}{time\PYGZus{}port\PYGZus{}marker\PYGZus{}sequence}\PYG{p}{=}\PYG{p}{[}\PYG{p}{]}\PYG{p}{;}
 \PYG{n}{Time}\PYG{p}{=}\PYG{l+m+mi}{0}\PYG{p}{;}
 \PYG{n}{time\PYGZus{}port\PYGZus{}marker\PYGZus{}sequence}\PYG{p}{(}\PYG{n}{NumberOfTrials}\PYG{p}{,}\PYG{l+m+mi}{3}\PYG{p}{)}\PYG{p}{=}\PYG{l+m+mi}{0}\PYG{p}{;} \PYG{c}{\PYGZpc{}Pre filling the array}
 \PYG{n}{for} \PYG{l+s}{TrialNumber=1:NumberOfTrials}
     \PYG{n}{Time}\PYG{p}{=}\PYG{n}{Time}\PYG{o}{+}\PYG{n}{ITI}\PYG{p}{(}\PYG{l+m+mi}{1}\PYG{p}{)}\PYG{o}{+} \PYG{p}{(}\PYG{n}{ITI}\PYG{p}{(}\PYG{l+m+mi}{2}\PYG{p}{)}\PYG{o}{\PYGZhy{}}\PYG{n}{ITI}\PYG{p}{(}\PYG{l+m+mi}{1}\PYG{p}{)}\PYG{p}{)}\PYG{o}{.*}\PYG{n+nb}{rand}\PYG{p}{(}\PYG{l+m+mi}{1}\PYG{p}{,}\PYG{l+m+mi}{1}\PYG{p}{)}\PYG{p}{;} \PYG{c}{\PYGZpc{}Generating Sequence of Jittered ITIs for all Trials}
     \PYG{n}{Port}\PYG{p}{=}\PYG{l+m+mi}{1}\PYG{p}{;} \PYG{c}{\PYGZpc{}In order to generatre trigger always on first port , use 2 for 2nd port and 3 for third port}
     \PYG{n}{Marker}\PYG{p}{=}\PYG{n}{TrialNumber}\PYG{p}{;}
     \PYG{n}{time\PYGZus{}port\PYGZus{}marker\PYGZus{}sequence}\PYG{p}{(}\PYG{n}{TrialNumber}\PYG{p}{,}\PYG{p}{:}\PYG{p}{)}\PYG{p}{=}\PYG{p}{[}\PYG{n}{Time} \PYG{n}{Port} \PYG{n}{Marker}\PYG{p}{]}\PYG{p}{;}
 \PYG{n}{end}
 \PYG{l+s}{bd.configure\PYGZus{}time\PYGZus{}port\PYGZus{}marker(time\PYGZus{}port\PYGZus{}marker\PYGZus{}sequence)}\PYG{p}{;} \PYG{c}{\PYGZpc{}Assigning Pregenerated sequence to the bossdevice research}
 \PYG{n}{bd}\PYG{p}{.}\PYG{n}{manualTrigger} \PYG{c}{\PYGZpc{} Triggering the bossdevice research to start sequence TTL Output Generation}
 \PYG{n+nb}{disp}\PYG{p}{(}\PYG{l+s}{\PYGZsq{}}\PYG{l+s}{Trigger Sequence Started by bossdevice research\PYGZsq{}}\PYG{p}{)}
\end{sphinxVerbatim}


\bigskip\hrule\bigskip



\section{mu\sphinxhyphen{}Rhythm Phase Locked Triggering}
\label{\detokenize{5_demo_scripts:mu-rhythm-phase-locked-triggering}}
This demo script uses bossdevice to deliver mu Rising or Falling Flank Phase locked Trigger outputs
Resources: 1) bossdevice Switched On
2) bossdevice Open Source MATLAB API
3) Biosignal Amplifier streaming atleast 5 EEG Channels
4) The stimulator is Switched On, External Trigger mode is turned on and the Stimulator is Enabled

\begin{sphinxVerbatim}[commandchars=\\\{\}]
\PYG{n}{no\PYGZus{}of\PYGZus{}trials}\PYG{p}{=}\PYG{l+m+mi}{10}\PYG{p}{;}
 \PYG{n}{minimium\PYGZus{}inter\PYGZus{}trigger\PYGZus{}interval}\PYG{p}{=}\PYG{l+m+mi}{5}\PYG{p}{;} \PYG{c}{\PYGZpc{}s}
 \PYG{n}{phase}\PYG{p}{=}\PYG{p}{[}\PYG{o}{+}\PYG{n+nb}{pi}\PYG{o}{/}\PYG{l+m+mi}{2} \PYG{o}{\PYGZhy{}}\PYG{n+nb}{pi}\PYG{o}{/}\PYG{l+m+mi}{2}\PYG{p}{]}\PYG{p}{;} \PYG{c}{\PYGZpc{}[falling\PYGZus{}flank rising\PYGZus{}flank]}
 \PYG{n}{phase\PYGZus{}tolerance}\PYG{p}{=}\PYG{n+nb}{pi}\PYG{o}{/}\PYG{l+m+mi}{40}\PYG{p}{;}
 \PYG{n}{individual\PYGZus{}peak\PYGZus{}frequency}\PYG{p}{=}\PYG{l+m+mi}{11}\PYG{p}{;} \PYG{c}{\PYGZpc{} Hz}
 \PYG{n}{bandpassfilter\PYGZus{}order}\PYG{p}{=} \PYG{l+m+mi}{75}\PYG{p}{;}
 \PYG{n}{eeg\PYGZus{}channels}\PYG{p}{=}\PYG{l+m+mi}{5}\PYG{p}{;} \PYG{c}{\PYGZpc{}Assigning Number of channels as equivalent to Num of Channels streamed by Biosignal Processor}
 \PYG{n}{spatial\PYGZus{}filter\PYGZus{}weights}\PYG{p}{=}\PYG{p}{[}\PYG{l+m+mi}{1} \PYG{o}{\PYGZhy{}}\PYG{l+m+mf}{0.25} \PYG{o}{\PYGZhy{}}\PYG{l+m+mf}{0.25} \PYG{o}{\PYGZhy{}}\PYG{l+m+mf}{0.25} \PYG{o}{\PYGZhy{}}\PYG{l+m+mf}{0.25}\PYG{p}{]}\PYG{o}{\PYGZsq{}}\PYG{p}{;} \PYG{c}{\PYGZpc{}Column Vector of Spatial Filter Indexed wrt corrosponding Channels}

 \PYG{n}{time}\PYG{p}{=}\PYG{l+m+mi}{0}\PYG{p}{;}
 \PYG{n}{plasticity\PYGZus{}protocol\PYGZus{}sequence}\PYG{p}{=}\PYG{p}{[}\PYG{p}{]}\PYG{p}{;}

 \PYG{c}{\PYGZpc{}\PYGZpc{} Initializing bossdevice research API}
 \PYG{n}{bd}\PYG{p}{=}\PYG{n}{bossdevice}\PYG{p}{;}
 \PYG{n}{bd}\PYG{p}{.}\PYG{n}{sample\PYGZus{}and\PYGZus{}hold\PYGZus{}period}\PYG{p}{=}\PYG{l+m+mi}{0}\PYG{p}{;}
 \PYG{n}{bd}\PYG{p}{.}\PYG{n}{calibration\PYGZus{}mode} \PYG{p}{=} \PYG{l+s}{\PYGZsq{}}\PYG{l+s}{no\PYGZsq{}}\PYG{p}{;}
 \PYG{n}{bd}\PYG{p}{.}\PYG{n}{armed} \PYG{p}{=} \PYG{l+s}{\PYGZsq{}}\PYG{l+s}{no\PYGZsq{}}\PYG{p}{;}
 \PYG{n}{bd}\PYG{p}{.}\PYG{n}{sample\PYGZus{}and\PYGZus{}hold\PYGZus{}period}\PYG{p}{=}\PYG{l+m+mi}{0}\PYG{p}{;}
 \PYG{n}{bd}\PYG{p}{.}\PYG{n}{theta}\PYG{p}{.}\PYG{n}{ignore}\PYG{p}{;} \PYG{n}{pause}\PYG{p}{(}\PYG{l+m+mf}{0.1}\PYG{p}{)}
 \PYG{n}{bd}\PYG{p}{.}\PYG{n+nb}{beta}\PYG{p}{.}\PYG{n}{ignore}\PYG{p}{;} \PYG{n}{pause}\PYG{p}{(}\PYG{l+m+mf}{0.1}\PYG{p}{)}
 \PYG{n}{bd}\PYG{p}{.}\PYG{n}{alpha}\PYG{p}{.}\PYG{n}{ignore}\PYG{p}{;} \PYG{n}{pause}\PYG{p}{(}\PYG{l+m+mf}{0.1}\PYG{p}{)}
 \PYG{n}{bd}\PYG{p}{.}\PYG{n}{eeg\PYGZus{}channels}\PYG{p}{=}\PYG{n}{eeg\PYGZus{}channels}\PYG{p}{;}

 \PYG{c}{\PYGZpc{}\PYGZpc{} Preparing an Individual Peak Frequency based Band Pass Filter for mu Alpha}
 \PYG{n}{bpf\PYGZus{}fir\PYGZus{}coeffs} \PYG{p}{=} \PYG{n}{firls}\PYG{p}{(}\PYG{n}{bandpassfilter\PYGZus{}order}\PYG{p}{,} \PYG{p}{[}\PYG{l+m+mi}{0} \PYG{p}{(}\PYG{n}{individual\PYGZus{}peak\PYGZus{}frequency} \PYG{o}{+} \PYG{p}{[}\PYG{o}{\PYGZhy{}}\PYG{l+m+mi}{5} \PYG{o}{\PYGZhy{}}\PYG{l+m+mi}{2} \PYG{o}{+}\PYG{l+m+mi}{2} \PYG{o}{+}\PYG{l+m+mi}{5}\PYG{p}{]}\PYG{p}{)} \PYG{p}{(}\PYG{l+m+mi}{500}\PYG{o}{/}\PYG{l+m+mi}{2}\PYG{p}{)}\PYG{p}{]}\PYG{o}{/}\PYG{p}{(}\PYG{l+m+mi}{500}\PYG{o}{/}\PYG{l+m+mi}{2}\PYG{p}{)}\PYG{p}{,} \PYG{p}{[}\PYG{l+m+mi}{0} \PYG{l+m+mi}{0} \PYG{l+m+mi}{1} \PYG{l+m+mi}{1} \PYG{l+m+mi}{0} \PYG{l+m+mi}{0}\PYG{p}{]}\PYG{p}{,} \PYG{p}{[}\PYG{l+m+mi}{1} \PYG{l+m+mi}{1} \PYG{l+m+mi}{1}\PYG{p}{]} \PYG{p}{)}\PYG{p}{;}

 \PYG{c}{\PYGZpc{}\PYGZpc{} Setting Filters on bossdevice research}
 \PYG{n}{bd}\PYG{p}{.}\PYG{n}{spatial\PYGZus{}filter\PYGZus{}weights}\PYG{p}{=}\PYG{n}{spatial\PYGZus{}filter\PYGZus{}weights}\PYG{p}{;}
 \PYG{n}{bd}\PYG{p}{.}\PYG{n}{alpha}\PYG{p}{.}\PYG{n}{bpf\PYGZus{}fir\PYGZus{}coeffs} \PYG{p}{=} \PYG{n}{bpf\PYGZus{}fir\PYGZus{}coeffs}\PYG{p}{;}

 \PYG{c}{\PYGZpc{}\PYGZpc{} Controlling bossdevice research for mu Alpha Phase Locked Triggering}
 \PYG{n}{condition\PYGZus{}index}\PYG{p}{=}\PYG{l+m+mi}{0}\PYG{p}{;}
 \PYG{k}{while} \PYG{p}{(}\PYG{n}{condition\PYGZus{}index} \PYG{o}{\PYGZlt{}=} \PYG{n}{no\PYGZus{}of\PYGZus{}trials}\PYG{p}{)}
     \PYG{k}{if}\PYG{p}{(}\PYG{n}{strcmp}\PYG{p}{(}\PYG{n}{bb}\PYG{p}{.}\PYG{n}{armed}\PYG{p}{,} \PYG{l+s}{\PYGZsq{}}\PYG{l+s}{no\PYGZsq{}}\PYG{p}{)}\PYG{p}{)}
         \PYG{n}{bb}\PYG{p}{.}\PYG{n}{triggers\PYGZus{}remaining} \PYG{p}{=} \PYG{l+m+mi}{1}\PYG{p}{;}
         \PYG{n}{bb}\PYG{p}{.}\PYG{n}{alpha}\PYG{p}{.}\PYG{n}{phase\PYGZus{}target}\PYG{p}{(}\PYG{l+m+mi}{1}\PYG{p}{)} \PYG{p}{=} \PYG{n}{phase}\PYG{p}{(}\PYG{n}{randi}\PYG{p}{(}\PYG{l+m+mi}{1}\PYG{p}{:}\PYG{n+nb}{numel}\PYG{p}{(}\PYG{n}{phase}\PYG{p}{)}\PYG{p}{,} \PYG{l+m+mi}{1}\PYG{p}{)}\PYG{p}{)}\PYG{p}{;}
         \PYG{n}{bb}\PYG{p}{.}\PYG{n}{alpha}\PYG{p}{.}\PYG{n}{phase\PYGZus{}plusminus}\PYG{p}{(}\PYG{l+m+mi}{1}\PYG{p}{)} \PYG{p}{=} \PYG{n}{phase\PYGZus{}tolerance}\PYG{p}{;}
         \PYG{n}{bb}\PYG{p}{.}\PYG{n}{configure\PYGZus{}time\PYGZus{}port\PYGZus{}marker}\PYG{p}{(}\PYG{p}{(}\PYG{p}{[}\PYG{l+m+mi}{0}\PYG{p}{,} \PYG{l+m+mi}{1}\PYG{p}{,} \PYG{l+m+mi}{0}\PYG{p}{]}\PYG{p}{)}\PYG{p}{)}
         \PYG{n}{bb}\PYG{p}{.}\PYG{n}{min\PYGZus{}inter\PYGZus{}trig\PYGZus{}interval} \PYG{p}{=} \PYG{n}{minimium\PYGZus{}inter\PYGZus{}trigger\PYGZus{}interval}\PYG{p}{;}
         \PYG{n}{pause}\PYG{p}{(}\PYG{l+m+mf}{0.1}\PYG{p}{)}
         \PYG{n}{bb}\PYG{p}{.}\PYG{n}{arm}\PYG{p}{;}
     \PYG{n}{end}
     \PYG{l+s}{\PYGZpc{}} \PYG{l+s}{trigger} \PYG{l+s}{has} \PYG{l+s}{been} \PYG{l+s}{executed,} \PYG{l+s}{move} \PYG{l+s}{to} \PYG{l+s}{the} \PYG{l+s}{next} \PYG{l+s}{condition}
     \PYG{k}{if}\PYG{p}{(}\PYG{n}{bb}\PYG{p}{.}\PYG{n}{triggers\PYGZus{}remaining} \PYG{o}{==} \PYG{l+m+mi}{0}\PYG{p}{)}
         \PYG{n}{condition\PYGZus{}index} \PYG{p}{=} \PYG{n}{condition\PYGZus{}index} \PYG{o}{+} \PYG{l+m+mi}{1}\PYG{p}{;}
         \PYG{n}{bb}\PYG{p}{.}\PYG{n}{disarm}\PYG{p}{;}
         \PYG{n+nb}{disp} \PYG{p}{(}\PYG{p}{[}\PYG{l+s}{\PYGZsq{}}\PYG{l+s}{Triggered around \PYGZsq{}} \PYG{p}{(}\PYG{n}{num2str}\PYG{p}{(}\PYG{n}{rad2deg}\PYG{p}{(}\PYG{n}{bb}\PYG{p}{.}\PYG{n}{alpha}\PYG{p}{.}\PYG{n}{phase\PYGZus{}target}\PYG{p}{(}\PYG{l+m+mi}{1}\PYG{p}{)}\PYG{p}{)}\PYG{p}{)}\PYG{p}{)} \PYG{l+s}{\PYGZsq{}}\PYG{l+s}{ degrees Phase angle.\PYGZsq{}}\PYG{p}{]}\PYG{p}{)}
         \PYG{n}{pause}\PYG{p}{(}\PYG{n}{minimium\PYGZus{}inter\PYGZus{}trigger\PYGZus{}interval}\PYG{p}{)}
     \PYG{n}{end}
     \PYG{l+s}{pause(0.01)}\PYG{p}{;}
 \PYG{n}{end}

 \PYG{l+s}{\PYGZpc{}\PYGZpc{}} \PYG{l+s}{End}
 \PYG{n+nb}{disp} \PYG{p}{(}\PYG{l+s}{\PYGZsq{}}\PYG{l+s}{Protocol has been completed\PYGZsq{}}\PYG{p}{)}\PYG{p}{;}
\end{sphinxVerbatim}


\bigskip\hrule\bigskip



\section{Phase Triggered Plasticity Protocol}
\label{\detokenize{5_demo_scripts:phase-triggered-plasticity-protocol}}
This demo script uses bossdevice research and 2 different approaches to generate jittered open loop stimulus

Resources: 1) bossdevice Switched On, 2) bossdevice Open Source MATLAB API, 3) The stimulator is Switched On, External Trigger mode is turned on and the Stimulator is Enabled

\begin{sphinxVerbatim}[commandchars=\\\{\}]
\PYG{n}{no\PYGZus{}of\PYGZus{}trials}\PYG{p}{=}\PYG{l+m+mi}{25}\PYG{p}{;}
 \PYG{n}{no\PYGZus{}of\PYGZus{}pulses}\PYG{p}{=}\PYG{l+m+mi}{100}\PYG{p}{;}
 \PYG{n}{pulse\PYGZus{}frequency}\PYG{p}{=}\PYG{l+m+mi}{100}\PYG{p}{;} \PYG{c}{\PYGZpc{}Hz}
 \PYG{n}{minimium\PYGZus{}inter\PYGZus{}trigger\PYGZus{}interval}\PYG{p}{=}\PYG{l+m+mi}{5}\PYG{p}{;} \PYG{c}{\PYGZpc{}s}
 \PYG{n}{phase}\PYG{p}{=}\PYG{l+m+mi}{0}\PYG{p}{;} \PYG{c}{\PYGZpc{}peak}
 \PYG{n}{phase\PYGZus{}tolerance}\PYG{p}{=}\PYG{n+nb}{pi}\PYG{o}{/}\PYG{l+m+mi}{40}\PYG{p}{;}
 \PYG{n}{individual\PYGZus{}peak\PYGZus{}frequency}\PYG{p}{=}\PYG{l+m+mi}{11}\PYG{p}{;} \PYG{c}{\PYGZpc{} Hz}
 \PYG{n}{bandpassfilter\PYGZus{}order}\PYG{p}{=} \PYG{l+m+mi}{75}\PYG{p}{;}
 \PYG{n}{eeg\PYGZus{}channels}\PYG{p}{=}\PYG{l+m+mi}{5}\PYG{p}{;} \PYG{c}{\PYGZpc{}Assigning Number of channels as equivalent to Num of Channels streamed by Biosignal Processor}
 \PYG{n}{spatial\PYGZus{}filter\PYGZus{}weights}\PYG{p}{=}\PYG{p}{[}\PYG{l+m+mi}{1} \PYG{o}{\PYGZhy{}}\PYG{l+m+mf}{0.25} \PYG{o}{\PYGZhy{}}\PYG{l+m+mf}{0.25} \PYG{o}{\PYGZhy{}}\PYG{l+m+mf}{0.25} \PYG{o}{\PYGZhy{}}\PYG{l+m+mf}{0.25}\PYG{p}{]}\PYG{o}{\PYGZsq{}}\PYG{p}{;} \PYG{c}{\PYGZpc{}Column Vector of Spatial Filter Indexed wrt corrosponding Channels}

 \PYG{n}{time}\PYG{p}{=}\PYG{l+m+mi}{0}\PYG{p}{;}
 \PYG{n}{plasticity\PYGZus{}protocol\PYGZus{}sequence}\PYG{p}{=}\PYG{p}{[}\PYG{p}{]}\PYG{p}{;}

 \PYG{c}{\PYGZpc{}\PYGZpc{} Initializing bossdevice research API}
 \PYG{n}{bd}\PYG{p}{=}\PYG{n}{bossdevice}\PYG{p}{;}
 \PYG{n}{bd}\PYG{p}{.}\PYG{n}{sample\PYGZus{}and\PYGZus{}hold\PYGZus{}period}\PYG{p}{=}\PYG{l+m+mi}{0}\PYG{p}{;}
 \PYG{n}{bd}\PYG{p}{.}\PYG{n}{calibration\PYGZus{}mode} \PYG{p}{=} \PYG{l+s}{\PYGZsq{}}\PYG{l+s}{no\PYGZsq{}}\PYG{p}{;}
 \PYG{n}{bd}\PYG{p}{.}\PYG{n}{armed} \PYG{p}{=} \PYG{l+s}{\PYGZsq{}}\PYG{l+s}{no\PYGZsq{}}\PYG{p}{;}
 \PYG{n}{bd}\PYG{p}{.}\PYG{n}{sample\PYGZus{}and\PYGZus{}hold\PYGZus{}period}\PYG{p}{=}\PYG{l+m+mi}{0}\PYG{p}{;}
 \PYG{n}{bd}\PYG{p}{.}\PYG{n}{theta}\PYG{p}{.}\PYG{n}{ignore}\PYG{p}{;} \PYG{n}{pause}\PYG{p}{(}\PYG{l+m+mf}{0.1}\PYG{p}{)}
 \PYG{n}{bd}\PYG{p}{.}\PYG{n+nb}{beta}\PYG{p}{.}\PYG{n}{ignore}\PYG{p}{;} \PYG{n}{pause}\PYG{p}{(}\PYG{l+m+mf}{0.1}\PYG{p}{)}
 \PYG{n}{bd}\PYG{p}{.}\PYG{n}{alpha}\PYG{p}{.}\PYG{n}{ignore}\PYG{p}{;} \PYG{n}{pause}\PYG{p}{(}\PYG{l+m+mf}{0.1}\PYG{p}{)}
 \PYG{n}{bd}\PYG{p}{.}\PYG{n}{eeg\PYGZus{}channels}\PYG{p}{=}\PYG{n}{eeg\PYGZus{}channels}\PYG{p}{;}

 \PYG{c}{\PYGZpc{}\PYGZpc{} Preparing a Plasticity Protocol Seqeuence for bossdevice research}
 \PYG{n}{plasticity\PYGZus{}protocol\PYGZus{}sequence}\PYG{p}{(}\PYG{n}{no\PYGZus{}of\PYGZus{}pulses}\PYG{p}{,}\PYG{l+m+mi}{3}\PYG{p}{)}\PYG{p}{=}\PYG{l+m+mi}{0}\PYG{p}{;} \PYG{c}{\PYGZpc{}Pre filling the array}
 \PYG{n}{for} \PYG{l+s}{iPulse=1:no\PYGZus{}of\PYGZus{}pulses}
     \PYG{n}{time}\PYG{p}{=}\PYG{n}{time}\PYG{o}{+}\PYG{l+m+mf}{0.01}\PYG{p}{;}
     \PYG{n}{port}\PYG{p}{=}\PYG{l+m+mi}{1}\PYG{p}{;}
     \PYG{n}{marker}\PYG{p}{=}\PYG{n}{iPulse}\PYG{p}{;}
     \PYG{n}{plasticity\PYGZus{}protocol\PYGZus{}sequence}\PYG{p}{(}\PYG{n}{iPulse}\PYG{p}{,}\PYG{p}{:}\PYG{p}{)}\PYG{p}{=}\PYG{p}{[}\PYG{n}{time} \PYG{n}{port} \PYG{n}{marker}\PYG{p}{]}\PYG{p}{;}
 \PYG{n}{end}

 \PYG{l+s}{\PYGZpc{}\PYGZpc{}} \PYG{l+s}{Preparing} \PYG{l+s}{an} \PYG{l+s}{Individual} \PYG{l+s}{Peak} \PYG{l+s}{Frequency} \PYG{l+s}{based} \PYG{l+s}{Band} \PYG{l+s}{Pass} \PYG{l+s}{Filter} \PYG{l+s}{for} \PYG{l+s}{mu} \PYG{l+s}{Alpha}
 \PYG{n}{bpf\PYGZus{}fir\PYGZus{}coeffs} \PYG{p}{=} \PYG{n}{firls}\PYG{p}{(}\PYG{n}{bandpassfilter\PYGZus{}order}\PYG{p}{,} \PYG{p}{[}\PYG{l+m+mi}{0} \PYG{p}{(}\PYG{n}{individual\PYGZus{}peak\PYGZus{}frequency} \PYG{o}{+} \PYG{p}{[}\PYG{o}{\PYGZhy{}}\PYG{l+m+mi}{5} \PYG{o}{\PYGZhy{}}\PYG{l+m+mi}{2} \PYG{o}{+}\PYG{l+m+mi}{2} \PYG{o}{+}\PYG{l+m+mi}{5}\PYG{p}{]}\PYG{p}{)} \PYG{p}{(}\PYG{l+m+mi}{500}\PYG{o}{/}\PYG{l+m+mi}{2}\PYG{p}{)}\PYG{p}{]}\PYG{o}{/}\PYG{p}{(}\PYG{l+m+mi}{500}\PYG{o}{/}\PYG{l+m+mi}{2}\PYG{p}{)}\PYG{p}{,} \PYG{p}{[}\PYG{l+m+mi}{0} \PYG{l+m+mi}{0} \PYG{l+m+mi}{1} \PYG{l+m+mi}{1} \PYG{l+m+mi}{0} \PYG{l+m+mi}{0}\PYG{p}{]}\PYG{p}{,} \PYG{p}{[}\PYG{l+m+mi}{1} \PYG{l+m+mi}{1} \PYG{l+m+mi}{1}\PYG{p}{]} \PYG{p}{)}\PYG{p}{;}

 \PYG{c}{\PYGZpc{}\PYGZpc{} Setting Filters on bossdevice research}
 \PYG{n}{bd}\PYG{p}{.}\PYG{n}{spatial\PYGZus{}filter\PYGZus{}weights}\PYG{p}{=}\PYG{n}{spatial\PYGZus{}filter\PYGZus{}weights}\PYG{p}{;}
 \PYG{n}{bd}\PYG{p}{.}\PYG{n}{alpha}\PYG{p}{.}\PYG{n}{bpf\PYGZus{}fir\PYGZus{}coeffs} \PYG{p}{=} \PYG{n}{bpf\PYGZus{}fir\PYGZus{}coeffs}\PYG{p}{;}



 \PYG{c}{\PYGZpc{}\PYGZpc{} For plasticitz, we have the same condition, multiple times, we can run everything on the device:}
         \PYG{n}{bd}\PYG{p}{.}\PYG{n}{triggers\PYGZus{}remaining} \PYG{p}{=} \PYG{l+m+mi}{100}\PYG{p}{;}
         \PYG{n}{bd}\PYG{p}{.}\PYG{n}{alpha}\PYG{p}{.}\PYG{n}{phase\PYGZus{}target}\PYG{p}{(}\PYG{l+m+mi}{1}\PYG{p}{)} \PYG{p}{=} \PYG{n}{phase}\PYG{p}{;}
         \PYG{n}{bd}\PYG{p}{.}\PYG{n}{alpha}\PYG{p}{.}\PYG{n}{phase\PYGZus{}plusminus}\PYG{p}{(}\PYG{l+m+mi}{1}\PYG{p}{)} \PYG{p}{=} \PYG{n}{phase\PYGZus{}tolerance}\PYG{p}{;}
         \PYG{n}{bd}\PYG{p}{.}\PYG{n}{configure\PYGZus{}time\PYGZus{}port\PYGZus{}marker}\PYG{p}{(}\PYG{n}{plasticity\PYGZus{}protocol\PYGZus{}sequence}\PYG{p}{)}
         \PYG{n}{bd}\PYG{p}{.}\PYG{n}{min\PYGZus{}inter\PYGZus{}trig\PYGZus{}interval} \PYG{p}{=} \PYG{n}{minimium\PYGZus{}inter\PYGZus{}trigger\PYGZus{}interval}\PYG{p}{;}
         \PYG{n}{pause}\PYG{p}{(}\PYG{l+m+mf}{0.1}\PYG{p}{)}
         \PYG{n}{bd}\PYG{p}{.}\PYG{n}{arm}\PYG{p}{;}

         \PYG{n}{fprintf}\PYG{p}{(}\PYG{l+s}{\PYGZsq{}}\PYG{l+s}{\PYGZbs{}nSystem running, pulses remaining: \PYGZpc{}03i\PYGZsq{}}\PYG{p}{,} \PYG{n}{bd}\PYG{p}{.}\PYG{n}{triggers\PYGZus{}remaining}\PYG{p}{)}
         \PYG{k}{while} \PYG{p}{(}\PYG{n}{bd}\PYG{p}{.}\PYG{n}{triggers\PYGZus{}remaining} \PYG{o}{\PYGZgt{}} \PYG{l+m+mi}{0}\PYG{p}{)}
             \PYG{n}{fprintf}\PYG{p}{(}\PYG{l+s}{\PYGZsq{}}\PYG{l+s}{\PYGZbs{}b\PYGZbs{}b\PYGZbs{}b\PYGZpc{}03i\PYGZsq{}}\PYG{p}{,} \PYG{n}{bd}\PYG{p}{.}\PYG{n}{triggers\PYGZus{}remaining}\PYG{p}{)}\PYG{p}{;}
             \PYG{n}{pause}\PYG{p}{(}\PYG{l+m+mf}{0.1}\PYG{p}{)}
         \PYG{n}{end}
         \PYG{l+s}{fprintf(}\PYG{l+s}{\PYGZsq{}\PYGZbs{}b\PYGZbs{}b\PYGZbs{}bDone\PYGZbs{}n\PYGZsq{}}\PYG{l+s}{)}


 \PYG{c}{\PYGZpc{}\PYGZpc{} Controlling bossdevice research for mu Alpha Phase Locked Triggering \PYGZpc{} this could be for excitability, where we have interleaved different conditions}
 \PYG{n}{condition\PYGZus{}index}\PYG{p}{=}\PYG{l+m+mi}{0}\PYG{p}{;}
 \PYG{k}{while} \PYG{p}{(}\PYG{n}{condition\PYGZus{}index} \PYG{o}{\PYGZlt{}=} \PYG{n}{no\PYGZus{}of\PYGZus{}trials}\PYG{p}{)}
     \PYG{k}{if}\PYG{p}{(}\PYG{n}{strcmp}\PYG{p}{(}\PYG{n}{bd}\PYG{p}{.}\PYG{n}{armed}\PYG{p}{,} \PYG{l+s}{\PYGZsq{}}\PYG{l+s}{no\PYGZsq{}}\PYG{p}{)}\PYG{p}{)}
         \PYG{n}{bd}\PYG{p}{.}\PYG{n}{triggers\PYGZus{}remaining} \PYG{p}{=} \PYG{l+m+mi}{1}\PYG{p}{;}
         \PYG{n}{bd}\PYG{p}{.}\PYG{n}{alpha}\PYG{p}{.}\PYG{n}{phase\PYGZus{}target}\PYG{p}{(}\PYG{l+m+mi}{1}\PYG{p}{)} \PYG{p}{=} \PYG{n}{phase}\PYG{p}{;}
         \PYG{n}{bd}\PYG{p}{.}\PYG{n}{alpha}\PYG{p}{.}\PYG{n}{phase\PYGZus{}plusminus}\PYG{p}{(}\PYG{l+m+mi}{1}\PYG{p}{)} \PYG{p}{=} \PYG{n}{phase\PYGZus{}tolerance}\PYG{p}{;}
         \PYG{n}{bd}\PYG{p}{.}\PYG{n}{configure\PYGZus{}time\PYGZus{}port\PYGZus{}marker}\PYG{p}{(}\PYG{n}{plasticity\PYGZus{}protocol\PYGZus{}sequence}\PYG{p}{)}
         \PYG{n}{bd}\PYG{p}{.}\PYG{n}{min\PYGZus{}inter\PYGZus{}trig\PYGZus{}interval} \PYG{p}{=} \PYG{n}{minimium\PYGZus{}inter\PYGZus{}trigger\PYGZus{}interval}\PYG{p}{;}
         \PYG{n}{pause}\PYG{p}{(}\PYG{l+m+mf}{0.1}\PYG{p}{)}
         \PYG{n}{bd}\PYG{p}{.}\PYG{n}{arm}\PYG{p}{;}
     \PYG{n}{end}
     \PYG{l+s}{\PYGZpc{}} \PYG{l+s}{trigger} \PYG{l+s}{has} \PYG{l+s}{been} \PYG{l+s}{executed,} \PYG{l+s}{move} \PYG{l+s}{to} \PYG{l+s}{the} \PYG{l+s}{next} \PYG{l+s}{condition}
     \PYG{k}{if}\PYG{p}{(}\PYG{n}{bd}\PYG{p}{.}\PYG{n}{triggers\PYGZus{}remaining} \PYG{o}{==} \PYG{l+m+mi}{0}\PYG{p}{)}
         \PYG{n}{condition\PYGZus{}index} \PYG{p}{=} \PYG{n}{condition\PYGZus{}index} \PYG{o}{+} \PYG{l+m+mi}{1}\PYG{p}{;}
         \PYG{n}{bd}\PYG{p}{.}\PYG{n}{disarm}\PYG{p}{;}
         \PYG{n}{disp} \PYG{l+s}{Triggered!}
         \PYG{n}{pause}\PYG{p}{(}\PYG{n}{minimium\PYGZus{}inter\PYGZus{}trigger\PYGZus{}interval}\PYG{p}{)}
     \PYG{n}{end}
     \PYG{l+s}{pause(0.01)}\PYG{p}{;}
 \PYG{n}{end}

 \PYG{l+s}{\PYGZpc{}\PYGZpc{}} \PYG{l+s}{End}
 \PYG{n+nb}{disp} \PYG{p}{(}\PYG{l+s}{\PYGZsq{}}\PYG{l+s}{Plasticity Protocol has been completed\PYGZsq{}}\PYG{p}{)}\PYG{p}{;}
\end{sphinxVerbatim}


\bigskip\hrule\bigskip



\section{Real\sphinxhyphen{}Time Oscillation Amplitude Threshold Tracking}
\label{\detokenize{5_demo_scripts:real-time-oscillation-amplitude-threshold-tracking}}
This demo script uses bossdevice research and 2 different approaches to generate jittered open loop stimulus

Resources: 1) bossdevice Switched On, 2) bossdevice Open Source MATLAB API, 3) The stimulator is Switched On, External Trigger mode is turned on and the Stimulator is Enabled

\begin{sphinxVerbatim}[commandchars=\\\{\}]
\PYG{n}{no\PYGZus{}of\PYGZus{}trials}\PYG{p}{=}\PYG{l+m+mi}{25}\PYG{p}{;}
\PYG{n}{minimium\PYGZus{}inter\PYGZus{}trigger\PYGZus{}interval}\PYG{p}{=}\PYG{l+m+mi}{4}\PYG{p}{;} \PYG{c}{\PYGZpc{}s}
\PYG{n}{phase}\PYG{p}{=}\PYG{l+m+mi}{0}\PYG{p}{;} \PYG{c}{\PYGZpc{}[positive]}
\PYG{n}{phase\PYGZus{}tolerance}\PYG{p}{=}\PYG{n+nb}{pi}\PYG{o}{/}\PYG{l+m+mi}{40}\PYG{p}{;}
\PYG{n}{amplitude\PYGZus{}threshold}\PYG{p}{=}\PYG{p}{[}\PYG{l+m+mi}{25} \PYG{l+m+mi}{75}\PYG{p}{]}\PYG{p}{;} \PYG{c}{\PYGZpc{}[min max] in percentile}
\PYG{n}{amplitude\PYGZus{}assignment\PYGZus{}period}\PYG{p}{=}\PYG{l+m+mi}{60}\PYG{o}{*}\PYG{l+m+mi}{2}\PYG{p}{;} \PYG{c}{\PYGZpc{}s}
\PYG{n}{individual\PYGZus{}peak\PYGZus{}frequency}\PYG{p}{=}\PYG{l+m+mi}{11}\PYG{p}{;} \PYG{c}{\PYGZpc{} Hz}
\PYG{n}{bandpassfilter\PYGZus{}order}\PYG{p}{=} \PYG{l+m+mi}{75}\PYG{p}{;}
\PYG{n}{eeg\PYGZus{}channels}\PYG{p}{=}\PYG{l+m+mi}{5}\PYG{p}{;} \PYG{c}{\PYGZpc{}Assigning Number of channels as equivalent to Num of Channels streamed by Biosignal Processor}
\PYG{n}{spatial\PYGZus{}filter\PYGZus{}weights}\PYG{p}{=}\PYG{p}{[}\PYG{l+m+mi}{1} \PYG{o}{\PYGZhy{}}\PYG{l+m+mf}{0.25} \PYG{o}{\PYGZhy{}}\PYG{l+m+mf}{0.25} \PYG{o}{\PYGZhy{}}\PYG{l+m+mf}{0.25} \PYG{o}{\PYGZhy{}}\PYG{l+m+mf}{0.25}\PYG{p}{]}\PYG{o}{\PYGZsq{}}\PYG{p}{;} \PYG{c}{\PYGZpc{}Column Vector of Spatial Filter Indexed wrt corrosponding Channels}
\PYG{n}{time}\PYG{p}{=}\PYG{l+m+mi}{0}\PYG{p}{;}
\PYG{n}{plasticity\PYGZus{}protocol\PYGZus{}sequence}\PYG{p}{=}\PYG{p}{[}\PYG{p}{]}\PYG{p}{;}

\PYG{c}{\PYGZpc{}\PYGZpc{} Initializing bossdevice research API}
\PYG{n}{bd}\PYG{p}{=}\PYG{n}{bossdevice}\PYG{p}{;}
\PYG{n}{bd}\PYG{p}{.}\PYG{n}{sample\PYGZus{}and\PYGZus{}hold\PYGZus{}period}\PYG{p}{=}\PYG{l+m+mi}{0}\PYG{p}{;}
\PYG{n}{bd}\PYG{p}{.}\PYG{n}{calibration\PYGZus{}mode} \PYG{p}{=} \PYG{l+s}{\PYGZsq{}}\PYG{l+s}{no\PYGZsq{}}\PYG{p}{;}
\PYG{n}{bd}\PYG{p}{.}\PYG{n}{armed} \PYG{p}{=} \PYG{l+s}{\PYGZsq{}}\PYG{l+s}{no\PYGZsq{}}\PYG{p}{;}
\PYG{n}{bd}\PYG{p}{.}\PYG{n}{sample\PYGZus{}and\PYGZus{}hold\PYGZus{}period}\PYG{p}{=}\PYG{l+m+mi}{0}\PYG{p}{;}
\PYG{n}{bd}\PYG{p}{.}\PYG{n}{theta}\PYG{p}{.}\PYG{n}{ignore}\PYG{p}{;} \PYG{n}{pause}\PYG{p}{(}\PYG{l+m+mf}{0.1}\PYG{p}{)}
\PYG{n}{bd}\PYG{p}{.}\PYG{n+nb}{beta}\PYG{p}{.}\PYG{n}{ignore}\PYG{p}{;} \PYG{n}{pause}\PYG{p}{(}\PYG{l+m+mf}{0.1}\PYG{p}{)}
\PYG{n}{bd}\PYG{p}{.}\PYG{n}{alpha}\PYG{p}{.}\PYG{n}{ignore}\PYG{p}{;} \PYG{n}{pause}\PYG{p}{(}\PYG{l+m+mf}{0.1}\PYG{p}{)}
\PYG{n}{bd}\PYG{p}{.}\PYG{n}{eeg\PYGZus{}channels}\PYG{p}{=}\PYG{n}{eeg\PYGZus{}channels}\PYG{p}{;}

\PYG{c}{\PYGZpc{}\PYGZpc{} Preparing an Individual Peak Frequency based Band Pass Filter for mu Alpha}
\PYG{n}{bpf\PYGZus{}fir\PYGZus{}coeffs} \PYG{p}{=} \PYG{n}{firls}\PYG{p}{(}\PYG{n}{bandpassfilter\PYGZus{}order}\PYG{p}{,} \PYG{p}{[}\PYG{l+m+mi}{0} \PYG{p}{(}\PYG{n}{individual\PYGZus{}peak\PYGZus{}frequency} \PYG{o}{+} \PYG{p}{[}\PYG{o}{\PYGZhy{}}\PYG{l+m+mi}{5} \PYG{o}{\PYGZhy{}}\PYG{l+m+mi}{2} \PYG{o}{+}\PYG{l+m+mi}{2} \PYG{o}{+}\PYG{l+m+mi}{5}\PYG{p}{]}\PYG{p}{)} \PYG{p}{(}\PYG{l+m+mi}{500}\PYG{o}{/}\PYG{l+m+mi}{2}\PYG{p}{)}\PYG{p}{]}\PYG{o}{/}\PYG{p}{(}\PYG{l+m+mi}{500}\PYG{o}{/}\PYG{l+m+mi}{2}\PYG{p}{)}\PYG{p}{,} \PYG{p}{[}\PYG{l+m+mi}{0} \PYG{l+m+mi}{0} \PYG{l+m+mi}{1} \PYG{l+m+mi}{1} \PYG{l+m+mi}{0} \PYG{l+m+mi}{0}\PYG{p}{]}\PYG{p}{,} \PYG{p}{[}\PYG{l+m+mi}{1} \PYG{l+m+mi}{1} \PYG{l+m+mi}{1}\PYG{p}{]} \PYG{p}{)}\PYG{p}{;}

\PYG{c}{\PYGZpc{}\PYGZpc{} Setting Filters on bossdevice research}
\PYG{n}{bd}\PYG{p}{.}\PYG{n}{spatial\PYGZus{}filter\PYGZus{}weights}\PYG{p}{=}\PYG{n}{spatial\PYGZus{}filter\PYGZus{}weights}\PYG{p}{;}
\PYG{n}{bd}\PYG{p}{.}\PYG{n}{alpha}\PYG{p}{.}\PYG{n}{bpf\PYGZus{}fir\PYGZus{}coeffs} \PYG{p}{=} \PYG{n}{bpf\PYGZus{}fir\PYGZus{}coeffs}\PYG{p}{;}

\PYG{c}{\PYGZpc{}\PYGZpc{} Configuring Real\PYGZhy{}Time Scopes for Amplitude Tracking}
\PYG{n}{AMP\PYGZus{}TRACING\PYGZus{}SCOPES\PYGZus{}IDS} \PYG{p}{=} \PYG{p}{[}\PYG{l+m+mi}{101} \PYG{l+m+mi}{102}\PYG{p}{]}\PYG{p}{;}

\PYG{c}{\PYGZpc{} remove any pre\PYGZhy{}existing scopes with these ids}
\PYG{k}{for} \PYG{n}{id} \PYG{p}{=} \PYG{n}{AMP\PYGZus{}TRACING\PYGZus{}SCOPES\PYGZus{}IDS}
    \PYG{k}{if}\PYG{p}{(}\PYG{n+nb}{find}\PYG{p}{(}\PYG{n}{bd}\PYG{p}{.}\PYG{n}{tg}\PYG{p}{.}\PYG{n}{Scopes} \PYG{o}{==} \PYG{n}{id}\PYG{p}{)}\PYG{p}{)}
        \PYG{n}{fprintf}\PYG{p}{(}\PYG{l+s}{\PYGZsq{}}\PYG{l+s}{\PYGZbs{}nRemoving scope \PYGZpc{}i\PYGZsq{}}\PYG{p}{,} \PYG{n}{id}\PYG{p}{)}
        \PYG{n}{remscope}\PYG{p}{(}\PYG{n}{bd}\PYG{p}{.}\PYG{n}{tg}\PYG{p}{,} \PYG{n}{id}\PYG{p}{)}\PYG{p}{;}
    \PYG{n}{end}
\PYG{l+s}{end}

\PYG{n}{sig\PYGZus{}id\PYGZus{}amp} \PYG{p}{=} \PYG{n}{getsignalid}\PYG{p}{(}\PYG{n}{bd}\PYG{p}{.}\PYG{n}{tg}\PYG{p}{,} \PYG{l+s}{\PYGZsq{}}\PYG{l+s}{OSC/alpha/IA\PYGZsq{}}\PYG{p}{)}\PYG{p}{;} \PYG{c}{\PYGZpc{}amplitude}
\PYG{n}{sig\PYGZus{}id\PYGZus{}qly} \PYG{p}{=} \PYG{n}{getsignalid}\PYG{p}{(}\PYG{n}{bd}\PYG{p}{.}\PYG{n}{tg}\PYG{p}{,} \PYG{l+s}{\PYGZsq{}}\PYG{l+s}{QLY/Logical Operator2\PYGZsq{}}\PYG{p}{)}\PYG{p}{;} \PYG{c}{\PYGZpc{}eeg\PYGZus{}is\PYGZus{}clean}

\PYG{n}{sc} \PYG{p}{=} \PYG{n}{addscope}\PYG{p}{(}\PYG{n}{bd}\PYG{p}{.}\PYG{n}{tg}\PYG{p}{,} \PYG{l+s}{\PYGZsq{}}\PYG{l+s}{host\PYGZsq{}}\PYG{p}{,} \PYG{n}{AMP\PYGZus{}TRACING\PYGZus{}SCOPES\PYGZus{}IDS}\PYG{p}{)}\PYG{p}{;}
\PYG{n}{addsignal}\PYG{p}{(}\PYG{n}{sc}\PYG{p}{,} \PYG{p}{[}\PYG{n}{sig\PYGZus{}id\PYGZus{}amp} \PYG{n}{sig\PYGZus{}id\PYGZus{}qly}\PYG{p}{]}\PYG{p}{)}\PYG{p}{;}

\PYG{n}{sc}\PYG{p}{(}\PYG{l+m+mi}{1}\PYG{p}{)}\PYG{p}{.}\PYG{n}{NumSamples} \PYG{p}{=} \PYG{l+m+mi}{500}\PYG{p}{;}
\PYG{n}{sc}\PYG{p}{(}\PYG{l+m+mi}{1}\PYG{p}{)}\PYG{p}{.}\PYG{n}{Decimation} \PYG{p}{=} \PYG{l+m+mi}{10}\PYG{p}{;}
\PYG{n}{sc}\PYG{p}{(}\PYG{l+m+mi}{1}\PYG{p}{)}\PYG{p}{.}\PYG{n}{TriggerSample} \PYG{p}{=} \PYG{o}{\PYGZhy{}}\PYG{l+m+mi}{1}\PYG{p}{;}

\PYG{n}{sc}\PYG{p}{(}\PYG{l+m+mi}{2}\PYG{p}{)}\PYG{p}{.}\PYG{n}{NumSamples} \PYG{p}{=} \PYG{l+m+mi}{500}\PYG{p}{;}
\PYG{n}{sc}\PYG{p}{(}\PYG{l+m+mi}{2}\PYG{p}{)}\PYG{p}{.}\PYG{n}{Decimation} \PYG{p}{=} \PYG{l+m+mi}{10}\PYG{p}{;}
\PYG{n}{sc}\PYG{p}{(}\PYG{l+m+mi}{2}\PYG{p}{)}\PYG{p}{.}\PYG{n}{TriggerSample} \PYG{p}{=} \PYG{o}{\PYGZhy{}}\PYG{l+m+mi}{1}\PYG{p}{;}

\PYG{n}{sc}\PYG{p}{(}\PYG{l+m+mi}{1}\PYG{p}{)}\PYG{p}{.}\PYG{n}{TriggerMode} \PYG{p}{=} \PYG{l+s}{\PYGZsq{}}\PYG{l+s}{Scope\PYGZsq{}}\PYG{p}{;}
\PYG{n}{sc}\PYG{p}{(}\PYG{l+m+mi}{1}\PYG{p}{)}\PYG{p}{.}\PYG{n}{TriggerScope} \PYG{p}{=} \PYG{n}{AMP\PYGZus{}TRACING\PYGZus{}SCOPES\PYGZus{}IDS}\PYG{p}{(}\PYG{l+m+mi}{2}\PYG{p}{)}\PYG{p}{;}

\PYG{n}{sc}\PYG{p}{(}\PYG{l+m+mi}{2}\PYG{p}{)}\PYG{p}{.}\PYG{n}{TriggerMode} \PYG{p}{=} \PYG{l+s}{\PYGZsq{}}\PYG{l+s}{Scope\PYGZsq{}}\PYG{p}{;}
\PYG{n}{sc}\PYG{p}{(}\PYG{l+m+mi}{2}\PYG{p}{)}\PYG{p}{.}\PYG{n}{TriggerScope} \PYG{p}{=} \PYG{n}{AMP\PYGZus{}TRACING\PYGZus{}SCOPES\PYGZus{}IDS}\PYG{p}{(}\PYG{l+m+mi}{1}\PYG{p}{)}\PYG{p}{;}

\PYG{n}{start}\PYG{p}{(}\PYG{n}{sc}\PYG{p}{)}\PYG{p}{;} \PYG{c}{\PYGZpc{} now they are ready for being triggered}

\PYG{n}{activeScope} \PYG{p}{=} \PYG{l+m+mi}{1}\PYG{p}{;}
\PYG{n}{mAmplitudeScopeCircBufTotalBlocks} \PYG{p}{=} \PYG{n}{amplitude\PYGZus{}assignment\PYGZus{}period}\PYG{p}{;}
\PYG{n}{mAmplitudeScopeCircBufCurrentBlock} \PYG{p}{=} \PYG{l+m+mi}{1}\PYG{p}{;}
\PYG{n}{mAmplitudeScopeCircBuf} \PYG{p}{=} \PYG{p}{[}\PYG{p}{]}\PYG{p}{;}
\PYG{n}{hAmplitudeHistoryAxes} \PYG{p}{=} \PYG{n}{subplot}\PYG{p}{(}\PYG{l+m+mi}{1}\PYG{p}{,}\PYG{l+m+mi}{2}\PYG{p}{,}\PYG{l+m+mi}{1}\PYG{p}{)}\PYG{p}{;}
\PYG{n}{hAmplitudeDistributionAxes} \PYG{p}{=} \PYG{n}{subplot}\PYG{p}{(}\PYG{l+m+mi}{1}\PYG{p}{,}\PYG{l+m+mi}{2}\PYG{p}{,}\PYG{l+m+mi}{2}\PYG{p}{)}\PYG{p}{;}

\PYG{n}{trigger}\PYG{p}{(}\PYG{n}{sc}\PYG{p}{(}\PYG{n}{activeScope}\PYG{p}{)}\PYG{p}{)}\PYG{p}{;}

\PYG{c}{\PYGZpc{}\PYGZpc{} Controlling bossdevice research for mu Alpha Phase Locked Triggering}
\PYG{n}{condition\PYGZus{}index}\PYG{p}{=}\PYG{l+m+mi}{0}\PYG{p}{;}
\PYG{k}{while} \PYG{p}{(}\PYG{n}{condition\PYGZus{}index} \PYG{o}{\PYGZlt{}=} \PYG{n}{no\PYGZus{}of\PYGZus{}trials}\PYG{p}{)}
    \PYG{k}{if} \PYG{p}{(}\PYG{n}{strcmp}\PYG{p}{(}\PYG{n}{sc}\PYG{p}{(}\PYG{n}{activeScope}\PYG{p}{)}\PYG{p}{.}\PYG{n}{Status}\PYG{p}{,} \PYG{l+s}{\PYGZsq{}}\PYG{l+s}{Finished\PYGZsq{}}\PYG{p}{)} \PYG{o}{||} \PYG{c}{...}
            \PYG{n}{strcmp}\PYG{p}{(}\PYG{n}{sc}\PYG{p}{(}\PYG{n}{activeScope}\PYG{p}{)}\PYG{p}{.}\PYG{n}{Status}\PYG{p}{,} \PYG{l+s}{\PYGZsq{}}\PYG{l+s}{Interrupted\PYGZsq{}}\PYG{p}{)}\PYG{p}{)}

        \PYG{n}{time} \PYG{p}{=} \PYG{n}{sc}\PYG{p}{(}\PYG{n}{activeScope}\PYG{p}{)}\PYG{p}{.}\PYG{n}{Time}\PYG{p}{;}
        \PYG{n}{data} \PYG{p}{=} \PYG{n}{sc}\PYG{p}{(}\PYG{n}{activeScope}\PYG{p}{)}\PYG{p}{.}\PYG{n}{Data}\PYG{p}{;}
        \PYG{n}{plot}\PYG{p}{(}\PYG{n}{hAmplitudeHistoryAxes}\PYG{p}{,} \PYG{n}{time}\PYG{p}{,} \PYG{n}{data}\PYG{p}{(}\PYG{p}{:}\PYG{p}{,}\PYG{l+m+mi}{1}\PYG{p}{)}\PYG{p}{)}\PYG{p}{;}

        \PYG{n}{fprintf}\PYG{p}{(}\PYG{p}{[}\PYG{l+s}{\PYGZsq{}}\PYG{l+s}{Restarting Scope \PYGZsq{}} \PYG{n}{num2str}\PYG{p}{(}\PYG{n}{activeScope}\PYG{p}{)}\PYG{p}{]}\PYG{p}{)}\PYG{p}{;}

        \PYG{c}{\PYGZpc{} Restart this scope.}
        \PYG{n}{start}\PYG{p}{(}\PYG{n}{sc}\PYG{p}{(}\PYG{n}{activeScope}\PYG{p}{)}\PYG{p}{)}\PYG{p}{;}

        \PYG{c}{\PYGZpc{} Switch to the next scope.}
        \PYG{k}{if}\PYG{p}{(}\PYG{n}{activeScope} \PYG{o}{==} \PYG{l+m+mi}{1}\PYG{p}{)}
            \PYG{n}{activeScope} \PYG{p}{=} \PYG{l+m+mi}{2}\PYG{p}{;}
        \PYG{n}{else}
            \PYG{l+s}{activeScope} \PYG{l+s}{=} \PYG{l+s}{1}\PYG{p}{;}
        \PYG{n}{end}

        \PYG{l+s}{\PYGZpc{}} \PYG{l+s}{append} \PYG{l+s}{data} \PYG{l+s}{in} \PYG{l+s}{circular} \PYG{l+s}{buffer}
        \PYG{n}{mAmplitudeScopeCircBuf}\PYG{p}{\PYGZob{}}\PYG{n}{mAmplitudeScopeCircBufCurrentBlock}\PYG{p}{\PYGZcb{}} \PYG{p}{=} \PYG{n}{data}\PYG{o}{\PYGZsq{}}\PYG{p}{;}

        \PYG{n}{maxmindata} \PYG{p}{=} \PYG{n}{cell2mat}\PYG{p}{(}\PYG{n}{cellfun}\PYG{p}{(}\PYG{p}{@}\PYG{p}{(}\PYG{n}{data}\PYG{p}{)} \PYG{n}{quantile}\PYG{p}{(}\PYG{n}{data}\PYG{p}{(}\PYG{l+m+mi}{1}\PYG{p}{,} \PYG{n}{data}\PYG{p}{(}\PYG{l+m+mi}{2}\PYG{p}{,}\PYG{p}{:}\PYG{p}{)} \PYG{o}{==} \PYG{l+m+mi}{1}\PYG{p}{)}\PYG{p}{,} \PYG{p}{[}\PYG{n}{amplitude\PYGZus{}threshold}\PYG{p}{(}\PYG{l+m+mi}{1}\PYG{p}{)}\PYG{o}{/}\PYG{l+m+mi}{100} \PYG{n}{amplitude\PYGZus{}threshold}\PYG{p}{(}\PYG{l+m+mi}{2}\PYG{p}{)}\PYG{o}{/}\PYG{l+m+mi}{100}\PYG{p}{]}\PYG{p}{)}\PYG{o}{\PYGZsq{}}\PYG{p}{,} \PYG{n}{mAmplitudeScopeCircBuf}\PYG{p}{,} \PYG{l+s}{\PYGZsq{}}\PYG{l+s}{UniformOutput\PYGZsq{}}\PYG{p}{,} \PYG{n}{false}\PYG{p}{)}\PYG{p}{)}\PYG{o}{\PYGZsq{}}\PYG{p}{;}
        \PYG{n}{maxmindata} \PYG{p}{=} \PYG{n+nb}{circshift}\PYG{p}{(}\PYG{n}{maxmindata}\PYG{p}{,} \PYG{n}{mAmplitudeScopeCircBufCurrentBlock}\PYG{p}{)}\PYG{p}{;}
        \PYG{n}{plot}\PYG{p}{(}\PYG{n}{hAmplitudeHistoryAxes}\PYG{p}{,} \PYG{n}{maxmindata}\PYG{p}{)}
        \PYG{n}{xlim}\PYG{p}{(}\PYG{n}{hAmplitudeHistoryAxes}\PYG{p}{,} \PYG{p}{[}\PYG{l+m+mi}{1} \PYG{n}{mAmplitudeScopeCircBufTotalBlocks}\PYG{p}{]}\PYG{p}{)}
        \PYG{n}{set}\PYG{p}{(}\PYG{n}{hAmplitudeHistoryAxes}\PYG{p}{,} \PYG{l+s}{\PYGZsq{}}\PYG{l+s}{Xdir\PYGZsq{}}\PYG{p}{,} \PYG{l+s}{\PYGZsq{}}\PYG{l+s}{reverse\PYGZsq{}}\PYG{p}{)}

        \PYG{n}{circular\PYGZus{}buffer\PYGZus{}data} \PYG{p}{=} \PYG{n}{cell2mat}\PYG{p}{(}\PYG{n}{mAmplitudeScopeCircBuf}\PYG{p}{)}\PYG{p}{;}

        \PYG{c}{\PYGZpc{} Switch to the next data block}
        \PYG{k}{if}\PYG{p}{(}\PYG{n}{mAmplitudeScopeCircBufCurrentBlock} \PYG{o}{\PYGZlt{}} \PYG{n}{mAmplitudeScopeCircBufTotalBlocks}\PYG{p}{)}
            \PYG{n}{mAmplitudeScopeCircBufCurrentBlock} \PYG{p}{=} \PYG{n}{mAmplitudeScopeCircBufCurrentBlock} \PYG{o}{+} \PYG{l+m+mi}{1}\PYG{p}{;}
        \PYG{n}{else}
            \PYG{l+s}{mAmplitudeScopeCircBufCurrentBlock} \PYG{l+s}{=} \PYG{l+s}{1}\PYG{p}{;}
        \PYG{n}{end}

        \PYG{l+s}{\PYGZpc{}tic}

        \PYG{c}{\PYGZpc{} remove post\PYGZhy{}stimulus data}
        \PYG{n}{amplitude\PYGZus{}clean} \PYG{p}{=} \PYG{n}{circular\PYGZus{}buffer\PYGZus{}data}\PYG{p}{(}\PYG{l+m+mi}{1}\PYG{p}{,} \PYG{n}{circular\PYGZus{}buffer\PYGZus{}data}\PYG{p}{(}\PYG{l+m+mi}{2}\PYG{p}{,}\PYG{p}{:}\PYG{p}{)} \PYG{o}{==} \PYG{l+m+mi}{1}\PYG{p}{)}\PYG{p}{;}

        \PYG{c}{\PYGZpc{} calculate percentiles}
        \PYG{n}{amplitude\PYGZus{}sorted} \PYG{p}{=} \PYG{n}{sort}\PYG{p}{(}\PYG{n}{amplitude\PYGZus{}clean}\PYG{p}{)}\PYG{p}{;}
        \PYG{n}{plot}\PYG{p}{(}\PYG{n}{hAmplitudeDistributionAxes}\PYG{p}{,} \PYG{n}{amplitude\PYGZus{}sorted}\PYG{p}{)}

        \PYG{n}{amp\PYGZus{}lower} \PYG{p}{=} \PYG{n}{quantile}\PYG{p}{(}\PYG{n}{amplitude\PYGZus{}clean}\PYG{p}{,} \PYG{n}{amplitude\PYGZus{}threshold}\PYG{p}{(}\PYG{l+m+mi}{1}\PYG{p}{)}\PYG{o}{/}\PYG{l+m+mi}{100}\PYG{p}{)}\PYG{p}{;} \PYG{c}{\PYGZpc{} TODO: INCLUDE THIS IN INFO STRUCT}
        \PYG{n}{amp\PYGZus{}upper} \PYG{p}{=} \PYG{n}{quantile}\PYG{p}{(}\PYG{n}{amplitude\PYGZus{}clean}\PYG{p}{,} \PYG{n}{amplitude\PYGZus{}threshold}\PYG{p}{(}\PYG{l+m+mi}{2}\PYG{p}{)}\PYG{o}{/}\PYG{l+m+mi}{100}\PYG{p}{)}\PYG{p}{;} \PYG{c}{\PYGZpc{} TODO: INCLUDE THIS IN INFO STRUCT}

        \PYG{n}{hold}\PYG{p}{(}\PYG{n}{hAmplitudeDistributionAxes}\PYG{p}{,} \PYG{l+s}{\PYGZsq{}}\PYG{l+s}{on\PYGZsq{}}\PYG{p}{)}
        \PYG{n}{plot}\PYG{p}{(}\PYG{n}{hAmplitudeDistributionAxes}\PYG{p}{,} \PYG{p}{[}\PYG{l+m+mi}{1} \PYG{n+nb}{length}\PYG{p}{(}\PYG{n}{amplitude\PYGZus{}clean}\PYG{p}{)}\PYG{p}{]}\PYG{p}{,} \PYG{p}{[}\PYG{n}{amp\PYGZus{}lower} \PYG{n}{amp\PYGZus{}upper}\PYG{p}{;} \PYG{n}{amp\PYGZus{}lower} \PYG{l+s}{amp\PYGZus{}upper])}\PYG{p}{;}
        \PYG{n}{hold}\PYG{p}{(}\PYG{n}{hAmplitudeDistributionAxes}\PYG{p}{,} \PYG{l+s}{\PYGZsq{}}\PYG{l+s}{off\PYGZsq{}}\PYG{p}{)}

        \PYG{n}{if} \PYG{l+s}{length(amplitude\PYGZus{}clean)} \PYG{l+s}{\PYGZgt{}} \PYG{l+s}{1}
            \PYG{n}{xlim}\PYG{p}{(}\PYG{n}{hAmplitudeDistributionAxes}\PYG{p}{,} \PYG{p}{[}\PYG{l+m+mi}{1} \PYG{n+nb}{length}\PYG{p}{(}\PYG{n}{amplitude\PYGZus{}clean}\PYG{p}{)}\PYG{p}{]}\PYG{p}{)}\PYG{p}{;}
        \PYG{n}{end}
        \PYG{l+s}{if} \PYG{l+s}{(amplitude\PYGZus{}sorted(end)} \PYG{l+s}{\PYGZgt{}} \PYG{l+s}{amplitude\PYGZus{}sorted(1))}
            \PYG{n}{ylim}\PYG{p}{(}\PYG{n}{hAmplitudeDistributionAxes}\PYG{p}{,} \PYG{p}{[}\PYG{n}{amplitude\PYGZus{}sorted}\PYG{p}{(}\PYG{l+m+mi}{1}\PYG{p}{)} \PYG{n}{amplitude\PYGZus{}sorted}\PYG{p}{(}\PYG{k}{end}\PYG{p}{)}\PYG{p}{]}\PYG{p}{)}\PYG{p}{;}
        \PYG{n}{end}

        \PYG{l+s}{\PYGZpc{}toc}

        \PYG{c}{\PYGZpc{} set amplitude threshold}
        \PYG{n}{mDbsp}\PYG{p}{.}\PYG{n}{alpha}\PYG{p}{.}\PYG{n}{amplitude\PYGZus{}min}\PYG{p}{(}\PYG{l+m+mi}{1}\PYG{p}{)} \PYG{p}{=} \PYG{n}{amp\PYGZus{}lower}\PYG{p}{;}
        \PYG{n}{mDbsp}\PYG{p}{.}\PYG{n}{alpha}\PYG{p}{.}\PYG{n}{amplitude\PYGZus{}max}\PYG{p}{(}\PYG{l+m+mi}{1}\PYG{p}{)} \PYG{p}{=} \PYG{n}{amp\PYGZus{}upper}\PYG{p}{;}
        \PYG{n}{bd}\PYG{p}{.}\PYG{n}{alpha}\PYG{p}{.}\PYG{n}{amplitude\PYGZus{}min}\PYG{p}{(}\PYG{l+m+mi}{1}\PYG{p}{)}\PYG{p}{=}\PYG{n}{amp\PYGZus{}lower}\PYG{p}{;}
        \PYG{n}{bd}\PYG{p}{.}\PYG{n}{alpha}\PYG{p}{.}\PYG{n}{amplitude\PYGZus{}max}\PYG{p}{(}\PYG{l+m+mi}{1}\PYG{p}{)}\PYG{p}{=}\PYG{n}{amp\PYGZus{}upper}\PYG{p}{;}
        \PYG{n}{title}\PYG{p}{(}\PYG{n}{hAmplitudeDistributionAxes}\PYG{p}{,} \PYG{p}{[}\PYG{l+s}{\PYGZsq{}}\PYG{l+s}{Min Amplitude: \PYGZsq{}}\PYG{p}{,} \PYG{n}{num2str}\PYG{p}{(}\PYG{n}{amp\PYGZus{}lower}\PYG{p}{)}\PYG{p}{]}\PYG{p}{)}\PYG{p}{;}

    \PYG{n}{end} \PYG{l+s}{\PYGZpc{}} \PYG{l+s}{handle} \PYG{l+s}{the} \PYG{l+s}{amplitude} \PYG{l+s}{tracking}
    \PYG{k}{if}\PYG{p}{(}\PYG{n}{strcmp}\PYG{p}{(}\PYG{n}{bd}\PYG{p}{.}\PYG{n}{armed}\PYG{p}{,} \PYG{l+s}{\PYGZsq{}}\PYG{l+s}{no\PYGZsq{}}\PYG{p}{)}\PYG{p}{)}
        \PYG{n}{bd}\PYG{p}{.}\PYG{n}{triggers\PYGZus{}remaining} \PYG{p}{=} \PYG{l+m+mi}{1}\PYG{p}{;}
        \PYG{n}{bd}\PYG{p}{.}\PYG{n}{alpha}\PYG{p}{.}\PYG{n}{phase\PYGZus{}target}\PYG{p}{(}\PYG{l+m+mi}{1}\PYG{p}{)} \PYG{p}{=} \PYG{n}{phase}\PYG{p}{(}\PYG{n}{randi}\PYG{p}{(}\PYG{l+m+mi}{1}\PYG{p}{:}\PYG{n+nb}{numel}\PYG{p}{(}\PYG{n}{phase}\PYG{p}{)}\PYG{p}{,} \PYG{l+m+mi}{1}\PYG{p}{)}\PYG{p}{)}\PYG{p}{;}
        \PYG{n}{bd}\PYG{p}{.}\PYG{n}{alpha}\PYG{p}{.}\PYG{n}{phase\PYGZus{}plusminus}\PYG{p}{(}\PYG{l+m+mi}{1}\PYG{p}{)} \PYG{p}{=} \PYG{n}{phase\PYGZus{}tolerance}\PYG{p}{;}
        \PYG{n}{bd}\PYG{p}{.}\PYG{n}{configure\PYGZus{}time\PYGZus{}port\PYGZus{}marker}\PYG{p}{(}\PYG{p}{(}\PYG{p}{[}\PYG{l+m+mi}{0}\PYG{p}{,} \PYG{l+m+mi}{1}\PYG{p}{,} \PYG{l+m+mi}{0}\PYG{p}{]}\PYG{p}{)}\PYG{p}{)}
        \PYG{n}{bd}\PYG{p}{.}\PYG{n}{min\PYGZus{}inter\PYGZus{}trig\PYGZus{}interval} \PYG{p}{=} \PYG{n}{minimium\PYGZus{}inter\PYGZus{}trigger\PYGZus{}interval}\PYG{p}{;}
        \PYG{n}{pause}\PYG{p}{(}\PYG{l+m+mf}{0.1}\PYG{p}{)}
        \PYG{n}{bd}\PYG{p}{.}\PYG{n}{arm}\PYG{p}{;}
    \PYG{n}{end}
    \PYG{l+s}{\PYGZpc{}} \PYG{l+s}{trigger} \PYG{l+s}{has} \PYG{l+s}{been} \PYG{l+s}{executed,} \PYG{l+s}{move} \PYG{l+s}{to} \PYG{l+s}{the} \PYG{l+s}{next} \PYG{l+s}{condition}
    \PYG{k}{if}\PYG{p}{(}\PYG{n}{bd}\PYG{p}{.}\PYG{n}{triggers\PYGZus{}remaining} \PYG{o}{==} \PYG{l+m+mi}{0}\PYG{p}{)}
        \PYG{n}{condition\PYGZus{}index} \PYG{p}{=} \PYG{n}{condition\PYGZus{}index} \PYG{o}{+} \PYG{l+m+mi}{1}\PYG{p}{;}
        \PYG{n}{bd}\PYG{p}{.}\PYG{n}{disarm}\PYG{p}{;}
        \PYG{n}{disp} \PYG{l+s}{Triggered!}
    \PYG{n}{end}
    \PYG{l+s}{pause(0.01)}\PYG{p}{;}
\PYG{n}{end}

\PYG{l+s}{\PYGZpc{}\PYGZpc{}} \PYG{l+s}{End}
\end{sphinxVerbatim}


\bigskip\hrule\bigskip



\section{Phase Prediction Error Measurement}
\label{\detokenize{5_demo_scripts:phase-prediction-error-measurement}}
This demo script uses bossdevice research and 2 different approaches to generate jittered open loop stimulus

Resources: 1) bossdevice Switched On, 2) bossdevice Open Source MATLAB API, 3) The stimulator is Switched On, External Trigger mode is turned on and the Stimulator is Enabled

\begin{sphinxVerbatim}[commandchars=\\\{\}]
\PYG{n}{ang\PYGZus{}diff} \PYG{p}{=} \PYG{p}{@}\PYG{p}{(}\PYG{n}{x}\PYG{p}{,} \PYG{n}{y}\PYG{p}{)} \PYG{n+nb}{angle}\PYG{p}{(}\PYG{n+nb}{exp}\PYG{p}{(}1\PYG{n+nb}{i}\PYG{o}{*}\PYG{n}{x}\PYG{p}{)}\PYG{o}{./}\PYG{n+nb}{exp}\PYG{p}{(}1\PYG{n+nb}{i}\PYG{o}{*}\PYG{n}{y}\PYG{p}{)}\PYG{p}{)}\PYG{p}{;}
\PYG{n}{ang\PYGZus{}var} \PYG{p}{=} \PYG{p}{@}\PYG{p}{(}\PYG{n}{x}\PYG{p}{)} \PYG{l+m+mi}{1}\PYG{o}{\PYGZhy{}}\PYG{n+nb}{abs}\PYG{p}{(}\PYG{n}{mean}\PYG{p}{(}\PYG{n+nb}{exp}\PYG{p}{(}1\PYG{n+nb}{i}\PYG{o}{*}\PYG{n}{x}\PYG{p}{)}\PYG{p}{)}\PYG{p}{)}\PYG{p}{;}
\PYG{c}{\PYGZpc{}ang\PYGZus{}var2dev = @(v) sqrt(2*v); \PYGZpc{} circstat preferred formula uses angular deviation (bounded from 0 to sqrt(2)) which is sqrt(2*(1\PYGZhy{}r))}
\PYG{n}{ang\PYGZus{}var2dev} \PYG{p}{=} \PYG{p}{@}\PYG{p}{(}\PYG{n}{v}\PYG{p}{)} \PYG{n+nb}{sqrt}\PYG{p}{(}\PYG{o}{\PYGZhy{}}\PYG{l+m+mi}{2}\PYG{o}{*}\PYG{n+nb}{log}\PYG{p}{(}\PYG{l+m+mi}{1}\PYG{o}{\PYGZhy{}}\PYG{n}{v}\PYG{p}{)}\PYG{p}{)}\PYG{p}{;} \PYG{c}{\PYGZpc{} formula for circular standard deviation is sqrt(\PYGZhy{}2*ln(r))}

\PYG{c}{\PYGZpc{}\PYGZpc{}  Initializing bossdevice research API}
\PYG{n}{bb} \PYG{p}{=} \PYG{n}{bossdevice}\PYG{p}{;}
\PYG{n}{bd}\PYG{p}{.}\PYG{n}{eeg\PYGZus{}channels} \PYG{p}{=} \PYG{l+m+mi}{1}\PYG{p}{;}
\PYG{n}{bd}\PYG{p}{.}\PYG{n}{aux\PYGZus{}channels} \PYG{p}{=} \PYG{l+m+mi}{1}\PYG{p}{;}
\PYG{n}{bd}\PYG{p}{.}\PYG{n}{spatial\PYGZus{}filter\PYGZus{}weights} \PYG{p}{=} \PYG{l+m+mi}{1}\PYG{p}{;}

\PYG{n}{bd}\PYG{p}{.}\PYG{n}{alpha}\PYG{p}{.}\PYG{n}{offset\PYGZus{}samples} \PYG{p}{=} \PYG{l+m+mi}{5}\PYG{p}{;} \PYG{c}{\PYGZpc{}this depends on the loop\PYGZhy{}delay}

\PYG{c}{\PYGZpc{}\PYGZpc{} Setting Filters to bossdevice research}
\PYG{c}{\PYGZpc{} this allows calibrating the oscillation analysis to an individual peak frequency}
\PYG{n}{bd}\PYG{p}{.}\PYG{n}{alpha}\PYG{p}{.}\PYG{n}{bpf\PYGZus{}fir\PYGZus{}coeffs} \PYG{p}{=} \PYG{n}{firls}\PYG{p}{(}\PYG{l+m+mi}{70}\PYG{p}{,} \PYG{p}{[}\PYG{l+m+mi}{0} \PYG{l+m+mi}{6} \PYG{l+m+mi}{9} \PYG{l+m+mi}{13} \PYG{l+m+mi}{16} \PYG{p}{(}\PYG{l+m+mi}{500}\PYG{o}{/}\PYG{l+m+mi}{2}\PYG{p}{)}\PYG{p}{]}\PYG{o}{/}\PYG{p}{(}\PYG{l+m+mi}{500}\PYG{o}{/}\PYG{l+m+mi}{2}\PYG{p}{)}\PYG{p}{,} \PYG{p}{[}\PYG{l+m+mi}{0} \PYG{l+m+mi}{0} \PYG{l+m+mi}{1} \PYG{l+m+mi}{1} \PYG{l+m+mi}{0} \PYG{l+m+mi}{0}\PYG{p}{]}\PYG{p}{,} \PYG{p}{[}\PYG{l+m+mi}{1} \PYG{l+m+mi}{1} \PYG{l+m+mi}{1}\PYG{p}{]}\PYG{p}{)}\PYG{p}{;}
\PYG{c}{\PYGZpc{}fvtool(bd.alpha.bpf\PYGZus{}fir\PYGZus{}coeffs, \PYGZsq{}Fs\PYGZsq{}, 500) \PYGZpc{} visualize filter}

\PYG{c}{\PYGZpc{}\PYGZpc{} Configuring a scope to acquire data}
\PYG{n}{sc} \PYG{p}{=} \PYG{n}{addscope}\PYG{p}{(}\PYG{n}{bd}\PYG{p}{.}\PYG{n}{tg}\PYG{p}{,} \PYG{l+s}{\PYGZsq{}}\PYG{l+s}{host\PYGZsq{}}\PYG{p}{,} \PYG{l+m+mi}{101}\PYG{p}{)}\PYG{p}{;}
\PYG{n}{addsignal}\PYG{p}{(}\PYG{n}{sc}\PYG{p}{,} \PYG{n}{getsignalid}\PYG{p}{(}\PYG{n}{bd}\PYG{p}{.}\PYG{n}{tg}\PYG{p}{,} \PYG{l+s}{\PYGZsq{}}\PYG{l+s}{SPF/Matrix Multiply\PYGZsq{}}\PYG{p}{)}\PYG{p}{)}\PYG{p}{;} \PYG{c}{\PYGZpc{} this signals goes into the oscillation analysis}
\PYG{n}{addsignal}\PYG{p}{(}\PYG{n}{sc}\PYG{p}{,} \PYG{n}{getsignalid}\PYG{p}{(}\PYG{n}{bd}\PYG{p}{.}\PYG{n}{tg}\PYG{p}{,} \PYG{l+s}{\PYGZsq{}}\PYG{l+s}{OSC/alpha/IP\PYGZsq{}}\PYG{p}{)}\PYG{p}{)}\PYG{p}{;} \PYG{c}{\PYGZpc{} instantaneous phase estimate for alpha}

\PYG{n}{sc}\PYG{p}{.}\PYG{n}{NumSamples} \PYG{p}{=} \PYG{l+m+mi}{10} \PYG{o}{*} \PYG{l+m+mi}{5000}\PYG{p}{;}
\PYG{n}{sc}\PYG{p}{.}\PYG{n}{Decimation} \PYG{p}{=} \PYG{l+m+mi}{1}\PYG{p}{;}

\PYG{n}{fprintf}\PYG{p}{(}\PYG{l+s}{\PYGZsq{}}\PYG{l+s}{\PYGZbs{}nAcquiring data ...\PYGZsq{}}\PYG{p}{)}\PYG{p}{,} \PYG{n}{start}\PYG{p}{(}\PYG{n}{sc}\PYG{p}{)}\PYG{p}{;}
\PYG{k}{while}\PYG{p}{(}\PYG{n}{strcmp}\PYG{p}{(}\PYG{n}{sc}\PYG{p}{.}\PYG{n}{Status}\PYG{p}{,} \PYG{l+s}{\PYGZsq{}}\PYG{l+s}{Acquiring\PYGZsq{}}\PYG{p}{)}\PYG{p}{)}\PYG{p}{,} \PYG{n}{fprintf}\PYG{p}{(}\PYG{l+s}{\PYGZsq{}}\PYG{l+s}{.\PYGZsq{}}\PYG{p}{)}\PYG{p}{,} \PYG{n}{pause}\PYG{p}{(}\PYG{l+m+mi}{1}\PYG{p}{)}\PYG{p}{,} \PYG{k}{end}
\PYG{n}{fprintf}\PYG{p}{(}\PYG{l+s}{\PYGZsq{}}\PYG{l+s}{ done\PYGZsq{}}\PYG{p}{)}

\PYG{n}{data} \PYG{p}{=} \PYG{n}{sc}\PYG{p}{.}\PYG{n}{Data}\PYG{p}{(}\PYG{p}{:}\PYG{p}{,}\PYG{l+m+mi}{1}\PYG{p}{)}\PYG{p}{;}
\PYG{n}{ip\PYGZus{}estimate\PYGZus{}causal} \PYG{p}{=} \PYG{n}{sc}\PYG{p}{.}\PYG{n}{Data}\PYG{p}{(}\PYG{p}{:}\PYG{p}{,}\PYG{k}{end}\PYG{p}{)}\PYG{p}{;}
\PYG{n}{fs} \PYG{p}{=} \PYG{l+m+mi}{1}\PYG{o}{/}\PYG{n}{mean}\PYG{p}{(}\PYG{n}{diff}\PYG{p}{(}\PYG{n}{sc}\PYG{p}{.}\PYG{n}{Time}\PYG{p}{)}\PYG{p}{)}\PYG{p}{;}

\PYG{n}{fprintf}\PYG{p}{(}\PYG{l+s}{\PYGZsq{}}\PYG{l+s}{\PYGZbs{}nDetermining phase using standard non\PYGZhy{}causal methods ...\PYGZsq{}}\PYG{p}{)}
\PYG{c}{\PYGZpc{} demean}
\PYG{n}{data} \PYG{p}{=} \PYG{n}{data} \PYG{o}{\PYGZhy{}} \PYG{n}{mean}\PYG{p}{(}\PYG{n}{data}\PYG{p}{)}\PYG{p}{;}
\PYG{c}{\PYGZpc{} zero phase band\PYGZhy{}pass filter}
\PYG{n}{D} \PYG{p}{=} \PYG{n}{designfilt}\PYG{p}{(}\PYG{l+s}{\PYGZsq{}}\PYG{l+s}{bandpassfir\PYGZsq{}}\PYG{p}{,} \PYG{l+s}{\PYGZsq{}}\PYG{l+s}{FilterOrder\PYGZsq{}}\PYG{p}{,} \PYG{n+nb}{round}\PYG{p}{(}\PYG{l+m+mi}{1}\PYG{o}{*}\PYG{n}{fs}\PYG{p}{)}\PYG{p}{,} \PYG{l+s}{\PYGZsq{}}\PYG{l+s}{CutoffFrequency1\PYGZsq{}}\PYG{p}{,} \PYG{l+m+mi}{9}\PYG{p}{,} \PYG{l+s}{\PYGZsq{}}\PYG{l+s}{CutoffFrequency2\PYGZsq{}}\PYG{p}{,} \PYG{l+m+mi}{13}\PYG{p}{,} \PYG{l+s}{\PYGZsq{}}\PYG{l+s}{SampleRate\PYGZsq{}}\PYG{p}{,} \PYG{n}{fs}\PYG{p}{)}\PYG{p}{;}
\PYG{n}{data} \PYG{p}{=} \PYG{n}{filtfilt}\PYG{p}{(}\PYG{n}{D}\PYG{p}{,} \PYG{n}{data}\PYG{p}{)}\PYG{p}{;} \PYG{c}{\PYGZpc{}demean}

\PYG{n}{ip\PYGZus{}estimate\PYGZus{}noncausal} \PYG{p}{=} \PYG{n+nb}{angle}\PYG{p}{(}\PYG{n}{hilbert}\PYG{p}{(}\PYG{n}{data}\PYG{p}{)}\PYG{p}{)}\PYG{p}{;}
\PYG{n}{phase\PYGZus{}error} \PYG{p}{=} \PYG{n}{ang\PYGZus{}diff}\PYG{p}{(}\PYG{n}{ip\PYGZus{}estimate\PYGZus{}noncausal}\PYG{p}{,} \PYG{n}{ip\PYGZus{}estimate\PYGZus{}causal}\PYG{p}{)}\PYG{p}{;}

\PYG{n}{fprintf}\PYG{p}{(}\PYG{l+s}{\PYGZsq{}}\PYG{l+s}{\PYGZbs{}n\PYGZsq{}}\PYG{p}{)}

\PYG{c}{\PYGZpc{}\PYGZpc{} Visualize}
\PYG{n}{figure}
\PYG{l+s}{ax1} \PYG{l+s}{=} \PYG{l+s}{subplot(2,2,1)}\PYG{p}{;}
\PYG{n}{plot}\PYG{p}{(}\PYG{n}{sc}\PYG{p}{.}\PYG{n}{Time}\PYG{o}{\PYGZhy{}}\PYG{n}{sc}\PYG{p}{.}\PYG{n}{Time}\PYG{p}{(}\PYG{l+m+mi}{1}\PYG{p}{)}\PYG{p}{,} \PYG{p}{[}\PYG{n}{sc}\PYG{p}{.}\PYG{n}{Data}\PYG{p}{(}\PYG{p}{:}\PYG{p}{,}\PYG{l+m+mi}{1}\PYG{p}{)} \PYG{n}{data}\PYG{p}{]}\PYG{p}{)}
\PYG{n}{ax2} \PYG{p}{=} \PYG{n}{subplot}\PYG{p}{(}\PYG{l+m+mi}{2}\PYG{p}{,}\PYG{l+m+mi}{2}\PYG{p}{,}\PYG{l+m+mi}{2}\PYG{p}{)}\PYG{p}{;}
\PYG{n}{plot}\PYG{p}{(}\PYG{n}{sc}\PYG{p}{.}\PYG{n}{Time}\PYG{o}{\PYGZhy{}}\PYG{n}{sc}\PYG{p}{.}\PYG{n}{Time}\PYG{p}{(}\PYG{l+m+mi}{1}\PYG{p}{)}\PYG{p}{,} \PYG{p}{[}\PYG{n}{ip\PYGZus{}estimate\PYGZus{}causal} \PYG{n}{ip\PYGZus{}estimate\PYGZus{}noncausal}\PYG{p}{]}\PYG{p}{)}
\PYG{n}{linkaxes}\PYG{p}{(}\PYG{p}{[}\PYG{n}{ax1} \PYG{n}{ax2}\PYG{p}{]}\PYG{p}{,} \PYG{l+s}{\PYGZsq{}}\PYG{l+s}{x\PYGZsq{}}\PYG{p}{)}
\PYG{n}{subplot}\PYG{p}{(}\PYG{l+m+mi}{2}\PYG{p}{,}\PYG{l+m+mi}{2}\PYG{p}{,}\PYG{l+m+mi}{4}\PYG{p}{,}\PYG{n}{polaraxes}\PYG{p}{)}\PYG{p}{;}
\PYG{n}{polarhistogram}\PYG{p}{(}\PYG{n}{phase\PYGZus{}error}\PYG{p}{,} \PYG{l+s}{\PYGZsq{}}\PYG{l+s}{Normalization\PYGZsq{}}\PYG{p}{,} \PYG{l+s}{\PYGZsq{}}\PYG{l+s}{probability\PYGZsq{}}\PYG{p}{,} \PYG{l+s}{\PYGZsq{}}\PYG{l+s}{BinWidth\PYGZsq{}}\PYG{p}{,} \PYG{n+nb}{pi}\PYG{o}{/}\PYG{l+m+mi}{36}\PYG{p}{)}\PYG{p}{;}
\PYG{n}{title}\PYG{p}{(}\PYG{n}{sprintf}\PYG{p}{(}\PYG{l+s}{\PYGZsq{}}\PYG{l+s}{circular standard deviation = \PYGZpc{}.1f°\PYGZsq{}}\PYG{p}{,} \PYG{n}{rad2deg}\PYG{p}{(}\PYG{n}{ang\PYGZus{}var2dev}\PYG{p}{(}\PYG{n}{ang\PYGZus{}var}\PYG{p}{(}\PYG{n}{phase\PYGZus{}error}\PYG{p}{)}\PYG{p}{)}\PYG{p}{)}\PYG{p}{)}\PYG{p}{)}

\PYG{c}{\PYGZpc{} remove the scope}
\PYG{n}{remscope}\PYG{p}{(}\PYG{n}{bd}\PYG{p}{.}\PYG{n}{tg}\PYG{p}{,} \PYG{l+m+mi}{101}\PYG{p}{)}
\end{sphinxVerbatim}


\bigskip\hrule\bigskip



\section{Loop Latency Measurement}
\label{\detokenize{5_demo_scripts:loop-latency-measurement}}
This demo script uses bossdevice research and 2 different approaches to generate jittered open loop stimulus

Resources: 1) bossdevice Switched On, 2) bossdevice Open Source MATLAB API, 3) The stimulator is Switched On, External Trigger mode is turned on and the Stimulator is Enabled

\begin{sphinxVerbatim}[commandchars=\\\{\}]
\PYG{n}{bd} \PYG{p}{=} \PYG{n}{bossdevice}\PYG{p}{;}

\PYG{c}{\PYGZpc{}\PYGZpc{} Configuring Scope}
\PYG{n}{sc} \PYG{p}{=} \PYG{n}{addscope}\PYG{p}{(}\PYG{n}{bd}\PYG{p}{.}\PYG{n}{tg}\PYG{p}{,} \PYG{l+s}{\PYGZsq{}}\PYG{l+s}{host\PYGZsq{}}\PYG{p}{,} \PYG{l+m+mi}{255}\PYG{p}{)}\PYG{p}{;}

\PYG{n}{mrk\PYGZus{}signal\PYGZus{}id} \PYG{p}{=} \PYG{n}{getsignalid}\PYG{p}{(}\PYG{n}{bd}\PYG{p}{.}\PYG{n}{tg}\PYG{p}{,} \PYG{l+s}{\PYGZsq{}}\PYG{l+s}{UDP/raw\PYGZus{}mrk\PYGZsq{}}\PYG{p}{)} \PYG{o}{+} \PYG{n}{int32}\PYG{p}{(}\PYG{p}{[}\PYG{l+m+mi}{0} \PYG{l+m+mi}{1} \PYG{l+m+mi}{2}\PYG{p}{]}\PYG{p}{)}\PYG{p}{;}

\PYG{n}{addsignal}\PYG{p}{(}\PYG{n}{sc}\PYG{p}{,} \PYG{n}{mrk\PYGZus{}signal\PYGZus{}id}\PYG{p}{)}\PYG{p}{;}
\PYG{n}{sc}\PYG{p}{.}\PYG{n}{NumSamples} \PYG{p}{=} \PYG{l+m+mi}{100}\PYG{p}{;}
\PYG{n}{sc}\PYG{p}{.}\PYG{n}{NumPrePostSamples} \PYG{p}{=} \PYG{o}{\PYGZhy{}}\PYG{l+m+mi}{50}\PYG{p}{;}
\PYG{n}{sc}\PYG{p}{.}\PYG{n}{Decimation} \PYG{p}{=} \PYG{l+m+mi}{1}\PYG{p}{;}
\PYG{n}{sc}\PYG{p}{.}\PYG{n}{TriggerMode} \PYG{p}{=} \PYG{l+s}{\PYGZsq{}}\PYG{l+s}{Signal\PYGZsq{}}\PYG{p}{;}
\PYG{n}{sc}\PYG{p}{.}\PYG{n}{TriggerSignal} \PYG{p}{=} \PYG{n}{getsignalid}\PYG{p}{(}\PYG{n}{bd}\PYG{p}{.}\PYG{n}{tg}\PYG{p}{,} \PYG{l+s}{\PYGZsq{}}\PYG{l+s}{gen\PYGZus{}running\PYGZsq{}}\PYG{p}{)}\PYG{p}{;}
\PYG{n}{sc}\PYG{p}{.}\PYG{n}{TriggerLevel} \PYG{p}{=} \PYG{l+m+mf}{0.5}\PYG{p}{;}
\PYG{n}{sc}\PYG{p}{.}\PYG{n}{TriggerSlope} \PYG{p}{=} \PYG{l+s}{\PYGZsq{}}\PYG{l+s}{Rising\PYGZsq{}}\PYG{p}{;}

\PYG{c}{\PYGZpc{}\PYGZpc{} Generating Trigger}

\PYG{n}{fprintf}\PYG{p}{(}\PYG{l+s}{\PYGZsq{}}\PYG{l+s}{\PYGZbs{}nTesting... \PYGZsq{}}\PYG{p}{)}
\PYG{n}{start}\PYG{p}{(}\PYG{n}{sc}\PYG{p}{)}\PYG{p}{;}
\PYG{n}{pause}\PYG{p}{(}\PYG{l+m+mf}{0.1}\PYG{p}{)}\PYG{p}{;} \PYG{c}{\PYGZpc{} give the scope time to pre\PYGZhy{}aquire}
\PYG{n}{assert}\PYG{p}{(}\PYG{n}{strcmp}\PYG{p}{(}\PYG{n}{sc}\PYG{p}{.}\PYG{n}{Status}\PYG{p}{,} \PYG{l+s}{\PYGZsq{}}\PYG{l+s}{Ready for being Triggered\PYGZsq{}}\PYG{p}{)}\PYG{p}{)}\PYG{p}{;}

\PYG{n}{s} \PYG{p}{=} \PYG{p}{[}\PYG{l+m+mi}{0}\PYG{p}{,} \PYG{l+m+mi}{1}\PYG{p}{,} \PYG{l+m+mi}{0}\PYG{p}{]}\PYG{p}{;}
\PYG{n}{s}\PYG{p}{(}\PYG{l+m+mi}{1000}\PYG{p}{,}\PYG{l+m+mi}{3}\PYG{p}{)} \PYG{p}{=} \PYG{l+m+mi}{0}\PYG{p}{;} \PYG{c}{\PYGZpc{} fill with zeros (TODO: this should be done in the API)}
\PYG{n}{bd}\PYG{p}{.}\PYG{n}{generator\PYGZus{}sequence} \PYG{p}{=} \PYG{n}{s}\PYG{p}{;}

\PYG{n}{bd}\PYG{p}{.}\PYG{n}{manualTrigger}\PYG{p}{;}

\PYG{n}{pause}\PYG{p}{(}\PYG{l+m+mf}{0.1}\PYG{p}{)}
\PYG{n}{assert}\PYG{p}{(}\PYG{n}{strcmp}\PYG{p}{(}\PYG{n}{sc}\PYG{p}{.}\PYG{n}{Status}\PYG{p}{,} \PYG{l+s}{\PYGZsq{}}\PYG{l+s}{Finished\PYGZsq{}}\PYG{p}{)}\PYG{p}{)}

\PYG{n}{fprintf}\PYG{p}{(}\PYG{l+s}{\PYGZsq{}}\PYG{l+s}{loop delay is \PYGZpc{}2.1f ms\PYGZbs{}n\PYGZsq{}}\PYG{p}{,} \PYG{p}{(}\PYG{n+nb}{find}\PYG{p}{(}\PYG{n}{sc}\PYG{p}{.}\PYG{n}{Data}\PYG{p}{(}\PYG{p}{:}\PYG{p}{,}\PYG{l+m+mi}{1}\PYG{p}{)}\PYG{p}{,} \PYG{l+m+mi}{1}\PYG{p}{)}\PYG{o}{\PYGZhy{}}\PYG{l+m+mi}{50}\PYG{p}{)}\PYG{o}{/}\PYG{l+m+mi}{5}\PYG{p}{)}
\end{sphinxVerbatim}


\bigskip\hrule\bigskip



\chapter{Downloads \& Dependencies}
\label{\detokenize{6_downloads_n_dependencies:downloads-dependencies}}\label{\detokenize{6_downloads_n_dependencies::doc}}

\section{Download \& Setup API}
\label{\detokenize{6_downloads_n_dependencies:download-setup-api}}
bossdevice research API is an open source MATLAB repository incubated at sync2brain’s GitHub account and is linked below:

\begin{sphinxVerbatim}[commandchars=\\\{\}]
\PYG{n}{https}\PYG{p}{:}\PYG{o}{/}\PYG{o}{/}\PYG{n}{github}\PYG{p}{.}\PYG{n}{com}\PYG{o}{/}\PYG{n}{sync2brain}\PYG{o}{/}\PYG{n}{bossdevice}\PYG{o}{\PYGZhy{}}\PYG{n}{api}\PYG{o}{\PYGZhy{}}\PYG{n}{matlab}
\end{sphinxVerbatim}


\bigskip\hrule\bigskip



\section{Download bossdevice firmware}
\label{\detokenize{6_downloads_n_dependencies:download-bossdevice-firmware}}
bossdevice research simulink frmware is also incubated at sync2brain’s GitHub account and is linked below:

\begin{sphinxVerbatim}[commandchars=\\\{\}]
\PYG{n}{https}\PYG{p}{:}\PYG{o}{/}\PYG{o}{/}\PYG{n}{github}\PYG{p}{.}\PYG{n}{com}\PYG{o}{/}\PYG{n}{sync2brain}\PYG{o}{/}\PYG{n}{bossdevice}\PYG{o}{\PYGZhy{}}\PYG{n}{api}\PYG{o}{\PYGZhy{}}\PYG{n}{matlab}
\end{sphinxVerbatim}


\bigskip\hrule\bigskip



\section{Requirements:}
\label{\detokenize{6_downloads_n_dependencies:requirements}}
\begin{sphinxadmonition}{important}{Important:}
Matlab 2021a downloaded and installed
\end{sphinxadmonition}

\begin{sphinxadmonition}{important}{Important:}
Simulink Real\sphinxhyphen{}Time and Simulink Coder to be installed as an Adds\sphinxhyphen{}On from MATLAB Toolboxes
\end{sphinxadmonition}

\begin{sphinxadmonition}{important}{Important:}
Download the repository from master branch
\end{sphinxadmonition}

\begin{sphinxadmonition}{important}{Important:}
Add the bossdevice firmware and repository to the search path of MATLAB
\end{sphinxadmonition}

\begin{sphinxadmonition}{important}{Important:}
Observe from command line that if bossdevice firmware has been loaded from intended path, otherwise many copies of API are available on your device and MATLAB is confused about choosing one therefore it is recommended to change the current directory of MATLAB to the correct path of bossdevice latest API.
\end{sphinxadmonition}


\chapter{Issues, Bugs and Requests}
\label{\detokenize{7_issues_bugs_requests:issues-bugs-and-requests}}\label{\detokenize{7_issues_bugs_requests::doc}}

\section{Issues | Bugs | Requests}
\label{\detokenize{7_issues_bugs_requests:issues-bugs-requests}}
You first have to sign up for an account or log in on GitHub, subsequently you can report the issue on the GitHub repositroty linked below:

\begin{sphinxVerbatim}[commandchars=\\\{\}]
\PYG{n}{ttps}\PYG{p}{:}\PYG{o}{/}\PYG{o}{/}\PYG{n}{github}\PYG{p}{.}\PYG{n}{com}\PYG{o}{/}\PYG{n}{sync2brain}\PYG{o}{/}\PYG{n}{bossdevice}\PYG{o}{\PYGZhy{}}\PYG{n}{api}\PYG{o}{\PYGZhy{}}\PYG{n}{matlab}\PYG{o}{/}\PYG{n}{issues}
\end{sphinxVerbatim}

We automatically will receive an email and will follow up and keep you updated; i.e., you will get an email through GitHub whenever someone works on your issue.

\begin{sphinxadmonition}{important}{Important:}
Alternatively, you can also write an email to sync2brain support team for a quick response to your questions.
\end{sphinxadmonition}


\bigskip\hrule\bigskip



\section{Issues \& Bugs Details}
\label{\detokenize{7_issues_bugs_requests:issues-bugs-details}}
The easier it is for one of the developers to reproduce your bug, the more likely it is that we’ll fix the problem. Good bug reports include a small test script and the data (i.e. mat file) required to reproduce the bug.

Please create a small test script and a piece of data that are both as small and simple as possible to reproduce the problem. For example: a .mat file containing the bossdevice API class object and some screenshots can lead us to solve the issue.


\bigskip\hrule\bigskip



\section{Use Case Scenarios}
\label{\detokenize{7_issues_bugs_requests:use-case-scenarios}}
\begin{sphinxadmonition}{important}{Important:}
Use cases, previous research papers that might have implemented those methods or a brief description about the feature requests can help our developers to create the smartest solution for you.
\end{sphinxadmonition}



\renewcommand{\indexname}{Index}
\printindex
\end{document}