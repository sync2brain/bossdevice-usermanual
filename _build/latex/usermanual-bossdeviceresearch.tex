%% Generated by Sphinx.
\def\sphinxdocclass{report}
\documentclass[letterpaper,10pt,english]{sphinxmanual}
\ifdefined\pdfpxdimen
   \let\sphinxpxdimen\pdfpxdimen\else\newdimen\sphinxpxdimen
\fi \sphinxpxdimen=.75bp\relax
\ifdefined\pdfimageresolution
    \pdfimageresolution= \numexpr \dimexpr1in\relax/\sphinxpxdimen\relax
\fi
%% let collapsable pdf bookmarks panel have high depth per default
\PassOptionsToPackage{bookmarksdepth=5}{hyperref}

\PassOptionsToPackage{warn}{textcomp}
\usepackage[utf8]{inputenc}
\ifdefined\DeclareUnicodeCharacter
% support both utf8 and utf8x syntaxes
  \ifdefined\DeclareUnicodeCharacterAsOptional
    \def\sphinxDUC#1{\DeclareUnicodeCharacter{"#1}}
  \else
    \let\sphinxDUC\DeclareUnicodeCharacter
  \fi
  \sphinxDUC{00A0}{\nobreakspace}
  \sphinxDUC{2500}{\sphinxunichar{2500}}
  \sphinxDUC{2502}{\sphinxunichar{2502}}
  \sphinxDUC{2514}{\sphinxunichar{2514}}
  \sphinxDUC{251C}{\sphinxunichar{251C}}
  \sphinxDUC{2572}{\textbackslash}
\fi
\usepackage{cmap}
\usepackage[T1]{fontenc}
\usepackage{amsmath,amssymb,amstext}
\usepackage{babel}



\usepackage{tgtermes}
\usepackage{tgheros}
\renewcommand{\ttdefault}{txtt}



\usepackage[Bjarne]{fncychap}
\usepackage{sphinx}

\fvset{fontsize=auto}
\usepackage{geometry}


% Include hyperref last.
\usepackage{hyperref}
% Fix anchor placement for figures with captions.
\usepackage{hypcap}% it must be loaded after hyperref.
% Set up styles of URL: it should be placed after hyperref.
\urlstyle{same}

\addto\captionsenglish{\renewcommand{\contentsname}{Table of Contents}}

\usepackage{sphinxmessages}




\title{User Manual \sphinxhyphen{} bossdevice RESEARCH}
\date{Oct 20, 2021}
\release{20\sphinxhyphen{}10\sphinxhyphen{}2021}
\author{sync2brain GmbH}
\newcommand{\sphinxlogo}{\vbox{}}
\renewcommand{\releasename}{Release}
\makeindex
\begin{document}

\pagestyle{empty}
\sphinxmaketitle
\pagestyle{plain}
\sphinxtableofcontents
\pagestyle{normal}
\phantomsection\label{\detokenize{index::doc}}



\chapter{Introduction to bossdevice RESEARCH}
\label{\detokenize{1_introduction_to_bossdevice_research:introduction-to-bossdevice-research}}\label{\detokenize{1_introduction_to_bossdevice_research::doc}}
\sphinxAtStartPar
bossdevice RESEARCH is a real\sphinxhyphen{}time digital signal processor consisting of hardware and software algorithms. It is designed to read\sphinxhyphen{}in a real\sphinxhyphen{}time raw data stream from a biosignal amplifier (electroencephalography, EEG), to continuously analyze this data and to detect patterns based on oscillations in different frequencies. When such a specific bio\sphinxhyphen{}signal pattern is detected, the device indicates this through a standard output port. This enables a connected device to know with millisecond accuracy when a specific bio\sphinxhyphen{}signal pattern occurs.

\begin{sphinxadmonition}{important}{Important:}
\sphinxAtStartPar
bossdevice RESEARCH is not a medical device. It may not be used outside of research
and it may not be used in trials involving patients. It is not intended as an accessory
to a medical device or to control a medical device. It may only be connected to a
medical device through a connection that is not considered a “control” connection
but only an information transmission. What the connected device does in response
to information received from the bossdevice is determined by the connected device
only.
\end{sphinxadmonition}

\begin{figure}[htbp]
\centering
\capstart

\noindent\sphinxincludegraphics{{Fig1_bossdeviceandneurone}.png}
\caption{The bossdevice RESEARCH placed along with Bittium NeurOne biosignal amplifier.}\label{\detokenize{1_introduction_to_bossdevice_research:id1}}\end{figure}


\chapter{Getting Started \sphinxhyphen{} Checklist}
\label{\detokenize{1b_GettingStartedCheckList:getting-started-checklist}}\label{\detokenize{1b_GettingStartedCheckList::doc}}
\begin{sphinxadmonition}{important}{Important:}
\sphinxAtStartPar
This chapter is summarizing all the steps in order to get you started quickly. The detail information on each topic is available in respective chapters of the user manual.
\end{sphinxadmonition}


\section{Bittium NeurOne Users}
\label{\detokenize{1b_GettingStartedCheckList:bittium-neurone-users}}
\sphinxAtStartPar
The checlist below helps in setting up the bossdevice RESEARCH.
\begin{enumerate}
\sphinxsetlistlabels{\arabic}{enumi}{enumii}{}{.}%
\item {} 
\sphinxAtStartPar
The bossdevice have been received by you. The drivers have been installed from the sync2brain.com/downloads.

\item {} 
\sphinxAtStartPar
NeurOne biosignal amplifier with firmware 1.4.30 or above.

\item {} 
\sphinxAtStartPar
A computer running any windows 10 operating system.

\item {} 
\sphinxAtStartPar
NeurOne software installed on the windows computer version 1.5.1 or above.

\item {} 
\sphinxAtStartPar
MATLAB release 2021a installaled on the windows 10 computer.

\item {} 
\sphinxAtStartPar
Install Simulink, Simulink Real\sphinxhyphen{}Time, Simulink Coder and MATLAB Coder, Simulink Real\sphinxhyphen{}Time Target support package under MathWorks 2021a products. Hardware drivers also need to be installed according to instructions given in the chapter ‘Setup bossdevice Research chapter’.

\item {} 
\sphinxAtStartPar
1x LAN cable connecting NeurOne biosignal amplifier and the bossdevice as explained in section 3.1.

\item {} 
\sphinxAtStartPar
1x LAN cable connecting bossdevice and windows 10 computer as described in section 3.2 of bossdevice usermanual.

\item {} 
\sphinxAtStartPar
1x LAN cable connecting NeurOne biosignal amplifier and windows 10 computer, more details on this topic can be found in NeurOne usermanual.

\item {} 
\sphinxAtStartPar
Setup NeurOne LAN connection on the windows 10 computer, more details on this topic can be found in NeurOne usermanual.

\item {} 
\sphinxAtStartPar
Setup bossdevice LAN connection on the windows 10 computer as described in section 3.2 of bossdevice usermanual.

\item {} 
\sphinxAtStartPar
Downlaod the bossdevice API and latest firmware (DBSP.mldatx file)from the sync2brain support website available at api.sync2brain.com

\end{enumerate}

\begin{sphinxadmonition}{important}{Important:}
\sphinxAtStartPar
It is possible to run the NeurOne EEG software and the Matlab control script both on the same computer. In this case, we recommend a computer with 3 network connections: (1) Wired ethernet connection to the NeurOne Main Unit; (2) Wired ethernet connection to the bossdevice; (3) Wired or wireless internet/local area network connection.
\end{sphinxadmonition}


\section{Brainproducts actiCHamp/Turbolink Users}
\label{\detokenize{1b_GettingStartedCheckList:brainproducts-actichamp-turbolink-users}}
\sphinxAtStartPar
The checlist below helps in setting up the bossdevice RESEARCH.
\begin{enumerate}
\sphinxsetlistlabels{\arabic}{enumi}{enumii}{}{.}%
\item {} 
\sphinxAtStartPar
The bossdevice have been received by you. The drivers have been installed from the sync2brain.com/downloads.

\item {} 
\sphinxAtStartPar
Turbolink setup has been completed on a computer.

\item {} 
\sphinxAtStartPar
MATLAB release 2021a installaled on the windows 10 computer.

\item {} 
\sphinxAtStartPar
Install Simulink, Simulink Real\sphinxhyphen{}Time, Simulink Coder and MATLAB Coder, Simulink Real\sphinxhyphen{}Time Target support package under MathWorks 2021a products. Hardware drivers also need to be installed according to instructions given in the chapter ‘Setup bossdevice Research chapter’.

\item {} 
\sphinxAtStartPar
1x LAN cable connecting actiCHamp biosignal amplifier (via Turbolink) and the bossdevice.

\item {} 
\sphinxAtStartPar
1x LAN cable connecting bossdevice and windows 10 computer.

\item {} 
\sphinxAtStartPar
Setup bossdevice LAN connection on the windows 10 computer.

\item {} 
\sphinxAtStartPar
Downlaod the bossdevice API and latest firmware (DBSP.mldatx file)from the sync2brain support website available at sync2brain.com/downloads

\end{enumerate}


\chapter{Setup bossdevice RESEARCH}
\label{\detokenize{2_setup_bossdevice_research:setup-bossdevice-research}}\label{\detokenize{2_setup_bossdevice_research::doc}}
\sphinxAtStartPar
The bossdevice plugs into a compatible EEG amplifier on the input side and generates a fuzzy logic based binary signal on the output side. It reads a real\sphinxhyphen{}time biosignal data stream and generates a  signal when a pre\sphinxhyphen{}defined EEG\sphinxhyphen{}state is detected. The bossdevice is programmable from Matlab and enables complex experiments to be fully automated. Setting up the device involves several steps and the next sections descibre their details.


\section{Pre\sphinxhyphen{}requisites}
\label{\detokenize{2_setup_bossdevice_research:pre-requisites}}\begin{itemize}
\item {} 
\sphinxAtStartPar
Windows 10 operating system

\item {} 
\sphinxAtStartPar
MatLab R2021a

\item {} 
\sphinxAtStartPar
Simulink for MatLab R2021a update release 5

\item {} 
\sphinxAtStartPar
Simulink Coder add\sphinxhyphen{}on

\item {} 
\sphinxAtStartPar
Simulink Real\sphinxhyphen{}time add on

\item {} 
\sphinxAtStartPar
Simulink Real\sphinxhyphen{}Time target support package (add on)

\item {} 
\sphinxAtStartPar
The bossdevice RESEARCH API (download from here). The API contain the Matlab classes needed to control the device as well as a few MatLab demo scripts as example of codes that could be implemented to work with the device.

\item {} 
\sphinxAtStartPar
The bossdevice RESEARCH firmware, accessible at sync2brain.com/downloads.

\item {} 
\sphinxAtStartPar
Hardware drivers named as Speedgoat I/O Blockset R2021a (v9.2.2.1) drivers, accessible at sync2brain.com/downloads.

\item {} 
\sphinxAtStartPar
Optional: a screen (recommended in case of troubleshooting)

\end{itemize}


\section{Connecting bossdevice with Control PC}
\label{\detokenize{2_setup_bossdevice_research:connecting-bossdevice-with-control-pc}}
\sphinxAtStartPar
In order to program the bossdevice RESEARCH, it needs to be connected to any standard PC hosting a Matlab and other dependencies as described in the “Downloads \& Dependencies” chapter and Pre\sphinxhyphen{}requistes section above. The “Control PC” LAN port at the backplate of the bossdevice RESEARCH is connected to any standard LAN/Ethernet port on the backplate of Control PC. Then in order to open the network IP address, follow these steps:
\begin{enumerate}
\sphinxsetlistlabels{\arabic}{enumi}{enumii}{}{.}%
\item {} 
\sphinxAtStartPar
On Microsoft Windows 7/8/10 Click Start Menu \textgreater{} Control Panel \textgreater{} Network and Sharing Centre or Network and Internet \textgreater{} Network and Sharing Centre.

\item {} 
\sphinxAtStartPar
Click Change adapter settings.

\item {} 
\sphinxAtStartPar
Right\sphinxhyphen{}click on Local Area Connection corresponding to LAN/Ethernet port where the “Control PC” LAN cable is connected and click Properties.

\end{enumerate}

\begin{figure}[htbp]
\centering
\capstart

\noindent\sphinxincludegraphics{{Fig3_controlpc1}.png}
\caption{Step 3: Local Area Connection setup of Control PC and bossdevice RESEARCH}\label{\detokenize{2_setup_bossdevice_research:id1}}\end{figure}
\begin{enumerate}
\sphinxsetlistlabels{\arabic}{enumi}{enumii}{}{.}%
\setcounter{enumi}{3}
\item {} 
\sphinxAtStartPar
Select Internet Protocol Version 4 (TCP/IPv4) and click Properties.

\end{enumerate}

\begin{figure}[htbp]
\centering
\capstart

\noindent\sphinxincludegraphics{{Figure4_controlpc2}.png}
\caption{Step 4: Local Area Connection setup of Control PC and bossdevice RESEARCH}\label{\detokenize{2_setup_bossdevice_research:id2}}\end{figure}

\sphinxAtStartPar
5. Select Use the following IP address. Then enter the following IP address, Subnet mask, Default gateway, and DNS server.
.. code\sphinxhyphen{}block:

\begin{sphinxVerbatim}[commandchars=\\\{\}]
\PYG{n}{IP} \PYG{n}{Address}\PYG{p}{:}          \PYG{l+m+mf}{192.168}\PYG{o}{.}\PYG{l+m+mf}{7.10}
\PYG{n}{Subnet} \PYG{n}{mask}\PYG{p}{:}         \PYG{l+m+mf}{255.255}\PYG{o}{.}\PYG{l+m+mf}{255.0}
\PYG{n}{Default} \PYG{n}{gateway}\PYG{p}{:}
\PYG{n}{DNS} \PYG{n}{server}\PYG{p}{:}
\end{sphinxVerbatim}
\begin{enumerate}
\sphinxsetlistlabels{\arabic}{enumi}{enumii}{}{.}%
\setcounter{enumi}{5}
\item {} 
\sphinxAtStartPar
Click OK and OK on your way out of the wizard.

\end{enumerate}


\section{Simulink Real\sphinxhyphen{}Time Target Support Package}
\label{\detokenize{2_setup_bossdevice_research:simulink-real-time-target-support-package}}
\sphinxAtStartPar
When starting Matlab 2021 release 5 for the first time, additional installations are needed.
In the Home tab, click on Add\sphinxhyphen{}Ons. Go to Get Add\sphinxhyphen{}ons and look for Simulink Real\sphinxhyphen{}Time Support Package. Install it, it will ask you to log in with your MatLab credentials.


\section{Download \& Install Hardware Drivers}
\label{\detokenize{2_setup_bossdevice_research:download-install-hardware-drivers}}
\sphinxAtStartPar
Specific drivers for the hardware of bossdevice RESEARCH  must be downloaded installed at the first time using bossdevice RESEARCH with the new installation of Matlab, specifically the Speedgoat I/O Blockset for R2021a (v9.2.2.1). In order to download the drivers, visit sync2brain.com/downloads and download the Speedgoat I/O Blockset. When downloaded, unzip the file and save it in a temporary folder that must be deleted from the computer after the installation have been completed.

\sphinxAtStartPar
The hardware drivers must be installed manually. Go into the unzipped drivers’ folder from Matlab path search, right click on the executable in there and run the installation (a documentation page will pop up, just close it as not necessary). When asked by Matlab, accept the waiver and type Y (i.e., yes) twice; the second Y will close the software as Matlab needs to be restarted after this installation.
At the end of these two installations, reopen Matlabv and type Speedgoat to check that everything ran smoothly. If successful, a hardware dedicated window of speedgoat should pop up with different options available (e.g., troubleshoot, support, etc.).


\section{Windows Firewall Check}
\label{\detokenize{2_setup_bossdevice_research:windows-firewall-check}}
\sphinxAtStartPar
First, make sure all the network communication for Matlab R2021a are all allowed in the Windows Defender Firewall. To do this, search for Windows Defender Firewall with Advanced Security and look for Matlab R2021a Inbound and Outbound rules: if they are blocked, right click and Allow them. If they are already allowed, exit and continue to the next step.


\section{Software Update on bossdevice}
\label{\detokenize{2_setup_bossdevice_research:software-update-on-bossdevice}}
\sphinxAtStartPar
This step is also needed only once in the beginning of setup. In Matlab, type slrtExplorer: a new window should pop up. If you want, change the name of the device (meaningful could be bossdevice), and click on Software Update (not a necessity but still good step just in case). When the update is finished, click on the Disconnected button at the top left corner of the window. A connection (shown by a green symbol) should be established. If this is not the case, the sync2brain support team should be contacted for support.

\sphinxAtStartPar
Please note: if the bossdevice RESEARCH main screen shows a Matlab version different from Matlab R2021a, a major update is necessary. In this case, the sync2brain team should be contacted for support. After the update is made, make sure to switch off and on the device for complete update.


\section{Configure biosignal Ethernet port of bossdevice}
\label{\detokenize{2_setup_bossdevice_research:configure-biosignal-ethernet-port-of-bossdevice}}
\sphinxAtStartPar
The connection between the bossdevice RESEARCH and biosignal amplifier (NeurOne/TurboLink) can be done now. In Matlab, type speedgoat.configureEthernet and a new window should pop up. If the window have not popped up, carefully repeat all the above mentioned steps. Select your device from the dropdown menu: its features should appear in the window below. On the right\sphinxhyphen{}hand side, type the following:

\begin{sphinxVerbatim}[commandchars=\\\{\}]
\PYG{n}{IP} \PYG{n}{Address}\PYG{p}{:}          \PYG{l+m+mf}{192.168}\PYG{o}{.}\PYG{l+m+mf}{200.255}
\PYG{n}{Subnet} \PYG{n}{mask}\PYG{p}{:}         \PYG{l+m+mf}{255.255}\PYG{o}{.}\PYG{l+m+mf}{255.0}
\PYG{n}{Default} \PYG{n}{gateway}\PYG{p}{:}     \PYG{l+m+mf}{192.168}\PYG{o}{.}\PYG{l+m+mf}{7.5}
\end{sphinxVerbatim}

\sphinxAtStartPar
You can then click on Load configuration. If the green light appears, the communication has been successfully established.


\section{Setup bossdevice API}
\label{\detokenize{2_setup_bossdevice_research:setup-bossdevice-api}}
\sphinxAtStartPar
Your bossdevice RESEARCH is now ready to work! In MatLab, go into the bossdevice API folder or simply add the path to the code. Type b = bossdevice; in Matlab and check that the firmware is being loaded (i.e., that no error appears and the status on the screen attached to the bossdevice is updated). You can now run the demo script or your experimental paradigm.

\sphinxAtStartPar
Please note: if you encounter an out\sphinxhyphen{}of\sphinxhyphen{}date error when trying to load the firmware it could be a problem of Matlab not being able to read the firmware properly. In this case, contact the sync2brain team for support.


\section{Connecting with Biosignal Amplifier (EEG Device)}
\label{\detokenize{2_setup_bossdevice_research:connecting-with-biosignal-amplifier-eeg-device}}
\sphinxAtStartPar
The input of the bossdevice i.e. the EEG data is provided by connecting the Biosignal Amplifier (EEG device e.g. Bittium’s NeurOne or Brain Product’s actiCHamp) with the bossdevice using a standard LAN Cable. Search the “REAL TIME OUT” LAN port at the back plate of your EEG device and connect it to the “Biosignal” LAN port available at the bossdevice research. More details on this topic are available in the chapter 4 ‘Setup Biosignal Amplifier’. You can also see the “Biosignal” port on the bossdevice research in the figure shown below.

\begin{figure}[htbp]
\centering
\capstart

\noindent\sphinxincludegraphics{{Figure2_biosignallanport}.png}
\caption{Ports available on the bossdevice (see the Biosignal LAN port).}\label{\detokenize{2_setup_bossdevice_research:id3}}\end{figure}


\section{Connecting with Synchronizing Devices}
\label{\detokenize{2_setup_bossdevice_research:connecting-with-synchronizing-devices}}
\sphinxAtStartPar
The bossdevice RESEARCH can send up to 4 TTL output signals to connected synchronizing devices from its output port. Search the TTL OUT 1\sphinxhyphen{}4” BNC ports at back plate of the bossdevice RESEARCH and connect it to the BNC ports of your intended device.

\begin{figure}[htbp]
\centering
\capstart

\noindent\sphinxincludegraphics{{Figure6_ttlports}.png}
\caption{Ports available on the bossdevice (see the BNC output ports 1\sphinxhyphen{}4 named as TTL OUT 1 till 4).}\label{\detokenize{2_setup_bossdevice_research:id4}}\end{figure}

\sphinxAtStartPar
In addition to the TTL output the 8 bit LPT marker ports can also be connected to the bossdevice RESEARCH for a numerical input or can also be output from bossdevice RESEARCH to write synchronizing markers to the raw EEG data or for any other purpose.

\begin{figure}[htbp]
\centering
\capstart

\noindent\sphinxincludegraphics{{Figure7_markersport}.png}
\caption{Ports available on the bossdevice (see the Marker 1 and Marker 2 8 bit LPT ports).}\label{\detokenize{2_setup_bossdevice_research:id5}}\end{figure}


\section{Connecting with Monitor}
\label{\detokenize{2_setup_bossdevice_research:connecting-with-monitor}}
\sphinxAtStartPar
Monitor setup is optional and not recommended for normal use cases. It shows the useful information and analytics of the bossdevice RESEARCH time of execution and errors (if any) at the ocassion of debugging and setting up the device. The Display Port from backplate of the bossdevice RESEARCH may be connected to the monitor.

\begin{figure}[htbp]
\centering
\capstart

\noindent\sphinxincludegraphics{{Figure2_biosignallanport}.png}
\caption{Ports available on the bossdevice (see the Display port (DP) for connection to monitor).}\label{\detokenize{2_setup_bossdevice_research:id6}}\end{figure}


\section{Power Supply}
\label{\detokenize{2_setup_bossdevice_research:power-supply}}
\sphinxAtStartPar
The power supply provided along with the bossdevice research can be plugged into the “DC IN 12V” port at the back plate of the bossdevice RESEARCH as shown in figure below.

\begin{figure}[htbp]
\centering
\capstart

\noindent\sphinxincludegraphics{{Figure2_biosignallanport}.png}
\caption{Ports available on the bossdevice RESEARCH (see the 12V DC power supply connection port).}\label{\detokenize{2_setup_bossdevice_research:id7}}\end{figure}


\section{Switching ON and OFF}
\label{\detokenize{2_setup_bossdevice_research:switching-on-and-off}}
\sphinxAtStartPar
The bossdevice can be switched on or off using the push button at on of the sides of bossdevice. When switched off, the LED present on the button is not illuminated while when it is switched on, a blue LED is illuminated.


\chapter{Setup biosignal Amplifier}
\label{\detokenize{3_setup_biosignal_amplifier:setup-biosignal-amplifier}}\label{\detokenize{3_setup_biosignal_amplifier::doc}}

\section{Bittium NeurOne}
\label{\detokenize{3_setup_biosignal_amplifier:bittium-neurone}}
\begin{sphinxadmonition}{important}{Important:}
\sphinxAtStartPar
bossdevice RESEARCH accepts a 1kHz sampling rate of the biosignal, therefore the sampling rate from the real time UDP out of NeurOne must be 1kHz.
\end{sphinxadmonition}

\begin{sphinxadmonition}{important}{Important:}
\sphinxAtStartPar
In your NeurOne protocol, keep all the EEG channels on the top and other Aux/bipolar/EMG/ECG channels at the bottom.
\end{sphinxadmonition}

\begin{sphinxadmonition}{important}{Important:}
\sphinxAtStartPar
If two 32 channels preamplifiers are used, do not use the Aux/bipolar inputs/EMG/ECG channels numberd from 33\sphinxhyphen{}40, but only the last 8 of the 2nd amplifier i.e. 73 to 80. Similar pattern applies to other combinition of preamplifiers.
\end{sphinxadmonition}

\begin{sphinxadmonition}{important}{Important:}
\sphinxAtStartPar
bossdevice RESEARCH can take maximum of 128 EEG channels only, and 8 Aux/Bipoloar/EMG/ECG channels.
\end{sphinxadmonition}


\subsection{Software Version}
\label{\detokenize{3_setup_biosignal_amplifier:software-version}}
\sphinxAtStartPar
The NeurOne software version should be v1.5.1 or higher.

\begin{figure}[htbp]
\centering

\noindent\sphinxincludegraphics{{Figure8_N1}.png}
\end{figure}


\subsection{Firmware Version}
\label{\detokenize{3_setup_biosignal_amplifier:firmware-version}}
\sphinxAtStartPar
The firmware version should be 1.4.30 or higher.


\subsection{Real Time Out Channels}
\label{\detokenize{3_setup_biosignal_amplifier:real-time-out-channels}}
\sphinxAtStartPar
The Target UDP Port should be changed to 25000 from default of 50000, and the rest of details as shown in the figure below.

\begin{figure}[htbp]
\centering

\noindent\sphinxincludegraphics{{Figure9_N1}.png}
\end{figure}


\section{Brain Products actiCHamp (Turbolink)}
\label{\detokenize{3_setup_biosignal_amplifier:brain-products-actichamp-turbolink}}
\begin{sphinxadmonition}{important}{Important:}
\sphinxAtStartPar
bossdevice RESEARCH accepts a 1kHz sampling rate of the biosignal, therefore the sampling rate from the real time UDP out must be 1kHz.
\end{sphinxadmonition}

\begin{sphinxadmonition}{important}{Important:}
\sphinxAtStartPar
bossdevice RESEARCH can take maximum of 128 EEG channels only, and 8 Aux/Bipoloar/EMG/ECG channels.
\end{sphinxadmonition}

\sphinxAtStartPar
Please follow the instructions provided with turbolink from brain products in order to set it up correctly to be able to use it with bossdevice RESEARCH.


\chapter{Matlab API Documentation}
\label{\detokenize{4_api_documentation:matlab-api-documentation}}\label{\detokenize{4_api_documentation::doc}}
\sphinxAtStartPar
The bossdevice API is a set of low\sphinxhyphen{}level functions that allows user to control the bossdevice in a user\sphinxhyphen{}friendly programmable manner. The combined execution of such low level functions allows the bossdevice to perform certain high level tasks that could be amongst but not limited to generating various number and combination of TTL outputs, timed in a certain fashion, TTL output at a certain Phase of an ongoing EEG Oscillation and guided by the Oscillation Amplitude Thresholds.

\begin{sphinxadmonition}{important}{Important:}
\sphinxAtStartPar
Please downloaded Matlab 2021a, Simulink Real\sphinxhyphen{}Time and Simulink Coder, Simulink and MATLAB Coder Adds\sphinxhyphen{}On from Matlab toolboxes, download the repository from master branch as detailed on “Downloads \& Dependencies” chapter of the user manual. Add the bossdevice firmware and repository to the search path of Matlab 2021a. Observe from command line that if bossdevice firmware has been loaded from intended path, otherwise many copies of API are available on your device and Matlab 2021a is confused about choosing one therefore it is recommended to change the current directory of Matlab 2021a to the correct path of bossdevice latest API.
\end{sphinxadmonition}


\section{Initialization}
\label{\detokenize{4_api_documentation:initialization}}

\subsection{Description}
\label{\detokenize{4_api_documentation:description}}
\sphinxAtStartPar
The bossdevice API can be initialized as a MATLAB Class object and provides access to its methods and functions to generate various TTL outputs and define multiple brain states upon which the predefined TTL output sequence to be released.


\subsection{Syntax}
\label{\detokenize{4_api_documentation:syntax}}
\begin{sphinxVerbatim}[commandchars=\\\{\}]
\PYG{p}{[}\PYG{n}{MATLAB\PYGZus{}class\PYGZus{}object}\PYG{p}{]}\PYG{+w}{ }\PYG{p}{=}\PYG{+w}{ }\PYG{n}{bossdevice}\PYG{p}{;}
\end{sphinxVerbatim}


\subsection{Example}
\label{\detokenize{4_api_documentation:example}}
\begin{sphinxVerbatim}[commandchars=\\\{\}]
\PYG{n}{bd}\PYG{p}{=}\PYG{n}{bossdevice}\PYG{p}{;}\PYG{+w}{ }\PYG{c}{\PYGZpc{} \PYGZdq{}bd\PYGZdq{} class object is created in MATLAB workspace}
\end{sphinxVerbatim}


\bigskip\hrule\bigskip



\section{sendPulse}
\label{\detokenize{4_api_documentation:sendpulse}}

\subsection{Description}
\label{\detokenize{4_api_documentation:id1}}
\sphinxAtStartPar
sendPulse method of the bossdevice class allows you to generate 1 TTL pusle at a specified Output port (out of 4) of the bossdevice.


\subsection{Syntax}
\label{\detokenize{4_api_documentation:id2}}
\begin{sphinxVerbatim}[commandchars=\\\{\}]
\PYG{n}{obj}\PYG{p}{.}\PYG{n}{sendPulse}\PYG{p}{(}\PYG{p}{[}\PYG{n}{bossdevice\PYGZus{}port\PYGZus{}number}\PYG{p}{]}\PYG{p}{)}\PYG{+w}{ }\PYG{c}{\PYGZpc{} [bossdevice\PYGZus{}port\PYGZus{}number] is non\PYGZhy{}zero positive integer less than or equal to 4}
\end{sphinxVerbatim}


\subsection{Example}
\label{\detokenize{4_api_documentation:id3}}
\begin{sphinxVerbatim}[commandchars=\\\{\}]
\PYG{n}{bd}\PYG{p}{.}\PYG{n}{sendPulse}\PYG{p}{(}\PYG{l+m+mi}{1}\PYG{p}{)}\PYG{+w}{ }\PYG{c}{\PYGZpc{} delivers a single pulse at the Output port 1}
\PYG{n}{bd}\PYG{p}{.}\PYG{n}{sendPulse}\PYG{p}{(}\PYG{l+m+mi}{3}\PYG{p}{)}\PYG{+w}{ }\PYG{c}{\PYGZpc{} delivers a single pulse at the Output port 3}
\end{sphinxVerbatim}


\bigskip\hrule\bigskip



\section{manualPulse}
\label{\detokenize{4_api_documentation:manualpulse}}

\subsection{Description}
\label{\detokenize{4_api_documentation:id4}}
\sphinxAtStartPar
manualPulse method of bossdevice allows you to deliver a pre\sphinxhyphen{}built sequence of up to 1500 TTL outputs where each TTL output is timed relative to the onset of manualPulse command, delivers on a specified Output port and also delivers an 8\sphinxhyphen{}bit marker assigned to each TTL output as well.
The pre\sphinxhyphen{}built sequence is generated using another command (see below: configure\_time\_port\_marker) however the bossdevice starts delivering the pulses configured before when manualPulse method is executed.


\subsection{Syntax}
\label{\detokenize{4_api_documentation:id5}}
\begin{sphinxVerbatim}[commandchars=\\\{\}]
\PYG{n}{obj}\PYG{p}{.}\PYG{n}{manualPulse}\PYG{+w}{ }\PYG{c}{\PYGZpc{} start the previously configured TTL pulse sequences in [time port marker] style.}
\end{sphinxVerbatim}


\subsection{Example}
\label{\detokenize{4_api_documentation:id6}}
\begin{sphinxVerbatim}[commandchars=\\\{\}]
\PYG{n}{bd}\PYG{p}{.}\PYG{n}{manualPulse}
\end{sphinxVerbatim}


\bigskip\hrule\bigskip



\section{configure\_time\_port\_marker}
\label{\detokenize{4_api_documentation:configure-time-port-marker}}

\subsection{Description}
\label{\detokenize{4_api_documentation:id7}}
\sphinxAtStartPar
configure\_time\_port\_marker method of bossdevice allows you to prepare a sequence of TTL output in {[}time port marker{]} vector style. Maximum of 1500 TTL outputs can be loaded onto this configuration method. In order to prepare a single TTL output in {[}time port marker{]} style, just give input argument as {[}0 1 1{]} vector.
Each TTL output in this vector has three elements , first index being time, second index being the port number of bossdevice to deliver the output pulse and the last index being the 8\sphinxhyphen{}bit event marker to be generated and written to the Biosignal processor for respective TTL output.


\subsection{Syntax}
\label{\detokenize{4_api_documentation:id8}}
\begin{sphinxVerbatim}[commandchars=\\\{\}]
\PYG{n}{obj}\PYG{p}{.}\PYG{n}{configure\PYGZus{}time\PYGZus{}port\PYGZus{}marker}\PYG{p}{(}\PYG{p}{[}\PYG{n}{ttloutput\PYGZus{}sequence\PYGZus{}vector}\PYG{p}{]}
\PYG{c}{\PYGZpc{} [ttloutput\PYGZus{}sequence vector] is an Nx3 vector , whereas N is number of TTL outputs and the three elements being the time onset, port number and event marker}
\end{sphinxVerbatim}


\subsection{Example}
\label{\detokenize{4_api_documentation:id9}}
\begin{sphinxVerbatim}[commandchars=\\\{\}]
\PYG{n}{bd}\PYG{p}{.}\PYG{n}{configure\PYGZus{}time\PYGZus{}port\PYGZus{}marker}\PYG{p}{(}\PYG{p}{[}\PYG{l+m+mi}{2}\PYG{+w}{ }\PYG{l+m+mi}{3}\PYG{+w}{ }\PYG{l+m+mi}{144}\PYG{p}{]}\PYG{p}{)}\PYG{+w}{ }\PYG{c}{\PYGZpc{}configures a TTL output sequence in which the pulse will be delivered 2 seconds after the manualPulse command execution, on the 3rd port and event writes 144 marker on the Biosignal processor}
\PYG{n}{bd}\PYG{p}{.}\PYG{n}{configure\PYGZus{}time\PYGZus{}port\PYGZus{}marker}\PYG{p}{(}\PYG{p}{[}\PYG{l+m+mi}{3}\PYG{+w}{ }\PYG{l+m+mi}{3}\PYG{+w}{ }\PYG{l+m+mi}{144}\PYG{p}{;}\PYG{n}{4}\PYG{+w}{ }\PYG{l+s}{2}\PYG{+w}{ }\PYG{l+s}{128])}\PYG{+w}{  }\PYG{l+s}{\PYGZpc{}configures}\PYG{+w}{ }\PYG{l+s}{2}\PYG{+w}{ }\PYG{l+s}{TTL}\PYG{+w}{ }\PYG{l+s}{output}\PYG{+w}{ }\PYG{l+s}{sequences}\PYG{+w}{ }\PYG{l+s}{in}\PYG{+w}{ }\PYG{l+s}{which}\PYG{+w}{ }\PYG{l+s}{the}\PYG{+w}{ }\PYG{l+s}{first}\PYG{+w}{ }\PYG{l+s}{pulse}\PYG{+w}{ }\PYG{l+s}{will}\PYG{+w}{ }\PYG{l+s}{be}\PYG{+w}{ }\PYG{l+s}{delivered}\PYG{+w}{ }\PYG{l+s}{3}\PYG{+w}{ }\PYG{l+s}{seconds}\PYG{+w}{ }\PYG{l+s}{after}\PYG{+w}{ }\PYG{l+s}{the}\PYG{+w}{ }\PYG{l+s}{manualPulse}\PYG{+w}{ }\PYG{l+s}{command}\PYG{+w}{ }\PYG{l+s}{execution,}\PYG{+w}{ }\PYG{l+s}{on}\PYG{+w}{ }\PYG{l+s}{the}\PYG{+w}{ }\PYG{l+s}{3rd}\PYG{+w}{ }\PYG{l+s}{port}\PYG{+w}{ }\PYG{l+s}{and}\PYG{+w}{ }\PYG{l+s}{will}\PYG{+w}{ }\PYG{l+s}{writes}\PYG{+w}{ }\PYG{l+s}{144}\PYG{+w}{ }\PYG{l+s}{marker}\PYG{+w}{ }\PYG{l+s}{on}\PYG{+w}{ }\PYG{l+s}{the}\PYG{+w}{ }\PYG{l+s}{Biosignal}\PYG{+w}{ }\PYG{l+s}{processor,}\PYG{+w}{ }\PYG{l+s}{and}\PYG{+w}{ }\PYG{l+s}{then}\PYG{+w}{ }\PYG{l+s}{the}\PYG{+w}{ }\PYG{l+s}{second}\PYG{+w}{ }\PYG{l+s}{pulse}\PYG{+w}{ }\PYG{l+s}{will}\PYG{+w}{ }\PYG{l+s}{be}\PYG{+w}{ }\PYG{l+s}{delivered}\PYG{+w}{ }\PYG{l+s}{4}\PYG{+w}{ }\PYG{l+s}{seconds}\PYG{+w}{ }\PYG{l+s}{after}\PYG{+w}{ }\PYG{l+s}{the}\PYG{+w}{ }\PYG{l+s}{manualPulse}\PYG{+w}{ }\PYG{l+s}{command}\PYG{+w}{ }\PYG{l+s}{execution,}\PYG{+w}{ }\PYG{l+s}{on}\PYG{+w}{ }\PYG{l+s}{the}\PYG{+w}{ }\PYG{l+s}{2nd}\PYG{+w}{ }\PYG{l+s}{port}\PYG{+w}{ }\PYG{l+s}{and}\PYG{+w}{ }\PYG{l+s}{writes}\PYG{+w}{ }\PYG{l+s}{128}\PYG{+w}{ }\PYG{l+s}{event}\PYG{+w}{ }\PYG{l+s}{marker}\PYG{+w}{ }\PYG{l+s}{on}\PYG{+w}{ }\PYG{l+s}{the}\PYG{+w}{ }\PYG{l+s}{Biosignal}\PYG{+w}{ }\PYG{l+s}{processor,}
\end{sphinxVerbatim}


\bigskip\hrule\bigskip



\section{stop}
\label{\detokenize{4_api_documentation:stop}}

\subsection{Description}
\label{\detokenize{4_api_documentation:id10}}
\sphinxAtStartPar
Quite often stopping the TTL output sequence is required before the already conigured and long sequence ends. The stop method allows you to stop further generation and delivery of any sequence loaded onto the TTL output generator of the bossdevice.
The last configured sequence can again be started by using “manualPulse” method after the stop command.


\subsection{Syntax}
\label{\detokenize{4_api_documentation:id11}}
\begin{sphinxVerbatim}[commandchars=\\\{\}]
\PYG{n}{obj}\PYG{p}{.}\PYG{n}{stop}\PYG{+w}{ }\PYG{c}{\PYGZpc{} stops the generation of output TTL pulses from bossdevice however the class object remain intact.}
\end{sphinxVerbatim}


\subsection{Example}
\label{\detokenize{4_api_documentation:id12}}
\begin{sphinxVerbatim}[commandchars=\\\{\}]
\PYG{n}{bd}\PYG{p}{.}\PYG{n}{stop}
\end{sphinxVerbatim}


\bigskip\hrule\bigskip



\section{eeg\_channels}
\label{\detokenize{4_api_documentation:eeg-channels}}

\subsection{Description}
\label{\detokenize{4_api_documentation:id13}}\begin{quote}

\sphinxAtStartPar
bossdevice can work on maximum of 128 EEG channels, however the minimum number of channels required to work with EEG associated features of bossdevice have been kept flexible and can be defined as in number of EEG cannnels being streamed from your Biosignal Processor.
\end{quote}


\subsection{Syntax}
\label{\detokenize{4_api_documentation:id14}}
\begin{sphinxVerbatim}[commandchars=\\\{\}]
\PYG{n}{obj}\PYG{p}{.}\PYG{n}{eeg\PYGZus{}channels}\PYG{p}{=}\PYG{p}{[}\PYG{n}{number\PYGZus{}of\PYGZus{}streaming\PYGZus{}eeg\PYGZus{}channels}\PYG{p}{]}\PYG{+w}{ }\PYG{c}{\PYGZpc{} [number\PYGZus{}of\PYGZus{}streaming\PYGZus{}eeg\PYGZus{}channels] is an integer from 1 up till 128.}
\end{sphinxVerbatim}


\subsection{Example}
\label{\detokenize{4_api_documentation:id15}}
\begin{sphinxVerbatim}[commandchars=\\\{\}]
\PYG{n}{bd}\PYG{p}{.}\PYG{n}{eeg\PYGZus{}channels}\PYG{p}{=}\PYG{l+m+mi}{32}\PYG{p}{;}\PYG{+w}{ }\PYG{c}{\PYGZpc{} informing bossbox that 32 EEG channels are being streamed frm your Biosignal Processor}
\PYG{n}{bd}\PYG{p}{.}\PYG{n}{eeg\PYGZus{}channels}\PYG{p}{=}\PYG{l+m+mi}{64}\PYG{p}{;}\PYG{+w}{ }\PYG{c}{\PYGZpc{} informing bossbox that 64 EEG channels are being streamed frm your Biosignal Processor}
\PYG{n}{bd}\PYG{p}{.}\PYG{n}{eeg\PYGZus{}channels}\PYG{p}{=}\PYG{l+m+mi}{128}\PYG{p}{;}\PYG{+w}{ }\PYG{c}{\PYGZpc{} informing bossbox that 128 EEG channels are being streamed frm your Biosignal Processor}
\PYG{n}{bd}\PYG{p}{.}\PYG{n}{eeg\PYGZus{}channels}\PYG{p}{=}\PYG{l+m+mi}{5}\PYG{p}{;}\PYG{+w}{ }\PYG{c}{\PYGZpc{} informing bossbox that 5 EEG channels are being streamed frm your Biosignal Processor}
\end{sphinxVerbatim}


\bigskip\hrule\bigskip



\section{aux\_channels}
\label{\detokenize{4_api_documentation:aux-channels}}

\subsection{Description}
\label{\detokenize{4_api_documentation:id16}}\begin{quote}

\sphinxAtStartPar
bossdevice can work on maximum of 8 Auxiliary/Biopolar/EMG channels, however the minimum number of channels required to work with Auxiliary Channels associated features of bossdevice have been kept flexible and can be defined as in number of Aux cannnels being streamed from your Biosignal Processor.
\end{quote}

\begin{sphinxadmonition}{important}{Important:}
\sphinxAtStartPar
Auxiliary channels should always be the last 8 or N channels in your Protocol of Biosignal Processor otherwise bossdevice will assume them to be EEG channels
\end{sphinxadmonition}


\subsection{Syntax}
\label{\detokenize{4_api_documentation:id17}}
\begin{sphinxVerbatim}[commandchars=\\\{\}]
\PYG{n}{obj}\PYG{p}{.}\PYG{n}{aux\PYGZus{}channels}\PYG{p}{=}\PYG{p}{[}\PYG{n}{number\PYGZus{}of\PYGZus{}streaming\PYGZus{}aux\PYGZus{}channels}\PYG{p}{]}\PYG{+w}{ }\PYG{c}{\PYGZpc{} [number\PYGZus{}of\PYGZus{}streaming\PYGZus{}eeg\PYGZus{}channels] is an integer from 1 up till 8.}
\end{sphinxVerbatim}


\subsection{Example}
\label{\detokenize{4_api_documentation:id18}}
\begin{sphinxVerbatim}[commandchars=\\\{\}]
\PYG{n}{bd}\PYG{p}{.}\PYG{n}{eeg\PYGZus{}channels}\PYG{p}{=}\PYG{l+m+mi}{8}\PYG{p}{;}\PYG{+w}{ }\PYG{c}{\PYGZpc{} informing bossbox that 8 aux/Emg/Bipolar channels are being streamed frm your Biosignal Processor}
\PYG{n}{bd}\PYG{p}{.}\PYG{n}{eeg\PYGZus{}channels}\PYG{p}{=}\PYG{l+m+mi}{2}\PYG{p}{;}\PYG{+w}{ }\PYG{c}{\PYGZpc{} informing bossbox that 2 aux/EMG/Bipolar channels are being}
\end{sphinxVerbatim}


\bigskip\hrule\bigskip



\section{spatial\_filter\_weights}
\label{\detokenize{4_api_documentation:spatial-filter-weights}}

\subsection{Description}
\label{\detokenize{4_api_documentation:id19}}\begin{quote}

\sphinxAtStartPar
Spatial filtering of the signals is an important step before commencing real\sphinxhyphen{}time brain states detection. The bossdevice allows to specify two different spatially filtered channels and can detect brain states (based on the target phase \& amplitude) for both of these spatial filtered channels in parallel. If any one of them is specified then the other one is ignored and assigned 0 weights. The index of eeg\_channels being streamed from the Biosignal processor e.g. ActiCHamp or NeurOne corrosponds to the index of the weights matrix explained below.
\end{quote}


\subsection{Syntax}
\label{\detokenize{4_api_documentation:id20}}
\begin{sphinxVerbatim}[commandchars=\\\{\}]
\PYG{n}{obj}\PYG{p}{.}\PYG{n}{spatial\PYGZus{}filter\PYGZus{}weights}\PYG{p}{=}\PYG{p}{[}\PYG{n}{weights\PYGZus{}for\PYGZus{}both\PYGZus{}channels}\PYG{p}{]}
\PYG{c}{\PYGZpc{}[weights\PYGZus{}for\PYGZus{}both\PYGZus{}channels] is a Nx2 matrix where N is Number of EEG channels specified before (see eeg\PYGZus{}channels) and the first column is first spatially filtered channel, similarly the second column is the second channel. Each element of the column vector is a \PYGZdq{}normalized weight\PYGZdq{} w.r.t. particular column such that the sum of all weights of a particular channel is zero.}
\end{sphinxVerbatim}


\subsection{Example}
\label{\detokenize{4_api_documentation:id21}}
\begin{sphinxVerbatim}[commandchars=\\\{\}]
\PYG{c}{\PYGZpc{} assume that there are 5 EEG channels}
\PYG{n}{bd}\PYG{p}{.}\PYG{n}{spatial\PYGZus{}filter\PYGZus{}weights}\PYG{p}{=}\PYG{p}{[}\PYG{l+m+mi}{0}\PYG{p}{;}\PYG{+w}{ }\PYG{o}{\PYGZhy{}}\PYG{l+m+mf}{0.5}\PYG{p}{;}\PYG{+w}{ }\PYG{l+m+mi}{1}\PYG{p}{;}\PYG{+w}{ }\PYG{o}{\PYGZhy{}}\PYG{l+m+mf}{0.5}\PYG{p}{;}\PYG{+w}{ }\PYG{l+m+mi}{0}\PYG{p}{]}\PYG{p}{;}\PYG{+w}{ }\PYG{c}{\PYGZpc{} the channel 1 will be assigned the given weights and all the eeg\PYGZus{}channels for 2nd spatially filtered channel would have 0 weights.}
\PYG{c}{\PYGZpc{} assume that there are 3 EEG channels}
\PYG{n}{bd}\PYG{p}{.}\PYG{n}{spatial\PYGZus{}filter\PYGZus{}weights}\PYG{p}{=}\PYG{p}{[}\PYG{l+m+mi}{0}\PYG{+w}{ }\PYG{o}{\PYGZhy{}}\PYG{l+m+mf}{0.5}\PYG{p}{;}\PYG{+w}{ }\PYG{n}{1}\PYG{+w}{ }\PYG{l+s}{1}\PYG{p}{;}\PYG{+w}{ }\PYG{l+m+mi}{0}\PYG{+w}{ }\PYG{o}{\PYGZhy{}}\PYG{l+m+mf}{0.5}\PYG{p}{]}\PYG{p}{;}\PYG{+w}{ }\PYG{c}{\PYGZpc{} the channel 1 will be assigned the weights given on first column and channel 2 will be assigned the weights given on 2nd column.}
\end{sphinxVerbatim}


\bigskip\hrule\bigskip



\section{phase\_target}
\label{\detokenize{4_api_documentation:phase-target}}

\subsection{Description}
\label{\detokenize{4_api_documentation:id22}}\begin{quote}

\sphinxAtStartPar
bossdevice contains 3 built in Oscillatory phase prediction models each for Theta (4\sphinxhyphen{}8 Hz), Alpha (8\sphinxhyphen{}14Hz) and Beta (14\sphinxhyphen{}30Hz) frequency bands. Real\sphinxhyphen{}time phase detection can be performed for maximum of 2 different, pre\sphinxhyphen{}specified spatially filtered channels in parallel. This method allows to define a target phase in radians for a particular frequency band.
\end{quote}


\subsection{Syntax}
\label{\detokenize{4_api_documentation:id23}}
\begin{sphinxVerbatim}[commandchars=\\\{\}]
\PYG{n}{obj}\PYG{p}{.}\PYG{n}{alpha}\PYG{p}{.}\PYG{n}{phase\PYGZus{}target}\PYG{p}{(}\PYG{p}{[}\PYG{n}{spatial\PYGZus{}filter\PYGZus{}channel\PYGZus{}number}\PYG{p}{]}\PYG{p}{)}\PYG{p}{=}\PYG{n}{phase\PYGZus{}angle\PYGZus{}in\PYGZus{}radians}\PYG{+w}{ }\PYG{c}{\PYGZpc{} sets target phase of Alpha Oscillation}
\PYG{n}{obj}\PYG{p}{.}\PYG{n}{theta}\PYG{p}{.}\PYG{n}{phase\PYGZus{}target}\PYG{p}{(}\PYG{p}{[}\PYG{n}{spatial\PYGZus{}filter\PYGZus{}channel\PYGZus{}number}\PYG{p}{]}\PYG{p}{)}\PYG{p}{=}\PYG{n}{phase\PYGZus{}angle\PYGZus{}in\PYGZus{}radians}\PYG{+w}{ }\PYG{c}{\PYGZpc{} sets target phase of Theta Oscillation}
\PYG{n}{obj}\PYG{p}{.}\PYG{n}{beta}\PYG{p}{.}\PYG{n}{phase\PYGZus{}target}\PYG{p}{(}\PYG{p}{[}\PYG{n}{spatial\PYGZus{}filter\PYGZus{}channel\PYGZus{}number}\PYG{p}{]}\PYG{p}{)}\PYG{p}{=}\PYG{n}{phase\PYGZus{}angle\PYGZus{}in\PYGZus{}radians}\PYG{+w}{ }\PYG{c}{\PYGZpc{} sets target phase of Beta Oscillation}
\PYG{c}{\PYGZpc{} [spatial\PYGZus{}filter\PYGZus{}channel\PYGZus{}number] is an integer having value 1 or 2.}
\PYG{c}{\PYGZpc{} [phase\PYGZus{}angle\PYGZus{}in\PYGZus{}radians] is 0 for Peak, pi for Trough, \PYGZhy{}pi/2 for Rising Flank \PYGZam{} +pi/2 for Falling Flank}
\end{sphinxVerbatim}

\begin{figure}[htbp]
\centering
\capstart

\noindent\sphinxincludegraphics{{Figure10_phases}.png}
\caption{Radian values of several target phases in bossdevice research}\label{\detokenize{4_api_documentation:id53}}\end{figure}


\subsection{Example}
\label{\detokenize{4_api_documentation:id24}}
\begin{sphinxVerbatim}[commandchars=\\\{\}]
\PYG{n}{bd}\PYG{p}{.}\PYG{n}{alpha}\PYG{p}{.}\PYG{n}{phase\PYGZus{}target}\PYG{p}{(}\PYG{l+m+mi}{1}\PYG{p}{)}\PYG{p}{=}\PYG{l+m+mi}{0}\PYG{p}{;}\PYG{+w}{ }\PYG{c}{\PYGZpc{} bossdevice is loaded to detect peak (0 radians) from first spatially filtered channel  with the assumption that the Oscillation is in Alpha frequency band}
\PYG{n}{bd}\PYG{p}{.}\PYG{n}{beta}\PYG{p}{.}\PYG{n}{phase\PYGZus{}target}\PYG{p}{(}\PYG{l+m+mi}{2}\PYG{p}{)}\PYG{p}{=}\PYG{n+nb}{pi}\PYG{p}{;}\PYG{+w}{ }\PYG{c}{\PYGZpc{} bossdevice is loaded to detect trough(pi radians) from second spatially filtered channel with the assumption that the Oscillation is in Beta frequency band}
\end{sphinxVerbatim}


\bigskip\hrule\bigskip



\section{phase\_plusminus}
\label{\detokenize{4_api_documentation:phase-plusminus}}

\subsection{Description}
\label{\detokenize{4_api_documentation:id25}}\begin{quote}

\sphinxAtStartPar
Defining absolute target phase angles in order to detect a brain state is often prone to error mainly due to the resolution of data obtained after sampling rate transition. In order to overcome this digitization resolution error another parameter has to be defined such that the vicinities of the target phase shall be made clear to the detection algorithm. For an instance, while detecting a 0 radians phase, the phase vector would probably look like this {[}\sphinxhyphen{}0.001324 \sphinxhyphen{}0.00234 0.00243 0.004324{]}, and since none of them are mathematically equivalent to zero therefore in order to not allow to skip such Oscillatory Peak events and to increase the accuracy of the phase detection, a tolerance value is to be provided.
\end{quote}


\subsection{Syntax}
\label{\detokenize{4_api_documentation:id26}}
\begin{sphinxVerbatim}[commandchars=\\\{\}]
\PYG{n}{obj}\PYG{p}{.}\PYG{n}{alpha}\PYG{p}{.}\PYG{n}{phase\PYGZus{}plusminus}\PYG{p}{(}\PYG{p}{[}\PYG{n}{spatial\PYGZus{}filter\PYGZus{}channel\PYGZus{}number}\PYG{p}{]}\PYG{p}{)}\PYG{p}{=}\PYG{n}{phase\PYGZus{}tolerance\PYGZus{}in\PYGZus{}radians}\PYG{+w}{ }\PYG{c}{\PYGZpc{} sets target phase tolerance of Alpha Oscillation}
\PYG{n}{obj}\PYG{p}{.}\PYG{n}{theta}\PYG{p}{.}\PYG{n}{phase\PYGZus{}plusminus}\PYG{p}{(}\PYG{p}{[}\PYG{n}{spatial\PYGZus{}filter\PYGZus{}channel\PYGZus{}number}\PYG{p}{]}\PYG{p}{)}\PYG{p}{=}\PYG{n}{phase\PYGZus{}tolerance\PYGZus{}in\PYGZus{}radians}\PYG{+w}{ }\PYG{c}{\PYGZpc{} sets target phase tolerance of Theta Oscillation}
\PYG{n}{obj}\PYG{p}{.}\PYG{n}{beta}\PYG{p}{.}\PYG{n}{phase\PYGZus{}plusminus}\PYG{p}{(}\PYG{p}{[}\PYG{n}{spatial\PYGZus{}filter\PYGZus{}channel\PYGZus{}number}\PYG{p}{]}\PYG{p}{)}\PYG{p}{=}\PYG{n}{phase\PYGZus{}tolerance\PYGZus{}in\PYGZus{}radians}\PYG{+w}{ }\PYG{c}{\PYGZpc{} sets target phase of Beta Oscillation}
\PYG{c}{\PYGZpc{} [spatial\PYGZus{}filter\PYGZus{}channel\PYGZus{}number] is an integer having value 1 or 2.}
\PYG{c}{\PYGZpc{}Note: While targeting a random phase, the tolerance could go as high as possible i.e. pi radians or Nxpi radians e.g. 2pi etc}
\end{sphinxVerbatim}


\subsection{Example}
\label{\detokenize{4_api_documentation:id27}}
\begin{sphinxVerbatim}[commandchars=\\\{\}]
\PYG{n}{bd}\PYG{p}{.}\PYG{n}{alpha}\PYG{p}{.}\PYG{n}{phase\PYGZus{}plusminus}\PYG{p}{(}\PYG{l+m+mi}{1}\PYG{p}{)}\PYG{p}{=}\PYG{n+nb}{pi}\PYG{o}{/}\PYG{l+m+mi}{40}\PYG{p}{;}\PYG{+w}{ }\PYG{c}{\PYGZpc{} bossdevice is loaded to detect the setted target phase with a tolerance of pi/40 radians from first spatially filtered channel with the assumption that the Oscillation is in Alpha frequency band}
\PYG{n}{bd}\PYG{p}{.}\PYG{n}{beta}\PYG{p}{.}\PYG{n}{phase\PYGZus{}plusminus}\PYG{p}{(}\PYG{l+m+mi}{2}\PYG{p}{)}\PYG{p}{=}\PYG{n+nb}{pi}\PYG{o}{/}\PYG{l+m+mi}{70}\PYG{p}{;}\PYG{+w}{ }\PYG{c}{\PYGZpc{} ossdevice is loaded to detect the setted target phase with a tolerance of pi/70 radians from second spatially filtered channel with the assumption that the Oscillation is in Beta frequency band}
\end{sphinxVerbatim}


\bigskip\hrule\bigskip



\section{amplitude\_min}
\label{\detokenize{4_api_documentation:amplitude-min}}

\subsection{Description}
\label{\detokenize{4_api_documentation:id28}}\begin{quote}

\sphinxAtStartPar
Defining minimum amplitude threshold in order to match a specific brain state is important. The bossdevice allows to define minimum amplitude threshold in micro volts that must be reached along with other brain state associated conditions such as maximum amplitude, phase target, and phase tolerance in order to be able to declare the brain state as detected and generate TTL output sequence. The bossdevice contains 3 built in Oscillatory minimum amplitude estimation models each for Theta (4\sphinxhyphen{}8 Hz), Alpha (8\sphinxhyphen{}14Hz) and Beta (14\sphinxhyphen{}30Hz) frequency bands. Real\sphinxhyphen{}time minimum amplitude detection can be performed for maximum of 2 different, pre\sphinxhyphen{}specified spatially filtered channels in parallel.
\end{quote}


\subsection{Syntax}
\label{\detokenize{4_api_documentation:id29}}
\begin{sphinxVerbatim}[commandchars=\\\{\}]
\PYG{n}{obj}\PYG{p}{.}\PYG{n}{alpha}\PYG{p}{.}\PYG{n}{amplitude\PYGZus{}min}\PYG{p}{(}\PYG{p}{[}\PYG{n}{spatial\PYGZus{}filter\PYGZus{}channel\PYGZus{}number}\PYG{p}{]}\PYG{p}{)}\PYG{p}{=}\PYG{n}{min\PYGZus{}amplitude\PYGZus{}microV}\PYG{+w}{ }\PYG{c}{\PYGZpc{} sets minimum amplitude threshold of Alpha Oscillation}
\PYG{n}{obj}\PYG{p}{.}\PYG{n}{theta}\PYG{p}{.}\PYG{n}{amplitude\PYGZus{}min}\PYG{p}{(}\PYG{p}{[}\PYG{n}{spatial\PYGZus{}filter\PYGZus{}channel\PYGZus{}number}\PYG{p}{]}\PYG{p}{)}\PYG{p}{=}\PYG{n}{min\PYGZus{}amplitude\PYGZus{}microV}\PYG{+w}{ }\PYG{c}{\PYGZpc{} sets minimum amplitude threshold of Theta Oscillation}
\PYG{n}{obj}\PYG{p}{.}\PYG{n}{beta}\PYG{p}{.}\PYG{n}{amplitude\PYGZus{}min}\PYG{p}{(}\PYG{p}{[}\PYG{n}{spatial\PYGZus{}filter\PYGZus{}channel\PYGZus{}number}\PYG{p}{]}\PYG{p}{)}\PYG{p}{=}\PYG{n}{min\PYGZus{}amplitude\PYGZus{}microV}\PYG{+w}{ }\PYG{c}{\PYGZpc{} sets minimum amplitude threshold of Beta Oscillation}
\PYG{c}{\PYGZpc{} [spatial\PYGZus{}filter\PYGZus{}channel\PYGZus{}number] is an integer having value 1 or 2.}
\end{sphinxVerbatim}


\subsection{Example}
\label{\detokenize{4_api_documentation:id30}}
\begin{sphinxVerbatim}[commandchars=\\\{\}]
\PYG{n}{bd}\PYG{p}{.}\PYG{n}{alpha}\PYG{p}{.}\PYG{n}{amplitude\PYGZus{}min}\PYG{p}{(}\PYG{l+m+mi}{1}\PYG{p}{)}\PYG{p}{=}\PYG{l+m+mi}{1000}\PYG{p}{;}\PYG{+w}{ }\PYG{c}{\PYGZpc{} bossdevice is loaded to monitor the specified minimum amplitude threshold from first spatially filtered channel with the assumption that the Oscillation is in Alpha frequency band}
\PYG{n}{bd}\PYG{p}{.}\PYG{n}{beta}\PYG{p}{.}\PYG{n}{amplitude\PYGZus{}min}\PYG{p}{(}\PYG{l+m+mi}{2}\PYG{p}{)}\PYG{p}{=}\PYG{l+m+mf}{1e5}\PYG{p}{;}\PYG{+w}{ }\PYG{c}{\PYGZpc{} bossdevice is loaded to monitor the specified minimum amplitude threshold from second spatially filtered channel with the assumption that the Oscillation is in Beta frequency band}
\end{sphinxVerbatim}


\bigskip\hrule\bigskip



\section{amplitude\_max}
\label{\detokenize{4_api_documentation:amplitude-max}}

\subsection{Description}
\label{\detokenize{4_api_documentation:id31}}\begin{quote}

\sphinxAtStartPar
Defining maximum amplitude threshold in order to match a specific brain state is important. The bossdevice allows to define maximum amplitude threshold in micro volts that must be not be reached in order to be able to declare the brain state as detected and generare TTL output sequence. The bossdevice contains 3 built in Oscillatory maximum amplitude estimation models each for Theta (4\sphinxhyphen{}8 Hz), Alpha (8\sphinxhyphen{}14Hz) and Beta (14\sphinxhyphen{}30Hz) frequency bands. Real\sphinxhyphen{}time maximum amplitude detection can be performed for maximum of 2 different, pre\sphinxhyphen{}specified spatially filtered channels in parallel.
\end{quote}


\subsection{Syntax}
\label{\detokenize{4_api_documentation:id32}}
\begin{sphinxVerbatim}[commandchars=\\\{\}]
\PYG{n}{obj}\PYG{p}{.}\PYG{n}{alpha}\PYG{p}{.}\PYG{n}{amplitude\PYGZus{}max}\PYG{p}{(}\PYG{p}{[}\PYG{n}{spatial\PYGZus{}filter\PYGZus{}channel\PYGZus{}number}\PYG{p}{]}\PYG{p}{)}\PYG{p}{=}\PYG{n}{max\PYGZus{}amplitude\PYGZus{}microV}\PYG{+w}{ }\PYG{c}{\PYGZpc{} sets maximum amplitude threshold of Alpha Oscillation}
\PYG{n}{obj}\PYG{p}{.}\PYG{n}{theta}\PYG{p}{.}\PYG{n}{amplitude\PYGZus{}max}\PYG{p}{(}\PYG{p}{[}\PYG{n}{spatial\PYGZus{}filter\PYGZus{}channel\PYGZus{}number}\PYG{p}{]}\PYG{p}{)}\PYG{p}{=}\PYG{n}{max\PYGZus{}amplitude\PYGZus{}microV}\PYG{+w}{ }\PYG{c}{\PYGZpc{} sets maximum amplitude threshold of Theta Oscillation}
\PYG{n}{obj}\PYG{p}{.}\PYG{n}{beta}\PYG{p}{.}\PYG{n}{amplitude\PYGZus{}max}\PYG{p}{(}\PYG{p}{[}\PYG{n}{spatial\PYGZus{}filter\PYGZus{}channel\PYGZus{}number}\PYG{p}{]}\PYG{p}{)}\PYG{p}{=}\PYG{n}{max\PYGZus{}amplitude\PYGZus{}microV}\PYG{+w}{ }\PYG{c}{\PYGZpc{} sets maximum amplitude threshold of Beta Oscillation}
\PYG{c}{\PYGZpc{} [spatial\PYGZus{}filter\PYGZus{}channel\PYGZus{}number] is an integer having value 1 or 2.}
\end{sphinxVerbatim}


\subsection{Example}
\label{\detokenize{4_api_documentation:id33}}
\begin{sphinxVerbatim}[commandchars=\\\{\}]
\PYG{n}{bd}\PYG{p}{.}\PYG{n}{alpha}\PYG{p}{.}\PYG{n}{amplitude\PYGZus{}max}\PYG{p}{(}\PYG{l+m+mi}{1}\PYG{p}{)}\PYG{p}{=}\PYG{l+m+mi}{1000}\PYG{p}{;}\PYG{+w}{ }\PYG{c}{\PYGZpc{} bossdevice is loaded to monitor the specified maximum amplitude threshold from first spatially filtered channel with the assumption that the Oscillation is in Alpha frequency band}
\PYG{n}{bd}\PYG{p}{.}\PYG{n}{beta}\PYG{p}{.}\PYG{n}{amplitude\PYGZus{}max}\PYG{p}{(}\PYG{l+m+mi}{2}\PYG{p}{)}\PYG{p}{=}\PYG{l+m+mf}{1e5}\PYG{p}{;}\PYG{+w}{ }\PYG{c}{\PYGZpc{} bossdevice is loaded to monitor the specified maximum amplitude threshold from second spatially filtered channel with the assumption that the Oscillation is in Beta frequency band}
\end{sphinxVerbatim}


\bigskip\hrule\bigskip



\section{lpf\_fir\_coeffs}
\label{\detokenize{4_api_documentation:lpf-fir-coeffs}}

\subsection{Description}
\label{\detokenize{4_api_documentation:id34}}\begin{quote}

\sphinxAtStartPar
Spatially filtered channel’s signals described in the “spatial\_filter\_weights” function are passed through a low pass FIR filter for anti\sphinxhyphen{}aliasing. The coefficients of this filter can be specified using this function provided that the filter has an order of 100 at maximum.
\end{quote}


\subsection{Syntax}
\label{\detokenize{4_api_documentation:id35}}
\begin{sphinxVerbatim}[commandchars=\\\{\}]
\PYG{n}{obj}\PYG{p}{.}\PYG{n}{alpha}\PYG{p}{.}\PYG{n}{lpf\PYGZus{}fir\PYGZus{}coeffs}\PYG{+w}{ }\PYG{p}{=}\PYG{+w}{ }\PYG{p}{[}\PYG{n}{filter\PYGZus{}coefficients}\PYG{p}{]}\PYG{p}{;}\PYG{+w}{ }\PYG{c}{\PYGZpc{}for Alpha}
\PYG{n}{obj}\PYG{p}{.}\PYG{n}{theta}\PYG{p}{.}\PYG{n}{lpf\PYGZus{}fir\PYGZus{}coeffs}\PYG{+w}{ }\PYG{p}{=}\PYG{+w}{ }\PYG{p}{[}\PYG{n}{filter\PYGZus{}coefficients}\PYG{p}{]}\PYG{p}{;}\PYG{+w}{ }\PYG{c}{\PYGZpc{}for Theta}
\PYG{n}{obj}\PYG{p}{.}\PYG{n}{beta}\PYG{p}{.}\PYG{n}{lpf\PYGZus{}fir\PYGZus{}coeffs}\PYG{+w}{ }\PYG{p}{=}\PYG{+w}{ }\PYG{p}{[}\PYG{n}{filter\PYGZus{}coefficients}\PYG{p}{]}\PYG{p}{;}\PYG{+w}{ }\PYG{c}{\PYGZpc{}for Beta}
\end{sphinxVerbatim}


\subsection{Sampling Rates}
\label{\detokenize{4_api_documentation:sampling-rates}}
\begin{sphinxVerbatim}[commandchars=\\\{\}]
\PYG{n}{Oscillation}\PYG{+w}{         }\PYG{l+s}{Sampling}\PYG{+w}{ }\PYG{l+s}{Rate}
\PYG{n}{Alpha}\PYG{+w}{ }\PYG{p}{(}\PYG{l+m+mi}{8}\PYG{o}{\PYGZhy{}}\PYG{l+m+mi}{14}\PYG{+w}{ }\PYG{n}{Hz}\PYG{p}{)}\PYG{+w}{         }\PYG{l+m+mi}{500}\PYG{+w}{ }\PYG{n}{Hz}
\PYG{n}{Theta}\PYG{+w}{ }\PYG{p}{(}\PYG{l+m+mi}{4}\PYG{o}{\PYGZhy{}}\PYG{l+m+mi}{8}\PYG{+w}{ }\PYG{n}{Hz}\PYG{p}{)}\PYG{+w}{          }\PYG{l+m+mi}{250}\PYG{+w}{ }\PYG{n}{Hz}
\PYG{n}{Beta}\PYG{p}{(}\PYG{l+m+mi}{14}\PYG{o}{\PYGZhy{}}\PYG{l+m+mi}{30}\PYG{+w}{ }\PYG{n}{Hz}\PYG{p}{)}\PYG{+w}{          }\PYG{l+m+mi}{1000}\PYG{+w}{ }\PYG{n}{Hz}
\end{sphinxVerbatim}


\subsection{Example}
\label{\detokenize{4_api_documentation:id36}}
\begin{sphinxVerbatim}[commandchars=\\\{\}]
\PYG{n}{bd}\PYG{p}{.}\PYG{n}{alpha}\PYG{p}{.}\PYG{n}{lpf\PYGZus{}fir\PYGZus{}coeffs}\PYG{+w}{ }\PYG{p}{=}\PYG{+w}{  }\PYG{p}{[}\PYG{n}{filter\PYGZus{}coefficients}\PYG{p}{]}\PYG{p}{;}
\PYG{n}{bd}\PYG{p}{.}\PYG{n}{beta}\PYG{p}{.}\PYG{n}{lpf\PYGZus{}fir\PYGZus{}coeffs}\PYG{+w}{ }\PYG{p}{=}\PYG{+w}{  }\PYG{p}{[}\PYG{n}{filter\PYGZus{}coefficients}\PYG{p}{]}\PYG{p}{;}
\end{sphinxVerbatim}


\bigskip\hrule\bigskip



\section{bpf\_fir\_coeffs}
\label{\detokenize{4_api_documentation:bpf-fir-coeffs}}

\subsection{Description}
\label{\detokenize{4_api_documentation:id37}}\begin{quote}

\sphinxAtStartPar
Low pass filtered channel’s signal described in the “lpf\_fir\_coeffs” function is then pass through an FIR Band Pass filter. The coefficients of this filter can be specified using this function provided that the filter has an order of 100 at maximum.
\end{quote}


\subsection{Syntax}
\label{\detokenize{4_api_documentation:id38}}
\begin{sphinxVerbatim}[commandchars=\\\{\}]
\PYG{n}{obj}\PYG{p}{.}\PYG{n}{alpha}\PYG{p}{.}\PYG{n}{bpf\PYGZus{}fir\PYGZus{}coeffs}\PYG{+w}{ }\PYG{p}{=}\PYG{+w}{ }\PYG{p}{[}\PYG{n}{filter\PYGZus{}coefficients}\PYG{p}{]}\PYG{p}{;}\PYG{+w}{ }\PYG{c}{\PYGZpc{}for Alpha}
\PYG{n}{obj}\PYG{p}{.}\PYG{n}{theta}\PYG{p}{.}\PYG{n}{bpf\PYGZus{}fir\PYGZus{}coeffs}\PYG{+w}{ }\PYG{p}{=}\PYG{+w}{ }\PYG{p}{[}\PYG{n}{filter\PYGZus{}coefficients}\PYG{p}{]}\PYG{p}{;}\PYG{+w}{ }\PYG{c}{\PYGZpc{}for Theta}
\PYG{n}{obj}\PYG{p}{.}\PYG{n}{beta}\PYG{p}{.}\PYG{n}{bpf\PYGZus{}fir\PYGZus{}coeffs}\PYG{+w}{ }\PYG{p}{=}\PYG{+w}{ }\PYG{p}{[}\PYG{n}{filter\PYGZus{}coefficients}\PYG{p}{]}\PYG{p}{;}\PYG{+w}{ }\PYG{c}{\PYGZpc{}for Beta}
\end{sphinxVerbatim}


\subsection{Sampling Rates}
\label{\detokenize{4_api_documentation:id39}}
\begin{sphinxVerbatim}[commandchars=\\\{\}]
\PYG{n}{Oscillation}\PYG{+w}{         }\PYG{l+s}{Sampling}\PYG{+w}{ }\PYG{l+s}{Rate}
\PYG{n}{Alpha}\PYG{+w}{ }\PYG{p}{(}\PYG{l+m+mi}{8}\PYG{o}{\PYGZhy{}}\PYG{l+m+mi}{14}\PYG{+w}{ }\PYG{n}{Hz}\PYG{p}{)}\PYG{+w}{         }\PYG{l+m+mi}{500}\PYG{+w}{ }\PYG{n}{Hz}
\PYG{n}{Theta}\PYG{+w}{ }\PYG{p}{(}\PYG{l+m+mi}{4}\PYG{o}{\PYGZhy{}}\PYG{l+m+mi}{8}\PYG{+w}{ }\PYG{n}{Hz}\PYG{p}{)}\PYG{+w}{          }\PYG{l+m+mi}{250}\PYG{+w}{ }\PYG{n}{Hz}
\PYG{n}{Beta}\PYG{p}{(}\PYG{l+m+mi}{14}\PYG{o}{\PYGZhy{}}\PYG{l+m+mi}{30}\PYG{+w}{ }\PYG{n}{Hz}\PYG{p}{)}\PYG{+w}{          }\PYG{l+m+mi}{1000}\PYG{+w}{ }\PYG{n}{Hz}
\end{sphinxVerbatim}


\subsection{Example}
\label{\detokenize{4_api_documentation:id40}}
\begin{sphinxVerbatim}[commandchars=\\\{\}]
\PYG{n}{bd}\PYG{p}{.}\PYG{n}{alpha}\PYG{p}{.}\PYG{n}{bpf\PYGZus{}fir\PYGZus{}coeffs}\PYG{+w}{ }\PYG{p}{=}\PYG{+w}{ }\PYG{n}{firls}\PYG{p}{(}\PYG{l+m+mi}{80}\PYG{p}{,}\PYG{+w}{ }\PYG{p}{[}\PYG{l+m+mi}{0}\PYG{+w}{ }\PYG{p}{(}\PYG{l+m+mi}{11}\PYG{+w}{ }\PYG{o}{+}\PYG{+w}{ }\PYG{p}{[}\PYG{o}{\PYGZhy{}}\PYG{l+m+mi}{5}\PYG{+w}{ }\PYG{o}{\PYGZhy{}}\PYG{l+m+mi}{2}\PYG{+w}{ }\PYG{o}{+}\PYG{l+m+mi}{2}\PYG{+w}{ }\PYG{o}{+}\PYG{l+m+mi}{5}\PYG{p}{]}\PYG{p}{)}\PYG{+w}{ }\PYG{p}{(}\PYG{l+m+mi}{500}\PYG{o}{/}\PYG{l+m+mi}{2}\PYG{p}{)}\PYG{p}{]}\PYG{o}{/}\PYG{p}{(}\PYG{l+m+mi}{500}\PYG{o}{/}\PYG{l+m+mi}{2}\PYG{p}{)}\PYG{p}{,}\PYG{+w}{ }\PYG{p}{[}\PYG{l+m+mi}{0}\PYG{+w}{ }\PYG{l+m+mi}{0}\PYG{+w}{ }\PYG{l+m+mi}{1}\PYG{+w}{ }\PYG{l+m+mi}{1}\PYG{+w}{ }\PYG{l+m+mi}{0}\PYG{+w}{ }\PYG{l+m+mi}{0}\PYG{p}{]}\PYG{p}{,}\PYG{+w}{ }\PYG{p}{[}\PYG{l+m+mi}{1}\PYG{+w}{ }\PYG{l+m+mi}{1}\PYG{+w}{ }\PYG{l+m+mi}{1}\PYG{p}{]}\PYG{+w}{ }\PYG{p}{)}\PYG{p}{;}\PYG{+w}{ }\PYG{c}{\PYGZpc{}creates FIR Band Pass filter of order \PYGZdq{}80\PYGZdq{} , around peak frequency of 11 Hz and assigned it to Alpha model of bossdevice}
\PYG{n}{bd}\PYG{p}{.}\PYG{n}{beta}\PYG{p}{.}\PYG{n}{bpf\PYGZus{}fir\PYGZus{}coeffs}\PYG{+w}{ }\PYG{p}{=}\PYG{+w}{ }\PYG{n}{firls}\PYG{p}{(}\PYG{l+m+mi}{100}\PYG{p}{,}\PYG{+w}{ }\PYG{p}{[}\PYG{l+m+mi}{0}\PYG{+w}{ }\PYG{p}{(}\PYG{l+m+mi}{19}\PYG{+w}{ }\PYG{o}{+}\PYG{+w}{ }\PYG{p}{[}\PYG{o}{\PYGZhy{}}\PYG{l+m+mi}{5}\PYG{+w}{ }\PYG{o}{\PYGZhy{}}\PYG{l+m+mi}{2}\PYG{+w}{ }\PYG{o}{+}\PYG{l+m+mi}{2}\PYG{+w}{ }\PYG{o}{+}\PYG{l+m+mi}{5}\PYG{p}{]}\PYG{p}{)}\PYG{+w}{ }\PYG{p}{(}\PYG{l+m+mi}{1000}\PYG{o}{/}\PYG{l+m+mi}{2}\PYG{p}{)}\PYG{p}{]}\PYG{o}{/}\PYG{p}{(}\PYG{l+m+mi}{1000}\PYG{o}{/}\PYG{l+m+mi}{2}\PYG{p}{)}\PYG{p}{,}\PYG{+w}{ }\PYG{p}{[}\PYG{l+m+mi}{0}\PYG{+w}{ }\PYG{l+m+mi}{0}\PYG{+w}{ }\PYG{l+m+mi}{1}\PYG{+w}{ }\PYG{l+m+mi}{1}\PYG{+w}{ }\PYG{l+m+mi}{0}\PYG{+w}{ }\PYG{l+m+mi}{0}\PYG{p}{]}\PYG{p}{,}\PYG{+w}{ }\PYG{p}{[}\PYG{l+m+mi}{1}\PYG{+w}{ }\PYG{l+m+mi}{1}\PYG{+w}{ }\PYG{l+m+mi}{1}\PYG{p}{]}\PYG{+w}{ }\PYG{p}{)}\PYG{p}{;}\PYG{+w}{ }\PYG{c}{\PYGZpc{}creates FIR Band Pass filter of order \PYGZdq{}100\PYGZdq{} , around peak frequency of 19 Hz and assigned it to Beta model of bossdevice}
\end{sphinxVerbatim}


\bigskip\hrule\bigskip



\section{arm}
\label{\detokenize{4_api_documentation:arm}}

\subsection{Description}
\label{\detokenize{4_api_documentation:id41}}
\sphinxAtStartPar
Arming bossdevice would allow bossdevice to actively search for the Brain states already set into it and generate the TTL output sequence configured using “configure\_time\_port\_marker” upon every instance of detection.


\subsection{Syntax}
\label{\detokenize{4_api_documentation:id42}}
\begin{sphinxVerbatim}[commandchars=\\\{\}]
\PYG{n}{obj}\PYG{p}{.}\PYG{n}{armed}\PYG{p}{;}
\end{sphinxVerbatim}


\subsection{Example}
\label{\detokenize{4_api_documentation:id43}}
\begin{sphinxVerbatim}[commandchars=\\\{\}]
\PYG{n}{bd}\PYG{p}{.}\PYG{n}{armed}\PYG{p}{;}\PYG{+w}{ }\PYG{c}{\PYGZpc{} will set bossdevice to detect the specified brain states and TTL output sequence}
\end{sphinxVerbatim}


\bigskip\hrule\bigskip



\section{disarm}
\label{\detokenize{4_api_documentation:disarm}}

\subsection{Description}
\label{\detokenize{4_api_documentation:id44}}
\sphinxAtStartPar
Disarming bossdevice would allow bossdevice to stop search for the Brain states already by now “Armed’ into it and immediately stops the generation of the planned TTL output sequence configured using “configure\_time\_port\_marker” .


\subsection{Syntax}
\label{\detokenize{4_api_documentation:id45}}
\begin{sphinxVerbatim}[commandchars=\\\{\}]
\PYG{n}{obj}\PYG{p}{.}\PYG{n}{disarm}\PYG{p}{;}
\end{sphinxVerbatim}


\subsection{Example}
\label{\detokenize{4_api_documentation:id46}}
\begin{sphinxVerbatim}[commandchars=\\\{\}]
\PYG{n}{bd}\PYG{p}{.}\PYG{n}{disarm}\PYG{p}{;}\PYG{+w}{ }\PYG{c}{\PYGZpc{} will stop bossdevice to detect the specified brain states and stop ongoing TTL output sequence}
\end{sphinxVerbatim}


\bigskip\hrule\bigskip



\section{sample\_and\_hold\_period}
\label{\detokenize{4_api_documentation:sample-and-hold-period}}

\subsection{Description}
\label{\detokenize{4_api_documentation:id47}}
\sphinxAtStartPar
Upon an artifactual event such as pulse artifact etc., the data gets distorted and is accumulated with a lot of noise, in order to hold the samples for a specified period of time, sample and hold period method can be helpful.


\subsection{Syntax}
\label{\detokenize{4_api_documentation:id48}}
\begin{sphinxVerbatim}[commandchars=\\\{\}]
\PYG{n}{obj}\PYG{p}{.}\PYG{n}{sample\PYGZus{}and\PYGZus{}hold\PYGZus{}period}\PYG{+w}{ }\PYG{p}{=}\PYG{+w}{ }\PYG{p}{[}\PYG{n}{no\PYGZus{}of\PYGZus{}sample\PYGZus{}to\PYGZus{}hold}\PYG{p}{]}\PYG{p}{;}\PYG{+w}{ }\PYG{c}{\PYGZpc{} no\PYGZus{}of\PYGZus{}sample\PYGZus{}to\PYGZus{}hold are calculated at the rate of 5KHz}
\end{sphinxVerbatim}


\subsection{Example}
\label{\detokenize{4_api_documentation:id49}}
\begin{sphinxVerbatim}[commandchars=\\\{\}]
\PYG{n}{bd}\PYG{p}{.}\PYG{n}{sample\PYGZus{}and\PYGZus{}hold\PYGZus{}period}\PYG{+w}{ }\PYG{p}{=}\PYG{+w}{ }\PYG{l+m+mi}{500}\PYG{p}{;}\PYG{+w}{ }\PYG{c}{\PYGZpc{} hold the samples for 100 ms}
\end{sphinxVerbatim}


\bigskip\hrule\bigskip



\section{min\_inter\_pulse\_interval}
\label{\detokenize{4_api_documentation:min-inter-pulse-interval}}

\subsection{Description}
\label{\detokenize{4_api_documentation:id50}}
\sphinxAtStartPar
Minimum inter pulse interval allows you to define the time period for which the bossdevice should wait before generating another TTL output.


\subsection{Syntax}
\label{\detokenize{4_api_documentation:id51}}
\begin{sphinxVerbatim}[commandchars=\\\{\}]
\PYG{n}{obj}\PYG{p}{.}\PYG{n}{min\PYGZus{}inter\PYGZus{}pulse\PYGZus{}interval}\PYG{+w}{ }\PYG{p}{=}\PYG{+w}{ }\PYG{p}{[}\PYG{n}{time\PYGZus{}in\PYGZus{}secodns}\PYG{p}{]}\PYG{p}{;}\PYG{+w}{ }\PYG{c}{\PYGZpc{} Seconds to wait before generating another TTL output}
\end{sphinxVerbatim}


\subsection{Example}
\label{\detokenize{4_api_documentation:id52}}
\begin{sphinxVerbatim}[commandchars=\\\{\}]
\PYG{n}{obj}\PYG{p}{.}\PYG{n}{min\PYGZus{}inter\PYGZus{}pulse\PYGZus{}interval}\PYG{+w}{ }\PYG{p}{=}\PYG{+w}{ }\PYG{l+m+mi}{2}\PYG{p}{;}\PYG{+w}{ }\PYG{c}{\PYGZpc{} bossdevivce will wait for 2 seconds before generating any TTL output}
\end{sphinxVerbatim}


\bigskip\hrule\bigskip



\chapter{Demo Scripts}
\label{\detokenize{5_demo_scripts:demo-scripts}}\label{\detokenize{5_demo_scripts::doc}}
\begin{sphinxadmonition}{important}{Important:}
\sphinxAtStartPar
Please note that some of the features in the demo scripts may not be work proprely due to a transition from Matlab 2017b to Matlab 2021a. If you identify any difficulty in the respect, please contact the support team of sync2brain.
\end{sphinxadmonition}


\section{Open Loop Jittered TTL Output}
\label{\detokenize{5_demo_scripts:open-loop-jittered-ttl-output}}
\sphinxAtStartPar
This demo script uses bossdevice RESEARCH and 2 different approaches to generate jittered open loop TTL output.

\sphinxAtStartPar
Resources: 1) bossdevice Switched On, 2) bossdevice Open Source MATLAB API

\begin{sphinxVerbatim}[commandchars=\\\{\}]
\PYG{n}{NumberOfTrials}\PYG{p}{=}\PYG{l+m+mi}{10}\PYG{p}{;}
\PYG{+w}{ }\PYG{n}{ITI}\PYG{p}{=}\PYG{p}{[}\PYG{l+m+mi}{4}\PYG{+w}{ }\PYG{l+m+mi}{6}\PYG{p}{]}\PYG{p}{;}\PYG{+w}{ }\PYG{c}{\PYGZpc{}ITI is seconds \PYGZhy{} a random number between these two values}
\PYG{+w}{ }\PYG{c}{\PYGZpc{}\PYGZpc{} Initializing bossdevice RESEARCH API}
\PYG{+w}{ }\PYG{n}{bd}\PYG{p}{=}\PYG{n}{bossdevice}\PYG{p}{;}
\PYG{+w}{ }\PYG{c}{\PYGZpc{}\PYGZpc{} Approach 1 \PYGZhy{} For Loop Based Open Loop TTL output generation}
\PYG{+w}{ }\PYG{n}{bd}\PYG{p}{.}\PYG{n}{configure\PYGZus{}time\PYGZus{}port\PYGZus{}marker}\PYG{p}{(}\PYG{p}{[}\PYG{l+m+mi}{0}\PYG{+w}{ }\PYG{l+m+mi}{1}\PYG{+w}{ }\PYG{l+m+mi}{1}\PYG{p}{]}\PYG{p}{)}\PYG{p}{;}\PYG{+w}{ }\PYG{c}{\PYGZpc{}Configuring TTL output Sequence in [Time Port Marker] format}
\PYG{+w}{ }\PYG{k}{for}\PYG{+w}{ }\PYG{n}{TrialNumber}\PYG{p}{=}\PYG{l+m+mi}{1}\PYG{p}{:}\PYG{n}{NumberOfTrials}
\PYG{+w}{     }\PYG{n}{bd}\PYG{p}{.}\PYG{n}{manualPulse}
\PYG{+w}{     }\PYG{n+nb}{disp}\PYG{p}{(}\PYG{p}{[}\PYG{l+s}{\PYGZsq{}}\PYG{l+s}{Generated Trial \PYGZsh{}\PYGZsq{}}\PYG{+w}{ }\PYG{n+nb}{num2str}\PYG{p}{(}\PYG{n}{TrialNumber}\PYG{p}{)}\PYG{p}{]}\PYG{p}{)}
\PYG{+w}{     }\PYG{n}{min\PYGZus{}inter\PYGZus{}pulse\PYGZus{}interval}\PYG{p}{=}\PYG{+w}{ }\PYG{n}{ITI}\PYG{p}{(}\PYG{l+m+mi}{1}\PYG{p}{)}\PYG{o}{+}\PYG{+w}{ }\PYG{p}{(}\PYG{n}{ITI}\PYG{p}{(}\PYG{l+m+mi}{2}\PYG{p}{)}\PYG{o}{\PYGZhy{}}\PYG{n}{ITI}\PYG{p}{(}\PYG{l+m+mi}{1}\PYG{p}{)}\PYG{p}{)}\PYG{o}{.*}\PYG{n+nb}{rand}\PYG{p}{(}\PYG{l+m+mi}{1}\PYG{p}{,}\PYG{l+m+mi}{1}\PYG{p}{)}\PYG{p}{;}\PYG{+w}{ }\PYG{c}{\PYGZpc{}Assigning New Random ITI for this Trial to the bossdevice RESEARCH}
\PYG{+w}{     }\PYG{n+nb}{pause}\PYG{p}{(}\PYG{n}{min\PYGZus{}inter\PYGZus{}pulse\PYGZus{}interval}\PYG{p}{)}\PYG{+w}{ }\PYG{c}{\PYGZpc{}Wait for next trial start}
\PYG{+w}{ }\PYG{k}{end}
\PYG{+w}{ }\PYG{n+nb}{disp}\PYG{p}{(}\PYG{l+s}{\PYGZsq{}}\PYG{l+s}{Trials Completed\PYGZsq{}}\PYG{p}{)}
\PYG{+w}{ }\PYG{c}{\PYGZpc{}\PYGZpc{} Approach 2 \PYGZhy{} bossdevice RESEARCH Sequence Generator Based Open Loop TTL output}
\PYG{+w}{ }\PYG{n}{time\PYGZus{}port\PYGZus{}marker\PYGZus{}sequence}\PYG{p}{=}\PYG{p}{[}\PYG{p}{]}\PYG{p}{;}
\PYG{+w}{ }\PYG{n}{Time}\PYG{p}{=}\PYG{l+m+mi}{0}\PYG{p}{;}
\PYG{+w}{ }\PYG{n}{time\PYGZus{}port\PYGZus{}marker\PYGZus{}sequence}\PYG{p}{(}\PYG{n}{NumberOfTrials}\PYG{p}{,}\PYG{l+m+mi}{3}\PYG{p}{)}\PYG{p}{=}\PYG{l+m+mi}{0}\PYG{p}{;}\PYG{+w}{ }\PYG{c}{\PYGZpc{}Pre filling the array}
\PYG{+w}{ }\PYG{k}{for}\PYG{+w}{ }\PYG{n}{TrialNumber}\PYG{p}{=}\PYG{l+m+mi}{1}\PYG{p}{:}\PYG{n}{NumberOfTrials}
\PYG{+w}{     }\PYG{n}{Time}\PYG{p}{=}\PYG{n}{Time}\PYG{o}{+}\PYG{n}{ITI}\PYG{p}{(}\PYG{l+m+mi}{1}\PYG{p}{)}\PYG{o}{+}\PYG{+w}{ }\PYG{p}{(}\PYG{n}{ITI}\PYG{p}{(}\PYG{l+m+mi}{2}\PYG{p}{)}\PYG{o}{\PYGZhy{}}\PYG{n}{ITI}\PYG{p}{(}\PYG{l+m+mi}{1}\PYG{p}{)}\PYG{p}{)}\PYG{o}{.*}\PYG{n+nb}{rand}\PYG{p}{(}\PYG{l+m+mi}{1}\PYG{p}{,}\PYG{l+m+mi}{1}\PYG{p}{)}\PYG{p}{;}\PYG{+w}{ }\PYG{c}{\PYGZpc{}Generating Sequence of Jittered ITIs for all Trials}
\PYG{+w}{     }\PYG{n}{Port}\PYG{p}{=}\PYG{l+m+mi}{1}\PYG{p}{;}\PYG{+w}{ }\PYG{c}{\PYGZpc{}In order to generatre TTL output always on first port , use 2 for 2nd port and 3 for third port}
\PYG{+w}{     }\PYG{n}{Marker}\PYG{p}{=}\PYG{n}{TrialNumber}\PYG{p}{;}
\PYG{+w}{     }\PYG{n}{time\PYGZus{}port\PYGZus{}marker\PYGZus{}sequence}\PYG{p}{(}\PYG{n}{TrialNumber}\PYG{p}{,}\PYG{p}{:}\PYG{p}{)}\PYG{p}{=}\PYG{p}{[}\PYG{n}{Time}\PYG{+w}{ }\PYG{n}{Port}\PYG{+w}{ }\PYG{n}{Marker}\PYG{p}{]}\PYG{p}{;}
\PYG{+w}{ }\PYG{k}{end}
\PYG{+w}{ }\PYG{n}{bd}\PYG{p}{.}\PYG{n}{configure\PYGZus{}time\PYGZus{}port\PYGZus{}marker}\PYG{p}{(}\PYG{n}{time\PYGZus{}port\PYGZus{}marker\PYGZus{}sequence}\PYG{p}{)}\PYG{p}{;}\PYG{+w}{ }\PYG{c}{\PYGZpc{}Assigning Pregenerated sequence to the bossdevice RESEARCH}
\PYG{+w}{ }\PYG{n}{bd}\PYG{p}{.}\PYG{n}{manualPulse}\PYG{+w}{ }\PYG{c}{\PYGZpc{} Generating the TTL output of the bossdevice RESEARCH to start sequence TTL Output Generation}
\PYG{+w}{ }\PYG{n+nb}{disp}\PYG{p}{(}\PYG{l+s}{\PYGZsq{}}\PYG{l+s}{TTL output Sequence Started by bossdevice RESEARCH\PYGZsq{}}\PYG{p}{)}
\end{sphinxVerbatim}


\bigskip\hrule\bigskip



\section{mu\sphinxhyphen{}Rhythm Phase Locked TTL Output}
\label{\detokenize{5_demo_scripts:mu-rhythm-phase-locked-ttl-output}}
\sphinxAtStartPar
This demo script uses bossdevice to deliver mu Rising or Falling Flank Phase locked TTL outputs
Resources: 1) bossdevice Switched On
2) bossdevice Open Source MATLAB API
3) Biosignal Amplifier streaming atleast 5 EEG Channels

\begin{sphinxVerbatim}[commandchars=\\\{\}]
\PYG{n}{no\PYGZus{}of\PYGZus{}trials}\PYG{p}{=}\PYG{l+m+mi}{10}\PYG{p}{;}
\PYG{+w}{ }\PYG{n}{minimium\PYGZus{}inter\PYGZus{}pulse\PYGZus{}interval}\PYG{p}{=}\PYG{l+m+mi}{5}\PYG{p}{;}\PYG{+w}{ }\PYG{c}{\PYGZpc{}s}
\PYG{+w}{ }\PYG{n}{phase}\PYG{p}{=}\PYG{p}{[}\PYG{o}{+}\PYG{n+nb}{pi}\PYG{o}{/}\PYG{l+m+mi}{2}\PYG{+w}{ }\PYG{o}{\PYGZhy{}}\PYG{n+nb}{pi}\PYG{o}{/}\PYG{l+m+mi}{2}\PYG{p}{]}\PYG{p}{;}\PYG{+w}{ }\PYG{c}{\PYGZpc{}[falling\PYGZus{}flank rising\PYGZus{}flank]}
\PYG{+w}{ }\PYG{n}{phase\PYGZus{}tolerance}\PYG{p}{=}\PYG{n+nb}{pi}\PYG{o}{/}\PYG{l+m+mi}{40}\PYG{p}{;}
\PYG{+w}{ }\PYG{n}{individual\PYGZus{}peak\PYGZus{}frequency}\PYG{p}{=}\PYG{l+m+mi}{11}\PYG{p}{;}\PYG{+w}{ }\PYG{c}{\PYGZpc{} Hz}
\PYG{+w}{ }\PYG{n}{bandpassfilter\PYGZus{}order}\PYG{p}{=}\PYG{+w}{ }\PYG{l+m+mi}{75}\PYG{p}{;}
\PYG{+w}{ }\PYG{n}{eeg\PYGZus{}channels}\PYG{p}{=}\PYG{l+m+mi}{5}\PYG{p}{;}\PYG{+w}{ }\PYG{c}{\PYGZpc{}Assigning Number of channels as equivalent to Num of Channels streamed by Biosignal Processor}
\PYG{+w}{ }\PYG{n}{spatial\PYGZus{}filter\PYGZus{}weights}\PYG{p}{=}\PYG{p}{[}\PYG{l+m+mi}{1}\PYG{+w}{ }\PYG{o}{\PYGZhy{}}\PYG{l+m+mf}{0.25}\PYG{+w}{ }\PYG{o}{\PYGZhy{}}\PYG{l+m+mf}{0.25}\PYG{+w}{ }\PYG{o}{\PYGZhy{}}\PYG{l+m+mf}{0.25}\PYG{+w}{ }\PYG{o}{\PYGZhy{}}\PYG{l+m+mf}{0.25}\PYG{p}{]}\PYG{o}{\PYGZsq{}}\PYG{p}{;}\PYG{+w}{ }\PYG{c}{\PYGZpc{}Column Vector of Spatial Filter Indexed wrt corrosponding Channels}

\PYG{+w}{ }\PYG{n+nb}{time}\PYG{p}{=}\PYG{l+m+mi}{0}\PYG{p}{;}
\PYG{+w}{ }\PYG{n}{plasticity\PYGZus{}protocol\PYGZus{}sequence}\PYG{p}{=}\PYG{p}{[}\PYG{p}{]}\PYG{p}{;}

\PYG{+w}{ }\PYG{c}{\PYGZpc{}\PYGZpc{} Initializing bossdevice RESEARCH API}
\PYG{+w}{ }\PYG{n}{bd}\PYG{p}{=}\PYG{n}{bossdevice}\PYG{p}{;}
\PYG{+w}{ }\PYG{n}{bd}\PYG{p}{.}\PYG{n}{sample\PYGZus{}and\PYGZus{}hold\PYGZus{}period}\PYG{p}{=}\PYG{l+m+mi}{0}\PYG{p}{;}
\PYG{+w}{ }\PYG{n}{bd}\PYG{p}{.}\PYG{n}{calibration\PYGZus{}mode}\PYG{+w}{ }\PYG{p}{=}\PYG{+w}{ }\PYG{l+s}{\PYGZsq{}}\PYG{l+s}{no\PYGZsq{}}\PYG{p}{;}
\PYG{+w}{ }\PYG{n}{bd}\PYG{p}{.}\PYG{n}{armed}\PYG{+w}{ }\PYG{p}{=}\PYG{+w}{ }\PYG{l+s}{\PYGZsq{}}\PYG{l+s}{no\PYGZsq{}}\PYG{p}{;}
\PYG{+w}{ }\PYG{n}{bd}\PYG{p}{.}\PYG{n}{sample\PYGZus{}and\PYGZus{}hold\PYGZus{}period}\PYG{p}{=}\PYG{l+m+mi}{0}\PYG{p}{;}
\PYG{+w}{ }\PYG{n}{bd}\PYG{p}{.}\PYG{n}{theta}\PYG{p}{.}\PYG{n}{ignore}\PYG{p}{;}\PYG{+w}{ }\PYG{n+nb}{pause}\PYG{p}{(}\PYG{l+m+mf}{0.1}\PYG{p}{)}
\PYG{+w}{ }\PYG{n}{bd}\PYG{p}{.}\PYG{n}{beta}\PYG{p}{.}\PYG{n}{ignore}\PYG{p}{;}\PYG{+w}{ }\PYG{n+nb}{pause}\PYG{p}{(}\PYG{l+m+mf}{0.1}\PYG{p}{)}
\PYG{+w}{ }\PYG{n}{bd}\PYG{p}{.}\PYG{n}{alpha}\PYG{p}{.}\PYG{n}{ignore}\PYG{p}{;}\PYG{+w}{ }\PYG{n+nb}{pause}\PYG{p}{(}\PYG{l+m+mf}{0.1}\PYG{p}{)}
\PYG{+w}{ }\PYG{n}{bd}\PYG{p}{.}\PYG{n}{eeg\PYGZus{}channels}\PYG{p}{=}\PYG{n}{eeg\PYGZus{}channels}\PYG{p}{;}

\PYG{+w}{ }\PYG{c}{\PYGZpc{}\PYGZpc{} Preparing an Individual Peak Frequency based Band Pass Filter for mu Alpha}
\PYG{+w}{ }\PYG{n}{bpf\PYGZus{}fir\PYGZus{}coeffs}\PYG{+w}{ }\PYG{p}{=}\PYG{+w}{ }\PYG{n}{firls}\PYG{p}{(}\PYG{n}{bandpassfilter\PYGZus{}order}\PYG{p}{,}\PYG{+w}{ }\PYG{p}{[}\PYG{l+m+mi}{0}\PYG{+w}{ }\PYG{p}{(}\PYG{n}{individual\PYGZus{}peak\PYGZus{}frequency}\PYG{+w}{ }\PYG{o}{+}\PYG{+w}{ }\PYG{p}{[}\PYG{o}{\PYGZhy{}}\PYG{l+m+mi}{5}\PYG{+w}{ }\PYG{o}{\PYGZhy{}}\PYG{l+m+mi}{2}\PYG{+w}{ }\PYG{o}{+}\PYG{l+m+mi}{2}\PYG{+w}{ }\PYG{o}{+}\PYG{l+m+mi}{5}\PYG{p}{]}\PYG{p}{)}\PYG{+w}{ }\PYG{p}{(}\PYG{l+m+mi}{500}\PYG{o}{/}\PYG{l+m+mi}{2}\PYG{p}{)}\PYG{p}{]}\PYG{o}{/}\PYG{p}{(}\PYG{l+m+mi}{500}\PYG{o}{/}\PYG{l+m+mi}{2}\PYG{p}{)}\PYG{p}{,}\PYG{+w}{ }\PYG{p}{[}\PYG{l+m+mi}{0}\PYG{+w}{ }\PYG{l+m+mi}{0}\PYG{+w}{ }\PYG{l+m+mi}{1}\PYG{+w}{ }\PYG{l+m+mi}{1}\PYG{+w}{ }\PYG{l+m+mi}{0}\PYG{+w}{ }\PYG{l+m+mi}{0}\PYG{p}{]}\PYG{p}{,}\PYG{+w}{ }\PYG{p}{[}\PYG{l+m+mi}{1}\PYG{+w}{ }\PYG{l+m+mi}{1}\PYG{+w}{ }\PYG{l+m+mi}{1}\PYG{p}{]}\PYG{+w}{ }\PYG{p}{)}\PYG{p}{;}

\PYG{+w}{ }\PYG{c}{\PYGZpc{}\PYGZpc{} Setting Filters on bossdevice RESEARCH}
\PYG{+w}{ }\PYG{n}{bd}\PYG{p}{.}\PYG{n}{spatial\PYGZus{}filter\PYGZus{}weights}\PYG{p}{=}\PYG{n}{spatial\PYGZus{}filter\PYGZus{}weights}\PYG{p}{;}
\PYG{+w}{ }\PYG{n}{bd}\PYG{p}{.}\PYG{n}{alpha}\PYG{p}{.}\PYG{n}{bpf\PYGZus{}fir\PYGZus{}coeffs}\PYG{+w}{ }\PYG{p}{=}\PYG{+w}{ }\PYG{n}{bpf\PYGZus{}fir\PYGZus{}coeffs}\PYG{p}{;}

\PYG{+w}{ }\PYG{c}{\PYGZpc{}\PYGZpc{} Controlling bossdevice RESEARCH for mu Alpha Phase Locked TTL Output}
\PYG{+w}{ }\PYG{n}{condition\PYGZus{}index}\PYG{p}{=}\PYG{l+m+mi}{0}\PYG{p}{;}
\PYG{+w}{ }\PYG{k}{while}\PYG{+w}{ }\PYG{p}{(}\PYG{n}{condition\PYGZus{}index}\PYG{+w}{ }\PYG{o}{\PYGZlt{}=}\PYG{+w}{ }\PYG{n}{no\PYGZus{}of\PYGZus{}trials}\PYG{p}{)}
\PYG{+w}{     }\PYG{k}{if}\PYG{p}{(}\PYG{n+nb}{strcmp}\PYG{p}{(}\PYG{n}{bb}\PYG{p}{.}\PYG{n}{armed}\PYG{p}{,}\PYG{+w}{ }\PYG{l+s}{\PYGZsq{}}\PYG{l+s}{no\PYGZsq{}}\PYG{p}{)}\PYG{p}{)}
\PYG{+w}{         }\PYG{n}{bb}\PYG{p}{.}\PYG{n}{pulses\PYGZus{}remaining}\PYG{+w}{ }\PYG{p}{=}\PYG{+w}{ }\PYG{l+m+mi}{1}\PYG{p}{;}
\PYG{+w}{         }\PYG{n}{bb}\PYG{p}{.}\PYG{n}{alpha}\PYG{p}{.}\PYG{n}{phase\PYGZus{}target}\PYG{p}{(}\PYG{l+m+mi}{1}\PYG{p}{)}\PYG{+w}{ }\PYG{p}{=}\PYG{+w}{ }\PYG{n}{phase}\PYG{p}{(}\PYG{n+nb}{randi}\PYG{p}{(}\PYG{l+m+mi}{1}\PYG{p}{:}\PYG{n+nb}{numel}\PYG{p}{(}\PYG{n}{phase}\PYG{p}{)}\PYG{p}{,}\PYG{+w}{ }\PYG{l+m+mi}{1}\PYG{p}{)}\PYG{p}{)}\PYG{p}{;}
\PYG{+w}{         }\PYG{n}{bb}\PYG{p}{.}\PYG{n}{alpha}\PYG{p}{.}\PYG{n}{phase\PYGZus{}plusminus}\PYG{p}{(}\PYG{l+m+mi}{1}\PYG{p}{)}\PYG{+w}{ }\PYG{p}{=}\PYG{+w}{ }\PYG{n}{phase\PYGZus{}tolerance}\PYG{p}{;}
\PYG{+w}{         }\PYG{n}{bb}\PYG{p}{.}\PYG{n}{configure\PYGZus{}time\PYGZus{}port\PYGZus{}marker}\PYG{p}{(}\PYG{p}{(}\PYG{p}{[}\PYG{l+m+mi}{0}\PYG{p}{,}\PYG{+w}{ }\PYG{l+m+mi}{1}\PYG{p}{,}\PYG{+w}{ }\PYG{l+m+mi}{0}\PYG{p}{]}\PYG{p}{)}\PYG{p}{)}
\PYG{+w}{         }\PYG{n}{bb}\PYG{p}{.}\PYG{n}{min\PYGZus{}inter\PYGZus{}pulse\PYGZus{}interval}\PYG{+w}{ }\PYG{p}{=}\PYG{+w}{ }\PYG{n}{minimium\PYGZus{}inter\PYGZus{}pulse\PYGZus{}interval}\PYG{p}{;}
\PYG{+w}{         }\PYG{n+nb}{pause}\PYG{p}{(}\PYG{l+m+mf}{0.1}\PYG{p}{)}
\PYG{+w}{         }\PYG{n}{bb}\PYG{p}{.}\PYG{n}{arm}\PYG{p}{;}
\PYG{+w}{     }\PYG{k}{end}
\PYG{+w}{     }\PYG{c}{\PYGZpc{} trigger has been executed, move to the next condition}
\PYG{+w}{     }\PYG{k}{if}\PYG{p}{(}\PYG{n}{bb}\PYG{p}{.}\PYG{n}{pulses\PYGZus{}remaining}\PYG{+w}{ }\PYG{o}{==}\PYG{+w}{ }\PYG{l+m+mi}{0}\PYG{p}{)}
\PYG{+w}{         }\PYG{n}{condition\PYGZus{}index}\PYG{+w}{ }\PYG{p}{=}\PYG{+w}{ }\PYG{n}{condition\PYGZus{}index}\PYG{+w}{ }\PYG{o}{+}\PYG{+w}{ }\PYG{l+m+mi}{1}\PYG{p}{;}
\PYG{+w}{         }\PYG{n}{bb}\PYG{p}{.}\PYG{n}{disarm}\PYG{p}{;}
\PYG{+w}{         }\PYG{n+nb}{disp}\PYG{+w}{ }\PYG{p}{(}\PYG{p}{[}\PYG{l+s}{\PYGZsq{}}\PYG{l+s}{Triggered around \PYGZsq{}}\PYG{+w}{ }\PYG{p}{(}\PYG{n+nb}{num2str}\PYG{p}{(}\PYG{n+nb}{rad2deg}\PYG{p}{(}\PYG{n}{bb}\PYG{p}{.}\PYG{n}{alpha}\PYG{p}{.}\PYG{n}{phase\PYGZus{}target}\PYG{p}{(}\PYG{l+m+mi}{1}\PYG{p}{)}\PYG{p}{)}\PYG{p}{)}\PYG{p}{)}\PYG{+w}{ }\PYG{l+s}{\PYGZsq{}}\PYG{l+s}{ degrees Phase angle.\PYGZsq{}}\PYG{p}{]}\PYG{p}{)}
\PYG{+w}{         }\PYG{n+nb}{pause}\PYG{p}{(}\PYG{n}{minimium\PYGZus{}inter\PYGZus{}pulse\PYGZus{}interval}\PYG{p}{)}
\PYG{+w}{     }\PYG{k}{end}
\PYG{+w}{     }\PYG{n+nb}{pause}\PYG{p}{(}\PYG{l+m+mf}{0.01}\PYG{p}{)}\PYG{p}{;}
\PYG{+w}{ }\PYG{k}{end}

\PYG{+w}{ }\PYG{c}{\PYGZpc{}\PYGZpc{} End}
\PYG{+w}{ }\PYG{n+nb}{disp}\PYG{+w}{ }\PYG{p}{(}\PYG{l+s}{\PYGZsq{}}\PYG{l+s}{Protocol has been completed\PYGZsq{}}\PYG{p}{)}\PYG{p}{;}
\end{sphinxVerbatim}


\bigskip\hrule\bigskip



\section{Phase Locked Plasticity Protocol}
\label{\detokenize{5_demo_scripts:phase-locked-plasticity-protocol}}
\sphinxAtStartPar
This demo script uses bossdevice RESEARCH and 2 different approaches to generate jittered open loopTTL output

\sphinxAtStartPar
Resources: 1) bossdevice Switched On, 2) bossdevice Open Source MATLAB API

\begin{sphinxVerbatim}[commandchars=\\\{\}]
\PYG{n}{no\PYGZus{}of\PYGZus{}trials}\PYG{p}{=}\PYG{l+m+mi}{25}\PYG{p}{;}
\PYG{+w}{ }\PYG{n}{no\PYGZus{}of\PYGZus{}pulses}\PYG{p}{=}\PYG{l+m+mi}{100}\PYG{p}{;}
\PYG{+w}{ }\PYG{n}{pulse\PYGZus{}frequency}\PYG{p}{=}\PYG{l+m+mi}{100}\PYG{p}{;}\PYG{+w}{ }\PYG{c}{\PYGZpc{}Hz}
\PYG{+w}{ }\PYG{n}{minimium\PYGZus{}inter\PYGZus{}pulse\PYGZus{}interval}\PYG{p}{=}\PYG{l+m+mi}{5}\PYG{p}{;}\PYG{+w}{ }\PYG{c}{\PYGZpc{}s}
\PYG{+w}{ }\PYG{n}{phase}\PYG{p}{=}\PYG{l+m+mi}{0}\PYG{p}{;}\PYG{+w}{ }\PYG{c}{\PYGZpc{}peak}
\PYG{+w}{ }\PYG{n}{phase\PYGZus{}tolerance}\PYG{p}{=}\PYG{n+nb}{pi}\PYG{o}{/}\PYG{l+m+mi}{40}\PYG{p}{;}
\PYG{+w}{ }\PYG{n}{individual\PYGZus{}peak\PYGZus{}frequency}\PYG{p}{=}\PYG{l+m+mi}{11}\PYG{p}{;}\PYG{+w}{ }\PYG{c}{\PYGZpc{} Hz}
\PYG{+w}{ }\PYG{n}{bandpassfilter\PYGZus{}order}\PYG{p}{=}\PYG{+w}{ }\PYG{l+m+mi}{75}\PYG{p}{;}
\PYG{+w}{ }\PYG{n}{eeg\PYGZus{}channels}\PYG{p}{=}\PYG{l+m+mi}{5}\PYG{p}{;}\PYG{+w}{ }\PYG{c}{\PYGZpc{}Assigning Number of channels as equivalent to Num of Channels streamed by Biosignal Processor}
\PYG{+w}{ }\PYG{n}{spatial\PYGZus{}filter\PYGZus{}weights}\PYG{p}{=}\PYG{p}{[}\PYG{l+m+mi}{1}\PYG{+w}{ }\PYG{o}{\PYGZhy{}}\PYG{l+m+mf}{0.25}\PYG{+w}{ }\PYG{o}{\PYGZhy{}}\PYG{l+m+mf}{0.25}\PYG{+w}{ }\PYG{o}{\PYGZhy{}}\PYG{l+m+mf}{0.25}\PYG{+w}{ }\PYG{o}{\PYGZhy{}}\PYG{l+m+mf}{0.25}\PYG{p}{]}\PYG{o}{\PYGZsq{}}\PYG{p}{;}\PYG{+w}{ }\PYG{c}{\PYGZpc{}Column Vector of Spatial Filter Indexed wrt corrosponding Channels}

\PYG{+w}{ }\PYG{n+nb}{time}\PYG{p}{=}\PYG{l+m+mi}{0}\PYG{p}{;}
\PYG{+w}{ }\PYG{n}{plasticity\PYGZus{}protocol\PYGZus{}sequence}\PYG{p}{=}\PYG{p}{[}\PYG{p}{]}\PYG{p}{;}

\PYG{+w}{ }\PYG{c}{\PYGZpc{}\PYGZpc{} Initializing bossdevice RESEARCH API}
\PYG{+w}{ }\PYG{n}{bd}\PYG{p}{=}\PYG{n}{bossdevice}\PYG{p}{;}
\PYG{+w}{ }\PYG{n}{bd}\PYG{p}{.}\PYG{n}{sample\PYGZus{}and\PYGZus{}hold\PYGZus{}period}\PYG{p}{=}\PYG{l+m+mi}{0}\PYG{p}{;}
\PYG{+w}{ }\PYG{n}{bd}\PYG{p}{.}\PYG{n}{calibration\PYGZus{}mode}\PYG{+w}{ }\PYG{p}{=}\PYG{+w}{ }\PYG{l+s}{\PYGZsq{}}\PYG{l+s}{no\PYGZsq{}}\PYG{p}{;}
\PYG{+w}{ }\PYG{n}{bd}\PYG{p}{.}\PYG{n}{armed}\PYG{+w}{ }\PYG{p}{=}\PYG{+w}{ }\PYG{l+s}{\PYGZsq{}}\PYG{l+s}{no\PYGZsq{}}\PYG{p}{;}
\PYG{+w}{ }\PYG{n}{bd}\PYG{p}{.}\PYG{n}{sample\PYGZus{}and\PYGZus{}hold\PYGZus{}period}\PYG{p}{=}\PYG{l+m+mi}{0}\PYG{p}{;}
\PYG{+w}{ }\PYG{n}{bd}\PYG{p}{.}\PYG{n}{theta}\PYG{p}{.}\PYG{n}{ignore}\PYG{p}{;}\PYG{+w}{ }\PYG{n+nb}{pause}\PYG{p}{(}\PYG{l+m+mf}{0.1}\PYG{p}{)}
\PYG{+w}{ }\PYG{n}{bd}\PYG{p}{.}\PYG{n}{beta}\PYG{p}{.}\PYG{n}{ignore}\PYG{p}{;}\PYG{+w}{ }\PYG{n+nb}{pause}\PYG{p}{(}\PYG{l+m+mf}{0.1}\PYG{p}{)}
\PYG{+w}{ }\PYG{n}{bd}\PYG{p}{.}\PYG{n}{alpha}\PYG{p}{.}\PYG{n}{ignore}\PYG{p}{;}\PYG{+w}{ }\PYG{n+nb}{pause}\PYG{p}{(}\PYG{l+m+mf}{0.1}\PYG{p}{)}
\PYG{+w}{ }\PYG{n}{bd}\PYG{p}{.}\PYG{n}{eeg\PYGZus{}channels}\PYG{p}{=}\PYG{n}{eeg\PYGZus{}channels}\PYG{p}{;}

\PYG{+w}{ }\PYG{c}{\PYGZpc{}\PYGZpc{} Preparing a Plasticity Protocol Seqeuence for bossdevice RESEARCH}
\PYG{+w}{ }\PYG{n}{plasticity\PYGZus{}protocol\PYGZus{}sequence}\PYG{p}{(}\PYG{n}{no\PYGZus{}of\PYGZus{}pulses}\PYG{p}{,}\PYG{l+m+mi}{3}\PYG{p}{)}\PYG{p}{=}\PYG{l+m+mi}{0}\PYG{p}{;}\PYG{+w}{ }\PYG{c}{\PYGZpc{}Pre filling the array}
\PYG{+w}{ }\PYG{k}{for}\PYG{+w}{ }\PYG{n}{iPulse}\PYG{p}{=}\PYG{l+m+mi}{1}\PYG{p}{:}\PYG{n}{no\PYGZus{}of\PYGZus{}pulses}
\PYG{+w}{     }\PYG{n+nb}{time}\PYG{p}{=}\PYG{n+nb}{time}\PYG{o}{+}\PYG{l+m+mf}{0.01}\PYG{p}{;}
\PYG{+w}{     }\PYG{n}{port}\PYG{p}{=}\PYG{l+m+mi}{1}\PYG{p}{;}
\PYG{+w}{     }\PYG{n}{marker}\PYG{p}{=}\PYG{n}{iPulse}\PYG{p}{;}
\PYG{+w}{     }\PYG{n}{plasticity\PYGZus{}protocol\PYGZus{}sequence}\PYG{p}{(}\PYG{n}{iPulse}\PYG{p}{,}\PYG{p}{:}\PYG{p}{)}\PYG{p}{=}\PYG{p}{[}\PYG{n+nb}{time}\PYG{+w}{ }\PYG{n}{port}\PYG{+w}{ }\PYG{n}{marker}\PYG{p}{]}\PYG{p}{;}
\PYG{+w}{ }\PYG{k}{end}

\PYG{+w}{ }\PYG{c}{\PYGZpc{}\PYGZpc{} Preparing an Individual Peak Frequency based Band Pass Filter for mu Alpha}
\PYG{+w}{ }\PYG{n}{bpf\PYGZus{}fir\PYGZus{}coeffs}\PYG{+w}{ }\PYG{p}{=}\PYG{+w}{ }\PYG{n}{firls}\PYG{p}{(}\PYG{n}{bandpassfilter\PYGZus{}order}\PYG{p}{,}\PYG{+w}{ }\PYG{p}{[}\PYG{l+m+mi}{0}\PYG{+w}{ }\PYG{p}{(}\PYG{n}{individual\PYGZus{}peak\PYGZus{}frequency}\PYG{+w}{ }\PYG{o}{+}\PYG{+w}{ }\PYG{p}{[}\PYG{o}{\PYGZhy{}}\PYG{l+m+mi}{5}\PYG{+w}{ }\PYG{o}{\PYGZhy{}}\PYG{l+m+mi}{2}\PYG{+w}{ }\PYG{o}{+}\PYG{l+m+mi}{2}\PYG{+w}{ }\PYG{o}{+}\PYG{l+m+mi}{5}\PYG{p}{]}\PYG{p}{)}\PYG{+w}{ }\PYG{p}{(}\PYG{l+m+mi}{500}\PYG{o}{/}\PYG{l+m+mi}{2}\PYG{p}{)}\PYG{p}{]}\PYG{o}{/}\PYG{p}{(}\PYG{l+m+mi}{500}\PYG{o}{/}\PYG{l+m+mi}{2}\PYG{p}{)}\PYG{p}{,}\PYG{+w}{ }\PYG{p}{[}\PYG{l+m+mi}{0}\PYG{+w}{ }\PYG{l+m+mi}{0}\PYG{+w}{ }\PYG{l+m+mi}{1}\PYG{+w}{ }\PYG{l+m+mi}{1}\PYG{+w}{ }\PYG{l+m+mi}{0}\PYG{+w}{ }\PYG{l+m+mi}{0}\PYG{p}{]}\PYG{p}{,}\PYG{+w}{ }\PYG{p}{[}\PYG{l+m+mi}{1}\PYG{+w}{ }\PYG{l+m+mi}{1}\PYG{+w}{ }\PYG{l+m+mi}{1}\PYG{p}{]}\PYG{+w}{ }\PYG{p}{)}\PYG{p}{;}

\PYG{+w}{ }\PYG{c}{\PYGZpc{}\PYGZpc{} Setting Filters on bossdevice RESEARCH}
\PYG{+w}{ }\PYG{n}{bd}\PYG{p}{.}\PYG{n}{spatial\PYGZus{}filter\PYGZus{}weights}\PYG{p}{=}\PYG{n}{spatial\PYGZus{}filter\PYGZus{}weights}\PYG{p}{;}
\PYG{+w}{ }\PYG{n}{bd}\PYG{p}{.}\PYG{n}{alpha}\PYG{p}{.}\PYG{n}{bpf\PYGZus{}fir\PYGZus{}coeffs}\PYG{+w}{ }\PYG{p}{=}\PYG{+w}{ }\PYG{n}{bpf\PYGZus{}fir\PYGZus{}coeffs}\PYG{p}{;}



\PYG{+w}{ }\PYG{c}{\PYGZpc{}\PYGZpc{} For plasticitz, we have the same condition, multiple times, we can run everything on the device:}
\PYG{+w}{         }\PYG{n}{bd}\PYG{p}{.}\PYG{n}{pulses\PYGZus{}remaining}\PYG{+w}{ }\PYG{p}{=}\PYG{+w}{ }\PYG{l+m+mi}{100}\PYG{p}{;}
\PYG{+w}{         }\PYG{n}{bd}\PYG{p}{.}\PYG{n}{alpha}\PYG{p}{.}\PYG{n}{phase\PYGZus{}target}\PYG{p}{(}\PYG{l+m+mi}{1}\PYG{p}{)}\PYG{+w}{ }\PYG{p}{=}\PYG{+w}{ }\PYG{n}{phase}\PYG{p}{;}
\PYG{+w}{         }\PYG{n}{bd}\PYG{p}{.}\PYG{n}{alpha}\PYG{p}{.}\PYG{n}{phase\PYGZus{}plusminus}\PYG{p}{(}\PYG{l+m+mi}{1}\PYG{p}{)}\PYG{+w}{ }\PYG{p}{=}\PYG{+w}{ }\PYG{n}{phase\PYGZus{}tolerance}\PYG{p}{;}
\PYG{+w}{         }\PYG{n}{bd}\PYG{p}{.}\PYG{n}{configure\PYGZus{}time\PYGZus{}port\PYGZus{}marker}\PYG{p}{(}\PYG{n}{plasticity\PYGZus{}protocol\PYGZus{}sequence}\PYG{p}{)}
\PYG{+w}{         }\PYG{n}{bd}\PYG{p}{.}\PYG{n}{min\PYGZus{}inter\PYGZus{}pulse\PYGZus{}interval}\PYG{+w}{ }\PYG{p}{=}\PYG{+w}{ }\PYG{n}{minimium\PYGZus{}inter\PYGZus{}pulse\PYGZus{}interval}\PYG{p}{;}
\PYG{+w}{         }\PYG{n+nb}{pause}\PYG{p}{(}\PYG{l+m+mf}{0.1}\PYG{p}{)}
\PYG{+w}{         }\PYG{n}{bd}\PYG{p}{.}\PYG{n}{arm}\PYG{p}{;}

\PYG{+w}{         }\PYG{n+nb}{fprintf}\PYG{p}{(}\PYG{l+s}{\PYGZsq{}}\PYG{l+s}{\PYGZbs{}nSystem running, pulses remaining: \PYGZpc{}03i\PYGZsq{}}\PYG{p}{,}\PYG{+w}{ }\PYG{n}{bd}\PYG{p}{.}\PYG{n}{pulses\PYGZus{}remaining}\PYG{p}{)}
\PYG{+w}{         }\PYG{k}{while}\PYG{+w}{ }\PYG{p}{(}\PYG{n}{bd}\PYG{p}{.}\PYG{n}{pulses\PYGZus{}remaining}\PYG{+w}{ }\PYG{o}{\PYGZgt{}}\PYG{+w}{ }\PYG{l+m+mi}{0}\PYG{p}{)}
\PYG{+w}{             }\PYG{n+nb}{fprintf}\PYG{p}{(}\PYG{l+s}{\PYGZsq{}}\PYG{l+s}{\PYGZbs{}b\PYGZbs{}b\PYGZbs{}b\PYGZpc{}03i\PYGZsq{}}\PYG{p}{,}\PYG{+w}{ }\PYG{n}{bd}\PYG{p}{.}\PYG{n}{pulses\PYGZus{}remaining}\PYG{p}{)}\PYG{p}{;}
\PYG{+w}{             }\PYG{n+nb}{pause}\PYG{p}{(}\PYG{l+m+mf}{0.1}\PYG{p}{)}
\PYG{+w}{         }\PYG{k}{end}
\PYG{+w}{         }\PYG{n+nb}{fprintf}\PYG{p}{(}\PYG{l+s}{\PYGZsq{}}\PYG{l+s}{\PYGZbs{}b\PYGZbs{}b\PYGZbs{}bDone\PYGZbs{}n\PYGZsq{}}\PYG{p}{)}


\PYG{+w}{ }\PYG{c}{\PYGZpc{}\PYGZpc{} Controlling bossdevice RESEARCH for mu Alpha Phase Locked TTL Ouput \PYGZpc{} this could be for excitability, where we have interleaved different conditions}
\PYG{+w}{ }\PYG{n}{condition\PYGZus{}index}\PYG{p}{=}\PYG{l+m+mi}{0}\PYG{p}{;}
\PYG{+w}{ }\PYG{k}{while}\PYG{+w}{ }\PYG{p}{(}\PYG{n}{condition\PYGZus{}index}\PYG{+w}{ }\PYG{o}{\PYGZlt{}=}\PYG{+w}{ }\PYG{n}{no\PYGZus{}of\PYGZus{}trials}\PYG{p}{)}
\PYG{+w}{     }\PYG{k}{if}\PYG{p}{(}\PYG{n+nb}{strcmp}\PYG{p}{(}\PYG{n}{bd}\PYG{p}{.}\PYG{n}{armed}\PYG{p}{,}\PYG{+w}{ }\PYG{l+s}{\PYGZsq{}}\PYG{l+s}{no\PYGZsq{}}\PYG{p}{)}\PYG{p}{)}
\PYG{+w}{         }\PYG{n}{bd}\PYG{p}{.}\PYG{n}{pulses\PYGZus{}remaining}\PYG{+w}{ }\PYG{p}{=}\PYG{+w}{ }\PYG{l+m+mi}{1}\PYG{p}{;}
\PYG{+w}{         }\PYG{n}{bd}\PYG{p}{.}\PYG{n}{alpha}\PYG{p}{.}\PYG{n}{phase\PYGZus{}target}\PYG{p}{(}\PYG{l+m+mi}{1}\PYG{p}{)}\PYG{+w}{ }\PYG{p}{=}\PYG{+w}{ }\PYG{n}{phase}\PYG{p}{;}
\PYG{+w}{         }\PYG{n}{bd}\PYG{p}{.}\PYG{n}{alpha}\PYG{p}{.}\PYG{n}{phase\PYGZus{}plusminus}\PYG{p}{(}\PYG{l+m+mi}{1}\PYG{p}{)}\PYG{+w}{ }\PYG{p}{=}\PYG{+w}{ }\PYG{n}{phase\PYGZus{}tolerance}\PYG{p}{;}
\PYG{+w}{         }\PYG{n}{bd}\PYG{p}{.}\PYG{n}{configure\PYGZus{}time\PYGZus{}port\PYGZus{}marker}\PYG{p}{(}\PYG{n}{plasticity\PYGZus{}protocol\PYGZus{}sequence}\PYG{p}{)}
\PYG{+w}{         }\PYG{n}{bd}\PYG{p}{.}\PYG{n}{min\PYGZus{}inter\PYGZus{}pulse\PYGZus{}interval}\PYG{+w}{ }\PYG{p}{=}\PYG{+w}{ }\PYG{n}{minimium\PYGZus{}inter\PYGZus{}pulse\PYGZus{}interval}\PYG{p}{;}
\PYG{+w}{         }\PYG{n+nb}{pause}\PYG{p}{(}\PYG{l+m+mf}{0.1}\PYG{p}{)}
\PYG{+w}{         }\PYG{n}{bd}\PYG{p}{.}\PYG{n}{arm}\PYG{p}{;}
\PYG{+w}{     }\PYG{k}{end}
\PYG{+w}{     }\PYG{c}{\PYGZpc{} TTL output has been generated, move to the next condition}
\PYG{+w}{     }\PYG{k}{if}\PYG{p}{(}\PYG{n}{bd}\PYG{p}{.}\PYG{n}{pulses\PYGZus{}remaining}\PYG{+w}{ }\PYG{o}{==}\PYG{+w}{ }\PYG{l+m+mi}{0}\PYG{p}{)}
\PYG{+w}{         }\PYG{n}{condition\PYGZus{}index}\PYG{+w}{ }\PYG{p}{=}\PYG{+w}{ }\PYG{n}{condition\PYGZus{}index}\PYG{+w}{ }\PYG{o}{+}\PYG{+w}{ }\PYG{l+m+mi}{1}\PYG{p}{;}
\PYG{+w}{         }\PYG{n}{bd}\PYG{p}{.}\PYG{n}{disarm}\PYG{p}{;}
\PYG{+w}{         }\PYG{n}{disp}\PYG{+w}{ }\PYG{l+s}{TTL\PYGZus{}Output\PYGZus{}Generated!}
\PYG{+w}{         }\PYG{n+nb}{pause}\PYG{p}{(}\PYG{n}{minimium\PYGZus{}inter\PYGZus{}pulse\PYGZus{}interval}\PYG{p}{)}
\PYG{+w}{     }\PYG{k}{end}
\PYG{+w}{     }\PYG{n+nb}{pause}\PYG{p}{(}\PYG{l+m+mf}{0.01}\PYG{p}{)}\PYG{p}{;}
\PYG{+w}{ }\PYG{k}{end}

\PYG{+w}{ }\PYG{c}{\PYGZpc{}\PYGZpc{} End}
\PYG{+w}{ }\PYG{n+nb}{disp}\PYG{+w}{ }\PYG{p}{(}\PYG{l+s}{\PYGZsq{}}\PYG{l+s}{Plasticity Protocol has been completed\PYGZsq{}}\PYG{p}{)}\PYG{p}{;}
\end{sphinxVerbatim}


\bigskip\hrule\bigskip



\section{Real\sphinxhyphen{}Time Oscillation Amplitude Threshold Tracking}
\label{\detokenize{5_demo_scripts:real-time-oscillation-amplitude-threshold-tracking}}
\sphinxAtStartPar
This demo script uses bossdevice RESEARCH and 2 different approaches to generate jittered open loop TTL output

\sphinxAtStartPar
Resources: 1) bossdevice Switched On, 2) bossdevice Open Source MATLAB API

\begin{sphinxVerbatim}[commandchars=\\\{\}]
\PYG{n}{no\PYGZus{}of\PYGZus{}trials}\PYG{p}{=}\PYG{l+m+mi}{25}\PYG{p}{;}
\PYG{n}{minimium\PYGZus{}inter\PYGZus{}pulse\PYGZus{}interval}\PYG{p}{=}\PYG{l+m+mi}{4}\PYG{p}{;}\PYG{+w}{ }\PYG{c}{\PYGZpc{}s}
\PYG{n}{phase}\PYG{p}{=}\PYG{l+m+mi}{0}\PYG{p}{;}\PYG{+w}{ }\PYG{c}{\PYGZpc{}[positive]}
\PYG{n}{phase\PYGZus{}tolerance}\PYG{p}{=}\PYG{n+nb}{pi}\PYG{o}{/}\PYG{l+m+mi}{40}\PYG{p}{;}
\PYG{n}{amplitude\PYGZus{}threshold}\PYG{p}{=}\PYG{p}{[}\PYG{l+m+mi}{25}\PYG{+w}{ }\PYG{l+m+mi}{75}\PYG{p}{]}\PYG{p}{;}\PYG{+w}{ }\PYG{c}{\PYGZpc{}[min max] in percentile}
\PYG{n}{amplitude\PYGZus{}assignment\PYGZus{}period}\PYG{p}{=}\PYG{l+m+mi}{60}\PYG{o}{*}\PYG{l+m+mi}{2}\PYG{p}{;}\PYG{+w}{ }\PYG{c}{\PYGZpc{}s}
\PYG{n}{individual\PYGZus{}peak\PYGZus{}frequency}\PYG{p}{=}\PYG{l+m+mi}{11}\PYG{p}{;}\PYG{+w}{ }\PYG{c}{\PYGZpc{} Hz}
\PYG{n}{bandpassfilter\PYGZus{}order}\PYG{p}{=}\PYG{+w}{ }\PYG{l+m+mi}{75}\PYG{p}{;}
\PYG{n}{eeg\PYGZus{}channels}\PYG{p}{=}\PYG{l+m+mi}{5}\PYG{p}{;}\PYG{+w}{ }\PYG{c}{\PYGZpc{}Assigning Number of channels as equivalent to Num of Channels streamed by Biosignal Processor}
\PYG{n}{spatial\PYGZus{}filter\PYGZus{}weights}\PYG{p}{=}\PYG{p}{[}\PYG{l+m+mi}{1}\PYG{+w}{ }\PYG{o}{\PYGZhy{}}\PYG{l+m+mf}{0.25}\PYG{+w}{ }\PYG{o}{\PYGZhy{}}\PYG{l+m+mf}{0.25}\PYG{+w}{ }\PYG{o}{\PYGZhy{}}\PYG{l+m+mf}{0.25}\PYG{+w}{ }\PYG{o}{\PYGZhy{}}\PYG{l+m+mf}{0.25}\PYG{p}{]}\PYG{o}{\PYGZsq{}}\PYG{p}{;}\PYG{+w}{ }\PYG{c}{\PYGZpc{}Column Vector of Spatial Filter Indexed wrt corrosponding Channels}
\PYG{n+nb}{time}\PYG{p}{=}\PYG{l+m+mi}{0}\PYG{p}{;}
\PYG{n}{plasticity\PYGZus{}protocol\PYGZus{}sequence}\PYG{p}{=}\PYG{p}{[}\PYG{p}{]}\PYG{p}{;}

\PYG{c}{\PYGZpc{}\PYGZpc{} Initializing bossdevice RESEARCH API}
\PYG{n}{bd}\PYG{p}{=}\PYG{n}{bossdevice}\PYG{p}{;}
\PYG{n}{bd}\PYG{p}{.}\PYG{n}{sample\PYGZus{}and\PYGZus{}hold\PYGZus{}period}\PYG{p}{=}\PYG{l+m+mi}{0}\PYG{p}{;}
\PYG{n}{bd}\PYG{p}{.}\PYG{n}{calibration\PYGZus{}mode}\PYG{+w}{ }\PYG{p}{=}\PYG{+w}{ }\PYG{l+s}{\PYGZsq{}}\PYG{l+s}{no\PYGZsq{}}\PYG{p}{;}
\PYG{n}{bd}\PYG{p}{.}\PYG{n}{armed}\PYG{+w}{ }\PYG{p}{=}\PYG{+w}{ }\PYG{l+s}{\PYGZsq{}}\PYG{l+s}{no\PYGZsq{}}\PYG{p}{;}
\PYG{n}{bd}\PYG{p}{.}\PYG{n}{sample\PYGZus{}and\PYGZus{}hold\PYGZus{}period}\PYG{p}{=}\PYG{l+m+mi}{0}\PYG{p}{;}
\PYG{n}{bd}\PYG{p}{.}\PYG{n}{theta}\PYG{p}{.}\PYG{n}{ignore}\PYG{p}{;}\PYG{+w}{ }\PYG{n+nb}{pause}\PYG{p}{(}\PYG{l+m+mf}{0.1}\PYG{p}{)}
\PYG{n}{bd}\PYG{p}{.}\PYG{n}{beta}\PYG{p}{.}\PYG{n}{ignore}\PYG{p}{;}\PYG{+w}{ }\PYG{n+nb}{pause}\PYG{p}{(}\PYG{l+m+mf}{0.1}\PYG{p}{)}
\PYG{n}{bd}\PYG{p}{.}\PYG{n}{alpha}\PYG{p}{.}\PYG{n}{ignore}\PYG{p}{;}\PYG{+w}{ }\PYG{n+nb}{pause}\PYG{p}{(}\PYG{l+m+mf}{0.1}\PYG{p}{)}
\PYG{n}{bd}\PYG{p}{.}\PYG{n}{eeg\PYGZus{}channels}\PYG{p}{=}\PYG{n}{eeg\PYGZus{}channels}\PYG{p}{;}

\PYG{c}{\PYGZpc{}\PYGZpc{} Preparing an Individual Peak Frequency based Band Pass Filter for mu Alpha}
\PYG{n}{bpf\PYGZus{}fir\PYGZus{}coeffs}\PYG{+w}{ }\PYG{p}{=}\PYG{+w}{ }\PYG{n}{firls}\PYG{p}{(}\PYG{n}{bandpassfilter\PYGZus{}order}\PYG{p}{,}\PYG{+w}{ }\PYG{p}{[}\PYG{l+m+mi}{0}\PYG{+w}{ }\PYG{p}{(}\PYG{n}{individual\PYGZus{}peak\PYGZus{}frequency}\PYG{+w}{ }\PYG{o}{+}\PYG{+w}{ }\PYG{p}{[}\PYG{o}{\PYGZhy{}}\PYG{l+m+mi}{5}\PYG{+w}{ }\PYG{o}{\PYGZhy{}}\PYG{l+m+mi}{2}\PYG{+w}{ }\PYG{o}{+}\PYG{l+m+mi}{2}\PYG{+w}{ }\PYG{o}{+}\PYG{l+m+mi}{5}\PYG{p}{]}\PYG{p}{)}\PYG{+w}{ }\PYG{p}{(}\PYG{l+m+mi}{500}\PYG{o}{/}\PYG{l+m+mi}{2}\PYG{p}{)}\PYG{p}{]}\PYG{o}{/}\PYG{p}{(}\PYG{l+m+mi}{500}\PYG{o}{/}\PYG{l+m+mi}{2}\PYG{p}{)}\PYG{p}{,}\PYG{+w}{ }\PYG{p}{[}\PYG{l+m+mi}{0}\PYG{+w}{ }\PYG{l+m+mi}{0}\PYG{+w}{ }\PYG{l+m+mi}{1}\PYG{+w}{ }\PYG{l+m+mi}{1}\PYG{+w}{ }\PYG{l+m+mi}{0}\PYG{+w}{ }\PYG{l+m+mi}{0}\PYG{p}{]}\PYG{p}{,}\PYG{+w}{ }\PYG{p}{[}\PYG{l+m+mi}{1}\PYG{+w}{ }\PYG{l+m+mi}{1}\PYG{+w}{ }\PYG{l+m+mi}{1}\PYG{p}{]}\PYG{+w}{ }\PYG{p}{)}\PYG{p}{;}

\PYG{c}{\PYGZpc{}\PYGZpc{} Setting Filters on bossdevice RESEARCH}
\PYG{n}{bd}\PYG{p}{.}\PYG{n}{spatial\PYGZus{}filter\PYGZus{}weights}\PYG{p}{=}\PYG{n}{spatial\PYGZus{}filter\PYGZus{}weights}\PYG{p}{;}
\PYG{n}{bd}\PYG{p}{.}\PYG{n}{alpha}\PYG{p}{.}\PYG{n}{bpf\PYGZus{}fir\PYGZus{}coeffs}\PYG{+w}{ }\PYG{p}{=}\PYG{+w}{ }\PYG{n}{bpf\PYGZus{}fir\PYGZus{}coeffs}\PYG{p}{;}

\PYG{c}{\PYGZpc{}\PYGZpc{} Configuring Real\PYGZhy{}Time Scopes for Amplitude Tracking}
\PYG{n}{AMP\PYGZus{}TRACING\PYGZus{}SCOPES\PYGZus{}IDS}\PYG{+w}{ }\PYG{p}{=}\PYG{+w}{ }\PYG{p}{[}\PYG{l+m+mi}{101}\PYG{+w}{ }\PYG{l+m+mi}{102}\PYG{p}{]}\PYG{p}{;}

\PYG{c}{\PYGZpc{} remove any pre\PYGZhy{}existing scopes with these ids}
\PYG{k}{for}\PYG{+w}{ }\PYG{n}{id}\PYG{+w}{ }\PYG{p}{=}\PYG{+w}{ }\PYG{n}{AMP\PYGZus{}TRACING\PYGZus{}SCOPES\PYGZus{}IDS}
\PYG{+w}{    }\PYG{k}{if}\PYG{p}{(}\PYG{n+nb}{find}\PYG{p}{(}\PYG{n}{bd}\PYG{p}{.}\PYG{n}{tg}\PYG{p}{.}\PYG{n}{Scopes}\PYG{+w}{ }\PYG{o}{==}\PYG{+w}{ }\PYG{n}{id}\PYG{p}{)}\PYG{p}{)}
\PYG{+w}{        }\PYG{n+nb}{fprintf}\PYG{p}{(}\PYG{l+s}{\PYGZsq{}}\PYG{l+s}{\PYGZbs{}nRemoving scope \PYGZpc{}i\PYGZsq{}}\PYG{p}{,}\PYG{+w}{ }\PYG{n}{id}\PYG{p}{)}
\PYG{+w}{        }\PYG{n}{remscope}\PYG{p}{(}\PYG{n}{bd}\PYG{p}{.}\PYG{n}{tg}\PYG{p}{,}\PYG{+w}{ }\PYG{n}{id}\PYG{p}{)}\PYG{p}{;}
\PYG{+w}{    }\PYG{k}{end}
\PYG{k}{end}

\PYG{n}{sig\PYGZus{}id\PYGZus{}amp}\PYG{+w}{ }\PYG{p}{=}\PYG{+w}{ }\PYG{n}{getsignalid}\PYG{p}{(}\PYG{n}{bd}\PYG{p}{.}\PYG{n}{tg}\PYG{p}{,}\PYG{+w}{ }\PYG{l+s}{\PYGZsq{}}\PYG{l+s}{OSC/alpha/IA\PYGZsq{}}\PYG{p}{)}\PYG{p}{;}\PYG{+w}{ }\PYG{c}{\PYGZpc{}amplitude}
\PYG{n}{sig\PYGZus{}id\PYGZus{}qly}\PYG{+w}{ }\PYG{p}{=}\PYG{+w}{ }\PYG{n}{getsignalid}\PYG{p}{(}\PYG{n}{bd}\PYG{p}{.}\PYG{n}{tg}\PYG{p}{,}\PYG{+w}{ }\PYG{l+s}{\PYGZsq{}}\PYG{l+s}{QLY/Logical Operator2\PYGZsq{}}\PYG{p}{)}\PYG{p}{;}\PYG{+w}{ }\PYG{c}{\PYGZpc{}eeg\PYGZus{}is\PYGZus{}clean}

\PYG{n}{sc}\PYG{+w}{ }\PYG{p}{=}\PYG{+w}{ }\PYG{n}{addscope}\PYG{p}{(}\PYG{n}{bd}\PYG{p}{.}\PYG{n}{tg}\PYG{p}{,}\PYG{+w}{ }\PYG{l+s}{\PYGZsq{}}\PYG{l+s}{host\PYGZsq{}}\PYG{p}{,}\PYG{+w}{ }\PYG{n}{AMP\PYGZus{}TRACING\PYGZus{}SCOPES\PYGZus{}IDS}\PYG{p}{)}\PYG{p}{;}
\PYG{n}{addsignal}\PYG{p}{(}\PYG{n}{sc}\PYG{p}{,}\PYG{+w}{ }\PYG{p}{[}\PYG{n}{sig\PYGZus{}id\PYGZus{}amp}\PYG{+w}{ }\PYG{n}{sig\PYGZus{}id\PYGZus{}qly}\PYG{p}{]}\PYG{p}{)}\PYG{p}{;}

\PYG{n}{sc}\PYG{p}{(}\PYG{l+m+mi}{1}\PYG{p}{)}\PYG{p}{.}\PYG{n}{NumSamples}\PYG{+w}{ }\PYG{p}{=}\PYG{+w}{ }\PYG{l+m+mi}{500}\PYG{p}{;}
\PYG{n}{sc}\PYG{p}{(}\PYG{l+m+mi}{1}\PYG{p}{)}\PYG{p}{.}\PYG{n}{Decimation}\PYG{+w}{ }\PYG{p}{=}\PYG{+w}{ }\PYG{l+m+mi}{10}\PYG{p}{;}
\PYG{n}{sc}\PYG{p}{(}\PYG{l+m+mi}{1}\PYG{p}{)}\PYG{p}{.}\PYG{n}{TriggerSample}\PYG{+w}{ }\PYG{p}{=}\PYG{+w}{ }\PYG{o}{\PYGZhy{}}\PYG{l+m+mi}{1}\PYG{p}{;}

\PYG{n}{sc}\PYG{p}{(}\PYG{l+m+mi}{2}\PYG{p}{)}\PYG{p}{.}\PYG{n}{NumSamples}\PYG{+w}{ }\PYG{p}{=}\PYG{+w}{ }\PYG{l+m+mi}{500}\PYG{p}{;}
\PYG{n}{sc}\PYG{p}{(}\PYG{l+m+mi}{2}\PYG{p}{)}\PYG{p}{.}\PYG{n}{Decimation}\PYG{+w}{ }\PYG{p}{=}\PYG{+w}{ }\PYG{l+m+mi}{10}\PYG{p}{;}
\PYG{n}{sc}\PYG{p}{(}\PYG{l+m+mi}{2}\PYG{p}{)}\PYG{p}{.}\PYG{n}{TriggerSample}\PYG{+w}{ }\PYG{p}{=}\PYG{+w}{ }\PYG{o}{\PYGZhy{}}\PYG{l+m+mi}{1}\PYG{p}{;}

\PYG{n}{sc}\PYG{p}{(}\PYG{l+m+mi}{1}\PYG{p}{)}\PYG{p}{.}\PYG{n}{TriggerMode}\PYG{+w}{ }\PYG{p}{=}\PYG{+w}{ }\PYG{l+s}{\PYGZsq{}}\PYG{l+s}{Scope\PYGZsq{}}\PYG{p}{;}
\PYG{n}{sc}\PYG{p}{(}\PYG{l+m+mi}{1}\PYG{p}{)}\PYG{p}{.}\PYG{n}{TriggerScope}\PYG{+w}{ }\PYG{p}{=}\PYG{+w}{ }\PYG{n}{AMP\PYGZus{}TRACING\PYGZus{}SCOPES\PYGZus{}IDS}\PYG{p}{(}\PYG{l+m+mi}{2}\PYG{p}{)}\PYG{p}{;}

\PYG{n}{sc}\PYG{p}{(}\PYG{l+m+mi}{2}\PYG{p}{)}\PYG{p}{.}\PYG{n}{TriggerMode}\PYG{+w}{ }\PYG{p}{=}\PYG{+w}{ }\PYG{l+s}{\PYGZsq{}}\PYG{l+s}{Scope\PYGZsq{}}\PYG{p}{;}
\PYG{n}{sc}\PYG{p}{(}\PYG{l+m+mi}{2}\PYG{p}{)}\PYG{p}{.}\PYG{n}{TriggerScope}\PYG{+w}{ }\PYG{p}{=}\PYG{+w}{ }\PYG{n}{AMP\PYGZus{}TRACING\PYGZus{}SCOPES\PYGZus{}IDS}\PYG{p}{(}\PYG{l+m+mi}{1}\PYG{p}{)}\PYG{p}{;}

\PYG{n+nb}{start}\PYG{p}{(}\PYG{n}{sc}\PYG{p}{)}\PYG{p}{;}\PYG{+w}{ }\PYG{c}{\PYGZpc{} now they are ready for TTL output generation}

\PYG{n}{activeScope}\PYG{+w}{ }\PYG{p}{=}\PYG{+w}{ }\PYG{l+m+mi}{1}\PYG{p}{;}
\PYG{n}{mAmplitudeScopeCircBufTotalBlocks}\PYG{+w}{ }\PYG{p}{=}\PYG{+w}{ }\PYG{n}{amplitude\PYGZus{}assignment\PYGZus{}period}\PYG{p}{;}
\PYG{n}{mAmplitudeScopeCircBufCurrentBlock}\PYG{+w}{ }\PYG{p}{=}\PYG{+w}{ }\PYG{l+m+mi}{1}\PYG{p}{;}
\PYG{n}{mAmplitudeScopeCircBuf}\PYG{+w}{ }\PYG{p}{=}\PYG{+w}{ }\PYG{p}{[}\PYG{p}{]}\PYG{p}{;}
\PYG{n}{hAmplitudeHistoryAxes}\PYG{+w}{ }\PYG{p}{=}\PYG{+w}{ }\PYG{n+nb}{subplot}\PYG{p}{(}\PYG{l+m+mi}{1}\PYG{p}{,}\PYG{l+m+mi}{2}\PYG{p}{,}\PYG{l+m+mi}{1}\PYG{p}{)}\PYG{p}{;}
\PYG{n}{hAmplitudeDistributionAxes}\PYG{+w}{ }\PYG{p}{=}\PYG{+w}{ }\PYG{n+nb}{subplot}\PYG{p}{(}\PYG{l+m+mi}{1}\PYG{p}{,}\PYG{l+m+mi}{2}\PYG{p}{,}\PYG{l+m+mi}{2}\PYG{p}{)}\PYG{p}{;}

\PYG{n}{trigger}\PYG{p}{(}\PYG{n}{sc}\PYG{p}{(}\PYG{n}{activeScope}\PYG{p}{)}\PYG{p}{)}\PYG{p}{;}

\PYG{c}{\PYGZpc{}\PYGZpc{} Controlling bossdevice RESEARCH for mu Alpha Phase Locked TTL Output}
\PYG{n}{condition\PYGZus{}index}\PYG{p}{=}\PYG{l+m+mi}{0}\PYG{p}{;}
\PYG{k}{while}\PYG{+w}{ }\PYG{p}{(}\PYG{n}{condition\PYGZus{}index}\PYG{+w}{ }\PYG{o}{\PYGZlt{}=}\PYG{+w}{ }\PYG{n}{no\PYGZus{}of\PYGZus{}trials}\PYG{p}{)}
\PYG{+w}{    }\PYG{k}{if}\PYG{+w}{ }\PYG{p}{(}\PYG{n+nb}{strcmp}\PYG{p}{(}\PYG{n}{sc}\PYG{p}{(}\PYG{n}{activeScope}\PYG{p}{)}\PYG{p}{.}\PYG{n}{Status}\PYG{p}{,}\PYG{+w}{ }\PYG{l+s}{\PYGZsq{}}\PYG{l+s}{Finished\PYGZsq{}}\PYG{p}{)}\PYG{+w}{ }\PYG{o}{||}\PYG{+w}{ }\PYG{k}{...}
\PYG{+w}{            }\PYG{n+nb}{strcmp}\PYG{p}{(}\PYG{n}{sc}\PYG{p}{(}\PYG{n}{activeScope}\PYG{p}{)}\PYG{p}{.}\PYG{n}{Status}\PYG{p}{,}\PYG{+w}{ }\PYG{l+s}{\PYGZsq{}}\PYG{l+s}{Interrupted\PYGZsq{}}\PYG{p}{)}\PYG{p}{)}

\PYG{+w}{        }\PYG{n+nb}{time}\PYG{+w}{ }\PYG{p}{=}\PYG{+w}{ }\PYG{n}{sc}\PYG{p}{(}\PYG{n}{activeScope}\PYG{p}{)}\PYG{p}{.}\PYG{n}{Time}\PYG{p}{;}
\PYG{+w}{        }\PYG{n}{data}\PYG{+w}{ }\PYG{p}{=}\PYG{+w}{ }\PYG{n}{sc}\PYG{p}{(}\PYG{n}{activeScope}\PYG{p}{)}\PYG{p}{.}\PYG{n}{Data}\PYG{p}{;}
\PYG{+w}{        }\PYG{n+nb}{plot}\PYG{p}{(}\PYG{n}{hAmplitudeHistoryAxes}\PYG{p}{,}\PYG{+w}{ }\PYG{n+nb}{time}\PYG{p}{,}\PYG{+w}{ }\PYG{n}{data}\PYG{p}{(}\PYG{p}{:}\PYG{p}{,}\PYG{l+m+mi}{1}\PYG{p}{)}\PYG{p}{)}\PYG{p}{;}

\PYG{+w}{        }\PYG{n+nb}{fprintf}\PYG{p}{(}\PYG{p}{[}\PYG{l+s}{\PYGZsq{}}\PYG{l+s}{Restarting Scope \PYGZsq{}}\PYG{+w}{ }\PYG{n+nb}{num2str}\PYG{p}{(}\PYG{n}{activeScope}\PYG{p}{)}\PYG{p}{]}\PYG{p}{)}\PYG{p}{;}

\PYG{+w}{        }\PYG{c}{\PYGZpc{} Restart this scope.}
\PYG{+w}{        }\PYG{n+nb}{start}\PYG{p}{(}\PYG{n}{sc}\PYG{p}{(}\PYG{n}{activeScope}\PYG{p}{)}\PYG{p}{)}\PYG{p}{;}

\PYG{+w}{        }\PYG{c}{\PYGZpc{} Switch to the next scope.}
\PYG{+w}{        }\PYG{k}{if}\PYG{p}{(}\PYG{n}{activeScope}\PYG{+w}{ }\PYG{o}{==}\PYG{+w}{ }\PYG{l+m+mi}{1}\PYG{p}{)}
\PYG{+w}{            }\PYG{n}{activeScope}\PYG{+w}{ }\PYG{p}{=}\PYG{+w}{ }\PYG{l+m+mi}{2}\PYG{p}{;}
\PYG{+w}{        }\PYG{k}{else}
\PYG{+w}{            }\PYG{n}{activeScope}\PYG{+w}{ }\PYG{p}{=}\PYG{+w}{ }\PYG{l+m+mi}{1}\PYG{p}{;}
\PYG{+w}{        }\PYG{k}{end}

\PYG{+w}{        }\PYG{c}{\PYGZpc{} append data in circular buffer}
\PYG{+w}{        }\PYG{n}{mAmplitudeScopeCircBuf}\PYG{p}{\PYGZob{}}\PYG{n}{mAmplitudeScopeCircBufCurrentBlock}\PYG{p}{\PYGZcb{}}\PYG{+w}{ }\PYG{p}{=}\PYG{+w}{ }\PYG{n}{data}\PYG{o}{\PYGZsq{}}\PYG{p}{;}

\PYG{+w}{        }\PYG{n}{maxmindata}\PYG{+w}{ }\PYG{p}{=}\PYG{+w}{ }\PYG{n+nb}{cell2mat}\PYG{p}{(}\PYG{n+nb}{cellfun}\PYG{p}{(}\PYG{p}{@}\PYG{p}{(}\PYG{n}{data}\PYG{p}{)}\PYG{+w}{ }\PYG{n}{quantile}\PYG{p}{(}\PYG{n}{data}\PYG{p}{(}\PYG{l+m+mi}{1}\PYG{p}{,}\PYG{+w}{ }\PYG{n}{data}\PYG{p}{(}\PYG{l+m+mi}{2}\PYG{p}{,}\PYG{p}{:}\PYG{p}{)}\PYG{+w}{ }\PYG{o}{==}\PYG{+w}{ }\PYG{l+m+mi}{1}\PYG{p}{)}\PYG{p}{,}\PYG{+w}{ }\PYG{p}{[}\PYG{n}{amplitude\PYGZus{}threshold}\PYG{p}{(}\PYG{l+m+mi}{1}\PYG{p}{)}\PYG{o}{/}\PYG{l+m+mi}{100}\PYG{+w}{ }\PYG{n}{amplitude\PYGZus{}threshold}\PYG{p}{(}\PYG{l+m+mi}{2}\PYG{p}{)}\PYG{o}{/}\PYG{l+m+mi}{100}\PYG{p}{]}\PYG{p}{)}\PYG{o}{\PYGZsq{}}\PYG{p}{,}\PYG{+w}{ }\PYG{n}{mAmplitudeScopeCircBuf}\PYG{p}{,}\PYG{+w}{ }\PYG{l+s}{\PYGZsq{}}\PYG{l+s}{UniformOutput\PYGZsq{}}\PYG{p}{,}\PYG{+w}{ }\PYG{n+nb}{false}\PYG{p}{)}\PYG{p}{)}\PYG{o}{\PYGZsq{}}\PYG{p}{;}
\PYG{+w}{        }\PYG{n}{maxmindata}\PYG{+w}{ }\PYG{p}{=}\PYG{+w}{ }\PYG{n+nb}{circshift}\PYG{p}{(}\PYG{n}{maxmindata}\PYG{p}{,}\PYG{+w}{ }\PYG{n}{mAmplitudeScopeCircBufCurrentBlock}\PYG{p}{)}\PYG{p}{;}
\PYG{+w}{        }\PYG{n+nb}{plot}\PYG{p}{(}\PYG{n}{hAmplitudeHistoryAxes}\PYG{p}{,}\PYG{+w}{ }\PYG{n}{maxmindata}\PYG{p}{)}
\PYG{+w}{        }\PYG{n+nb}{xlim}\PYG{p}{(}\PYG{n}{hAmplitudeHistoryAxes}\PYG{p}{,}\PYG{+w}{ }\PYG{p}{[}\PYG{l+m+mi}{1}\PYG{+w}{ }\PYG{n}{mAmplitudeScopeCircBufTotalBlocks}\PYG{p}{]}\PYG{p}{)}
\PYG{+w}{        }\PYG{n+nb}{set}\PYG{p}{(}\PYG{n}{hAmplitudeHistoryAxes}\PYG{p}{,}\PYG{+w}{ }\PYG{l+s}{\PYGZsq{}}\PYG{l+s}{Xdir\PYGZsq{}}\PYG{p}{,}\PYG{+w}{ }\PYG{l+s}{\PYGZsq{}}\PYG{l+s}{reverse\PYGZsq{}}\PYG{p}{)}

\PYG{+w}{        }\PYG{n}{circular\PYGZus{}buffer\PYGZus{}data}\PYG{+w}{ }\PYG{p}{=}\PYG{+w}{ }\PYG{n+nb}{cell2mat}\PYG{p}{(}\PYG{n}{mAmplitudeScopeCircBuf}\PYG{p}{)}\PYG{p}{;}

\PYG{+w}{        }\PYG{c}{\PYGZpc{} Switch to the next data block}
\PYG{+w}{        }\PYG{k}{if}\PYG{p}{(}\PYG{n}{mAmplitudeScopeCircBufCurrentBlock}\PYG{+w}{ }\PYG{o}{\PYGZlt{}}\PYG{+w}{ }\PYG{n}{mAmplitudeScopeCircBufTotalBlocks}\PYG{p}{)}
\PYG{+w}{            }\PYG{n}{mAmplitudeScopeCircBufCurrentBlock}\PYG{+w}{ }\PYG{p}{=}\PYG{+w}{ }\PYG{n}{mAmplitudeScopeCircBufCurrentBlock}\PYG{+w}{ }\PYG{o}{+}\PYG{+w}{ }\PYG{l+m+mi}{1}\PYG{p}{;}
\PYG{+w}{        }\PYG{k}{else}
\PYG{+w}{            }\PYG{n}{mAmplitudeScopeCircBufCurrentBlock}\PYG{+w}{ }\PYG{p}{=}\PYG{+w}{ }\PYG{l+m+mi}{1}\PYG{p}{;}
\PYG{+w}{        }\PYG{k}{end}

\PYG{+w}{        }\PYG{c}{\PYGZpc{}tic}

\PYG{+w}{        }\PYG{c}{\PYGZpc{} remove post\PYGZhy{}TTL pulse data}
\PYG{+w}{        }\PYG{n}{amplitude\PYGZus{}clean}\PYG{+w}{ }\PYG{p}{=}\PYG{+w}{ }\PYG{n}{circular\PYGZus{}buffer\PYGZus{}data}\PYG{p}{(}\PYG{l+m+mi}{1}\PYG{p}{,}\PYG{+w}{ }\PYG{n}{circular\PYGZus{}buffer\PYGZus{}data}\PYG{p}{(}\PYG{l+m+mi}{2}\PYG{p}{,}\PYG{p}{:}\PYG{p}{)}\PYG{+w}{ }\PYG{o}{==}\PYG{+w}{ }\PYG{l+m+mi}{1}\PYG{p}{)}\PYG{p}{;}

\PYG{+w}{        }\PYG{c}{\PYGZpc{} calculate percentiles}
\PYG{+w}{        }\PYG{n}{amplitude\PYGZus{}sorted}\PYG{+w}{ }\PYG{p}{=}\PYG{+w}{ }\PYG{n+nb}{sort}\PYG{p}{(}\PYG{n}{amplitude\PYGZus{}clean}\PYG{p}{)}\PYG{p}{;}
\PYG{+w}{        }\PYG{n+nb}{plot}\PYG{p}{(}\PYG{n}{hAmplitudeDistributionAxes}\PYG{p}{,}\PYG{+w}{ }\PYG{n}{amplitude\PYGZus{}sorted}\PYG{p}{)}

\PYG{+w}{        }\PYG{n}{amp\PYGZus{}lower}\PYG{+w}{ }\PYG{p}{=}\PYG{+w}{ }\PYG{n}{quantile}\PYG{p}{(}\PYG{n}{amplitude\PYGZus{}clean}\PYG{p}{,}\PYG{+w}{ }\PYG{n}{amplitude\PYGZus{}threshold}\PYG{p}{(}\PYG{l+m+mi}{1}\PYG{p}{)}\PYG{o}{/}\PYG{l+m+mi}{100}\PYG{p}{)}\PYG{p}{;}\PYG{+w}{ }\PYG{c}{\PYGZpc{} TODO: INCLUDE THIS IN INFO STRUCT}
\PYG{+w}{        }\PYG{n}{amp\PYGZus{}upper}\PYG{+w}{ }\PYG{p}{=}\PYG{+w}{ }\PYG{n}{quantile}\PYG{p}{(}\PYG{n}{amplitude\PYGZus{}clean}\PYG{p}{,}\PYG{+w}{ }\PYG{n}{amplitude\PYGZus{}threshold}\PYG{p}{(}\PYG{l+m+mi}{2}\PYG{p}{)}\PYG{o}{/}\PYG{l+m+mi}{100}\PYG{p}{)}\PYG{p}{;}\PYG{+w}{ }\PYG{c}{\PYGZpc{} TODO: INCLUDE THIS IN INFO STRUCT}

\PYG{+w}{        }\PYG{n+nb}{hold}\PYG{p}{(}\PYG{n}{hAmplitudeDistributionAxes}\PYG{p}{,}\PYG{+w}{ }\PYG{l+s}{\PYGZsq{}}\PYG{l+s}{on\PYGZsq{}}\PYG{p}{)}
\PYG{+w}{        }\PYG{n+nb}{plot}\PYG{p}{(}\PYG{n}{hAmplitudeDistributionAxes}\PYG{p}{,}\PYG{+w}{ }\PYG{p}{[}\PYG{l+m+mi}{1}\PYG{+w}{ }\PYG{n+nb}{length}\PYG{p}{(}\PYG{n}{amplitude\PYGZus{}clean}\PYG{p}{)}\PYG{p}{]}\PYG{p}{,}\PYG{+w}{ }\PYG{p}{[}\PYG{n}{amp\PYGZus{}lower}\PYG{+w}{ }\PYG{n}{amp\PYGZus{}upper}\PYG{p}{;}\PYG{+w}{ }\PYG{n}{amp\PYGZus{}lower}\PYG{+w}{ }\PYG{l+s}{amp\PYGZus{}upper])}\PYG{p}{;}
\PYG{+w}{        }\PYG{n+nb}{hold}\PYG{p}{(}\PYG{n}{hAmplitudeDistributionAxes}\PYG{p}{,}\PYG{+w}{ }\PYG{l+s}{\PYGZsq{}}\PYG{l+s}{off\PYGZsq{}}\PYG{p}{)}

\PYG{+w}{        }\PYG{k}{if}\PYG{+w}{ }\PYG{n+nb}{length}\PYG{p}{(}\PYG{n}{amplitude\PYGZus{}clean}\PYG{p}{)}\PYG{+w}{ }\PYG{o}{\PYGZgt{}}\PYG{+w}{ }\PYG{l+m+mi}{1}
\PYG{+w}{            }\PYG{n+nb}{xlim}\PYG{p}{(}\PYG{n}{hAmplitudeDistributionAxes}\PYG{p}{,}\PYG{+w}{ }\PYG{p}{[}\PYG{l+m+mi}{1}\PYG{+w}{ }\PYG{n+nb}{length}\PYG{p}{(}\PYG{n}{amplitude\PYGZus{}clean}\PYG{p}{)}\PYG{p}{]}\PYG{p}{)}\PYG{p}{;}
\PYG{+w}{        }\PYG{k}{end}
\PYG{+w}{        }\PYG{k}{if}\PYG{+w}{ }\PYG{p}{(}\PYG{n}{amplitude\PYGZus{}sorted}\PYG{p}{(}\PYG{k}{end}\PYG{p}{)}\PYG{+w}{ }\PYG{o}{\PYGZgt{}}\PYG{+w}{ }\PYG{n}{amplitude\PYGZus{}sorted}\PYG{p}{(}\PYG{l+m+mi}{1}\PYG{p}{)}\PYG{p}{)}
\PYG{+w}{            }\PYG{n+nb}{ylim}\PYG{p}{(}\PYG{n}{hAmplitudeDistributionAxes}\PYG{p}{,}\PYG{+w}{ }\PYG{p}{[}\PYG{n}{amplitude\PYGZus{}sorted}\PYG{p}{(}\PYG{l+m+mi}{1}\PYG{p}{)}\PYG{+w}{ }\PYG{n}{amplitude\PYGZus{}sorted}\PYG{p}{(}\PYG{k}{end}\PYG{p}{)}\PYG{p}{]}\PYG{p}{)}\PYG{p}{;}
\PYG{+w}{        }\PYG{k}{end}

\PYG{+w}{        }\PYG{c}{\PYGZpc{}toc}

\PYG{+w}{        }\PYG{c}{\PYGZpc{} set amplitude threshold}
\PYG{+w}{        }\PYG{n}{mDbsp}\PYG{p}{.}\PYG{n}{alpha}\PYG{p}{.}\PYG{n}{amplitude\PYGZus{}min}\PYG{p}{(}\PYG{l+m+mi}{1}\PYG{p}{)}\PYG{+w}{ }\PYG{p}{=}\PYG{+w}{ }\PYG{n}{amp\PYGZus{}lower}\PYG{p}{;}
\PYG{+w}{        }\PYG{n}{mDbsp}\PYG{p}{.}\PYG{n}{alpha}\PYG{p}{.}\PYG{n}{amplitude\PYGZus{}max}\PYG{p}{(}\PYG{l+m+mi}{1}\PYG{p}{)}\PYG{+w}{ }\PYG{p}{=}\PYG{+w}{ }\PYG{n}{amp\PYGZus{}upper}\PYG{p}{;}
\PYG{+w}{        }\PYG{n}{bd}\PYG{p}{.}\PYG{n}{alpha}\PYG{p}{.}\PYG{n}{amplitude\PYGZus{}min}\PYG{p}{(}\PYG{l+m+mi}{1}\PYG{p}{)}\PYG{p}{=}\PYG{n}{amp\PYGZus{}lower}\PYG{p}{;}
\PYG{+w}{        }\PYG{n}{bd}\PYG{p}{.}\PYG{n}{alpha}\PYG{p}{.}\PYG{n}{amplitude\PYGZus{}max}\PYG{p}{(}\PYG{l+m+mi}{1}\PYG{p}{)}\PYG{p}{=}\PYG{n}{amp\PYGZus{}upper}\PYG{p}{;}
\PYG{+w}{        }\PYG{n+nb}{title}\PYG{p}{(}\PYG{n}{hAmplitudeDistributionAxes}\PYG{p}{,}\PYG{+w}{ }\PYG{p}{[}\PYG{l+s}{\PYGZsq{}}\PYG{l+s}{Min Amplitude: \PYGZsq{}}\PYG{p}{,}\PYG{+w}{ }\PYG{n+nb}{num2str}\PYG{p}{(}\PYG{n}{amp\PYGZus{}lower}\PYG{p}{)}\PYG{p}{]}\PYG{p}{)}\PYG{p}{;}

\PYG{+w}{    }\PYG{k}{end}\PYG{+w}{ }\PYG{c}{\PYGZpc{} handle the amplitude tracking}
\PYG{+w}{    }\PYG{k}{if}\PYG{p}{(}\PYG{n+nb}{strcmp}\PYG{p}{(}\PYG{n}{bd}\PYG{p}{.}\PYG{n}{armed}\PYG{p}{,}\PYG{+w}{ }\PYG{l+s}{\PYGZsq{}}\PYG{l+s}{no\PYGZsq{}}\PYG{p}{)}\PYG{p}{)}
\PYG{+w}{        }\PYG{n}{bd}\PYG{p}{.}\PYG{n}{pulses\PYGZus{}remaining}\PYG{+w}{ }\PYG{p}{=}\PYG{+w}{ }\PYG{l+m+mi}{1}\PYG{p}{;}
\PYG{+w}{        }\PYG{n}{bd}\PYG{p}{.}\PYG{n}{alpha}\PYG{p}{.}\PYG{n}{phase\PYGZus{}target}\PYG{p}{(}\PYG{l+m+mi}{1}\PYG{p}{)}\PYG{+w}{ }\PYG{p}{=}\PYG{+w}{ }\PYG{n}{phase}\PYG{p}{(}\PYG{n+nb}{randi}\PYG{p}{(}\PYG{l+m+mi}{1}\PYG{p}{:}\PYG{n+nb}{numel}\PYG{p}{(}\PYG{n}{phase}\PYG{p}{)}\PYG{p}{,}\PYG{+w}{ }\PYG{l+m+mi}{1}\PYG{p}{)}\PYG{p}{)}\PYG{p}{;}
\PYG{+w}{        }\PYG{n}{bd}\PYG{p}{.}\PYG{n}{alpha}\PYG{p}{.}\PYG{n}{phase\PYGZus{}plusminus}\PYG{p}{(}\PYG{l+m+mi}{1}\PYG{p}{)}\PYG{+w}{ }\PYG{p}{=}\PYG{+w}{ }\PYG{n}{phase\PYGZus{}tolerance}\PYG{p}{;}
\PYG{+w}{        }\PYG{n}{bd}\PYG{p}{.}\PYG{n}{configure\PYGZus{}time\PYGZus{}port\PYGZus{}marker}\PYG{p}{(}\PYG{p}{(}\PYG{p}{[}\PYG{l+m+mi}{0}\PYG{p}{,}\PYG{+w}{ }\PYG{l+m+mi}{1}\PYG{p}{,}\PYG{+w}{ }\PYG{l+m+mi}{0}\PYG{p}{]}\PYG{p}{)}\PYG{p}{)}
\PYG{+w}{        }\PYG{n}{bd}\PYG{p}{.}\PYG{n}{min\PYGZus{}inter\PYGZus{}pulse\PYGZus{}interval}\PYG{+w}{ }\PYG{p}{=}\PYG{+w}{ }\PYG{n}{minimium\PYGZus{}inter\PYGZus{}pulse\PYGZus{}interval}\PYG{p}{;}
\PYG{+w}{        }\PYG{n+nb}{pause}\PYG{p}{(}\PYG{l+m+mf}{0.1}\PYG{p}{)}
\PYG{+w}{        }\PYG{n}{bd}\PYG{p}{.}\PYG{n}{arm}\PYG{p}{;}
\PYG{+w}{    }\PYG{k}{end}
\PYG{+w}{    }\PYG{c}{\PYGZpc{} TTL output has been generated, move to the next condition}
\PYG{+w}{    }\PYG{k}{if}\PYG{p}{(}\PYG{n}{bd}\PYG{p}{.}\PYG{n}{pulses\PYGZus{}remaining}\PYG{+w}{ }\PYG{o}{==}\PYG{+w}{ }\PYG{l+m+mi}{0}\PYG{p}{)}
\PYG{+w}{        }\PYG{n}{condition\PYGZus{}index}\PYG{+w}{ }\PYG{p}{=}\PYG{+w}{ }\PYG{n}{condition\PYGZus{}index}\PYG{+w}{ }\PYG{o}{+}\PYG{+w}{ }\PYG{l+m+mi}{1}\PYG{p}{;}
\PYG{+w}{        }\PYG{n}{bd}\PYG{p}{.}\PYG{n}{disarm}\PYG{p}{;}
\PYG{+w}{        }\PYG{n}{disp}\PYG{+w}{ }\PYG{l+s}{TTL\PYGZus{}Output\PYGZus{}Has\PYGZus{}Been\PYGZus{}Generated!}
\PYG{+w}{    }\PYG{k}{end}
\PYG{+w}{    }\PYG{n+nb}{pause}\PYG{p}{(}\PYG{l+m+mf}{0.01}\PYG{p}{)}\PYG{p}{;}
\PYG{k}{end}

\PYG{c}{\PYGZpc{}\PYGZpc{} End}
\end{sphinxVerbatim}


\bigskip\hrule\bigskip



\section{Phase Prediction Error Measurement}
\label{\detokenize{5_demo_scripts:phase-prediction-error-measurement}}
\sphinxAtStartPar
This demo script uses bossdevice RESEARCH and 2 different approaches to generate jittered open loop TTL output

\sphinxAtStartPar
Resources: 1) bossdevice Switched On, 2) bossdevice Open Source MATLAB API

\begin{sphinxVerbatim}[commandchars=\\\{\}]
\PYG{n}{ang\PYGZus{}diff}\PYG{+w}{ }\PYG{p}{=}\PYG{+w}{ }\PYG{p}{@}\PYG{p}{(}\PYG{n}{x}\PYG{p}{,}\PYG{+w}{ }\PYG{n}{y}\PYG{p}{)}\PYG{+w}{ }\PYG{n+nb}{angle}\PYG{p}{(}\PYG{n+nb}{exp}\PYG{p}{(}1\PYG{n+nb}{i}\PYG{o}{*}\PYG{n}{x}\PYG{p}{)}\PYG{o}{./}\PYG{n+nb}{exp}\PYG{p}{(}1\PYG{n+nb}{i}\PYG{o}{*}\PYG{n}{y}\PYG{p}{)}\PYG{p}{)}\PYG{p}{;}
\PYG{n}{ang\PYGZus{}var}\PYG{+w}{ }\PYG{p}{=}\PYG{+w}{ }\PYG{p}{@}\PYG{p}{(}\PYG{n}{x}\PYG{p}{)}\PYG{+w}{ }\PYG{l+m+mi}{1}\PYG{o}{\PYGZhy{}}\PYG{n+nb}{abs}\PYG{p}{(}\PYG{n+nb}{mean}\PYG{p}{(}\PYG{n+nb}{exp}\PYG{p}{(}1\PYG{n+nb}{i}\PYG{o}{*}\PYG{n}{x}\PYG{p}{)}\PYG{p}{)}\PYG{p}{)}\PYG{p}{;}
\PYG{c}{\PYGZpc{}ang\PYGZus{}var2dev = @(v) sqrt(2*v); \PYGZpc{} circstat preferred formula uses angular deviation (bounded from 0 to sqrt(2)) which is sqrt(2*(1\PYGZhy{}r))}
\PYG{n}{ang\PYGZus{}var2dev}\PYG{+w}{ }\PYG{p}{=}\PYG{+w}{ }\PYG{p}{@}\PYG{p}{(}\PYG{n}{v}\PYG{p}{)}\PYG{+w}{ }\PYG{n+nb}{sqrt}\PYG{p}{(}\PYG{o}{\PYGZhy{}}\PYG{l+m+mi}{2}\PYG{o}{*}\PYG{n+nb}{log}\PYG{p}{(}\PYG{l+m+mi}{1}\PYG{o}{\PYGZhy{}}\PYG{n}{v}\PYG{p}{)}\PYG{p}{)}\PYG{p}{;}\PYG{+w}{ }\PYG{c}{\PYGZpc{} formula for circular standard deviation is sqrt(\PYGZhy{}2*ln(r))}

\PYG{c}{\PYGZpc{}\PYGZpc{}  Initializing bossdevice RESEARCH API}
\PYG{n}{bb}\PYG{+w}{ }\PYG{p}{=}\PYG{+w}{ }\PYG{n}{bossdevice}\PYG{p}{;}
\PYG{n}{bd}\PYG{p}{.}\PYG{n}{eeg\PYGZus{}channels}\PYG{+w}{ }\PYG{p}{=}\PYG{+w}{ }\PYG{l+m+mi}{1}\PYG{p}{;}
\PYG{n}{bd}\PYG{p}{.}\PYG{n}{aux\PYGZus{}channels}\PYG{+w}{ }\PYG{p}{=}\PYG{+w}{ }\PYG{l+m+mi}{1}\PYG{p}{;}
\PYG{n}{bd}\PYG{p}{.}\PYG{n}{spatial\PYGZus{}filter\PYGZus{}weights}\PYG{+w}{ }\PYG{p}{=}\PYG{+w}{ }\PYG{l+m+mi}{1}\PYG{p}{;}

\PYG{n}{bd}\PYG{p}{.}\PYG{n}{alpha}\PYG{p}{.}\PYG{n}{offset\PYGZus{}samples}\PYG{+w}{ }\PYG{p}{=}\PYG{+w}{ }\PYG{l+m+mi}{5}\PYG{p}{;}\PYG{+w}{ }\PYG{c}{\PYGZpc{}this depends on the loop\PYGZhy{}delay}

\PYG{c}{\PYGZpc{}\PYGZpc{} Setting Filters to bossdevice RESEARCH}
\PYG{c}{\PYGZpc{} this allows calibrating the oscillation analysis to an individual peak frequency}
\PYG{n}{bd}\PYG{p}{.}\PYG{n}{alpha}\PYG{p}{.}\PYG{n}{bpf\PYGZus{}fir\PYGZus{}coeffs}\PYG{+w}{ }\PYG{p}{=}\PYG{+w}{ }\PYG{n}{firls}\PYG{p}{(}\PYG{l+m+mi}{70}\PYG{p}{,}\PYG{+w}{ }\PYG{p}{[}\PYG{l+m+mi}{0}\PYG{+w}{ }\PYG{l+m+mi}{6}\PYG{+w}{ }\PYG{l+m+mi}{9}\PYG{+w}{ }\PYG{l+m+mi}{13}\PYG{+w}{ }\PYG{l+m+mi}{16}\PYG{+w}{ }\PYG{p}{(}\PYG{l+m+mi}{500}\PYG{o}{/}\PYG{l+m+mi}{2}\PYG{p}{)}\PYG{p}{]}\PYG{o}{/}\PYG{p}{(}\PYG{l+m+mi}{500}\PYG{o}{/}\PYG{l+m+mi}{2}\PYG{p}{)}\PYG{p}{,}\PYG{+w}{ }\PYG{p}{[}\PYG{l+m+mi}{0}\PYG{+w}{ }\PYG{l+m+mi}{0}\PYG{+w}{ }\PYG{l+m+mi}{1}\PYG{+w}{ }\PYG{l+m+mi}{1}\PYG{+w}{ }\PYG{l+m+mi}{0}\PYG{+w}{ }\PYG{l+m+mi}{0}\PYG{p}{]}\PYG{p}{,}\PYG{+w}{ }\PYG{p}{[}\PYG{l+m+mi}{1}\PYG{+w}{ }\PYG{l+m+mi}{1}\PYG{+w}{ }\PYG{l+m+mi}{1}\PYG{p}{]}\PYG{p}{)}\PYG{p}{;}
\PYG{c}{\PYGZpc{}fvtool(bd.alpha.bpf\PYGZus{}fir\PYGZus{}coeffs, \PYGZsq{}Fs\PYGZsq{}, 500) \PYGZpc{} visualize filter}

\PYG{c}{\PYGZpc{}\PYGZpc{} Configuring a scope to acquire data}
\PYG{n}{sc}\PYG{+w}{ }\PYG{p}{=}\PYG{+w}{ }\PYG{n}{addscope}\PYG{p}{(}\PYG{n}{bd}\PYG{p}{.}\PYG{n}{tg}\PYG{p}{,}\PYG{+w}{ }\PYG{l+s}{\PYGZsq{}}\PYG{l+s}{host\PYGZsq{}}\PYG{p}{,}\PYG{+w}{ }\PYG{l+m+mi}{101}\PYG{p}{)}\PYG{p}{;}
\PYG{n}{addsignal}\PYG{p}{(}\PYG{n}{sc}\PYG{p}{,}\PYG{+w}{ }\PYG{n}{getsignalid}\PYG{p}{(}\PYG{n}{bd}\PYG{p}{.}\PYG{n}{tg}\PYG{p}{,}\PYG{+w}{ }\PYG{l+s}{\PYGZsq{}}\PYG{l+s}{SPF/Matrix Multiply\PYGZsq{}}\PYG{p}{)}\PYG{p}{)}\PYG{p}{;}\PYG{+w}{ }\PYG{c}{\PYGZpc{} this signals goes into the oscillation analysis}
\PYG{n}{addsignal}\PYG{p}{(}\PYG{n}{sc}\PYG{p}{,}\PYG{+w}{ }\PYG{n}{getsignalid}\PYG{p}{(}\PYG{n}{bd}\PYG{p}{.}\PYG{n}{tg}\PYG{p}{,}\PYG{+w}{ }\PYG{l+s}{\PYGZsq{}}\PYG{l+s}{OSC/alpha/IP\PYGZsq{}}\PYG{p}{)}\PYG{p}{)}\PYG{p}{;}\PYG{+w}{ }\PYG{c}{\PYGZpc{} instantaneous phase estimate for alpha}

\PYG{n}{sc}\PYG{p}{.}\PYG{n}{NumSamples}\PYG{+w}{ }\PYG{p}{=}\PYG{+w}{ }\PYG{l+m+mi}{10}\PYG{+w}{ }\PYG{o}{*}\PYG{+w}{ }\PYG{l+m+mi}{5000}\PYG{p}{;}
\PYG{n}{sc}\PYG{p}{.}\PYG{n}{Decimation}\PYG{+w}{ }\PYG{p}{=}\PYG{+w}{ }\PYG{l+m+mi}{1}\PYG{p}{;}

\PYG{n+nb}{fprintf}\PYG{p}{(}\PYG{l+s}{\PYGZsq{}}\PYG{l+s}{\PYGZbs{}nAcquiring data ...\PYGZsq{}}\PYG{p}{)}\PYG{p}{,}\PYG{+w}{ }\PYG{n+nb}{start}\PYG{p}{(}\PYG{n}{sc}\PYG{p}{)}\PYG{p}{;}
\PYG{k}{while}\PYG{p}{(}\PYG{n+nb}{strcmp}\PYG{p}{(}\PYG{n}{sc}\PYG{p}{.}\PYG{n}{Status}\PYG{p}{,}\PYG{+w}{ }\PYG{l+s}{\PYGZsq{}}\PYG{l+s}{Acquiring\PYGZsq{}}\PYG{p}{)}\PYG{p}{)}\PYG{p}{,}\PYG{+w}{ }\PYG{n+nb}{fprintf}\PYG{p}{(}\PYG{l+s}{\PYGZsq{}}\PYG{l+s}{.\PYGZsq{}}\PYG{p}{)}\PYG{p}{,}\PYG{+w}{ }\PYG{n+nb}{pause}\PYG{p}{(}\PYG{l+m+mi}{1}\PYG{p}{)}\PYG{p}{,}\PYG{+w}{ }\PYG{k}{end}
\PYG{n+nb}{fprintf}\PYG{p}{(}\PYG{l+s}{\PYGZsq{}}\PYG{l+s}{ done\PYGZsq{}}\PYG{p}{)}

\PYG{n}{data}\PYG{+w}{ }\PYG{p}{=}\PYG{+w}{ }\PYG{n}{sc}\PYG{p}{.}\PYG{n}{Data}\PYG{p}{(}\PYG{p}{:}\PYG{p}{,}\PYG{l+m+mi}{1}\PYG{p}{)}\PYG{p}{;}
\PYG{n}{ip\PYGZus{}estimate\PYGZus{}causal}\PYG{+w}{ }\PYG{p}{=}\PYG{+w}{ }\PYG{n}{sc}\PYG{p}{.}\PYG{n}{Data}\PYG{p}{(}\PYG{p}{:}\PYG{p}{,}\PYG{k}{end}\PYG{p}{)}\PYG{p}{;}
\PYG{n}{fs}\PYG{+w}{ }\PYG{p}{=}\PYG{+w}{ }\PYG{l+m+mi}{1}\PYG{o}{/}\PYG{n+nb}{mean}\PYG{p}{(}\PYG{n+nb}{diff}\PYG{p}{(}\PYG{n}{sc}\PYG{p}{.}\PYG{n}{Time}\PYG{p}{)}\PYG{p}{)}\PYG{p}{;}

\PYG{n+nb}{fprintf}\PYG{p}{(}\PYG{l+s}{\PYGZsq{}}\PYG{l+s}{\PYGZbs{}nDetermining phase using standard non\PYGZhy{}causal methods ...\PYGZsq{}}\PYG{p}{)}
\PYG{c}{\PYGZpc{} demean}
\PYG{n}{data}\PYG{+w}{ }\PYG{p}{=}\PYG{+w}{ }\PYG{n}{data}\PYG{+w}{ }\PYG{o}{\PYGZhy{}}\PYG{+w}{ }\PYG{n+nb}{mean}\PYG{p}{(}\PYG{n}{data}\PYG{p}{)}\PYG{p}{;}
\PYG{c}{\PYGZpc{} zero phase band\PYGZhy{}pass filter}
\PYG{n}{D}\PYG{+w}{ }\PYG{p}{=}\PYG{+w}{ }\PYG{n}{designfilt}\PYG{p}{(}\PYG{l+s}{\PYGZsq{}}\PYG{l+s}{bandpassfir\PYGZsq{}}\PYG{p}{,}\PYG{+w}{ }\PYG{l+s}{\PYGZsq{}}\PYG{l+s}{FilterOrder\PYGZsq{}}\PYG{p}{,}\PYG{+w}{ }\PYG{n+nb}{round}\PYG{p}{(}\PYG{l+m+mi}{1}\PYG{o}{*}\PYG{n}{fs}\PYG{p}{)}\PYG{p}{,}\PYG{+w}{ }\PYG{l+s}{\PYGZsq{}}\PYG{l+s}{CutoffFrequency1\PYGZsq{}}\PYG{p}{,}\PYG{+w}{ }\PYG{l+m+mi}{9}\PYG{p}{,}\PYG{+w}{ }\PYG{l+s}{\PYGZsq{}}\PYG{l+s}{CutoffFrequency2\PYGZsq{}}\PYG{p}{,}\PYG{+w}{ }\PYG{l+m+mi}{13}\PYG{p}{,}\PYG{+w}{ }\PYG{l+s}{\PYGZsq{}}\PYG{l+s}{SampleRate\PYGZsq{}}\PYG{p}{,}\PYG{+w}{ }\PYG{n}{fs}\PYG{p}{)}\PYG{p}{;}
\PYG{n}{data}\PYG{+w}{ }\PYG{p}{=}\PYG{+w}{ }\PYG{n}{filtfilt}\PYG{p}{(}\PYG{n}{D}\PYG{p}{,}\PYG{+w}{ }\PYG{n}{data}\PYG{p}{)}\PYG{p}{;}\PYG{+w}{ }\PYG{c}{\PYGZpc{}demean}

\PYG{n}{ip\PYGZus{}estimate\PYGZus{}noncausal}\PYG{+w}{ }\PYG{p}{=}\PYG{+w}{ }\PYG{n+nb}{angle}\PYG{p}{(}\PYG{n}{hilbert}\PYG{p}{(}\PYG{n}{data}\PYG{p}{)}\PYG{p}{)}\PYG{p}{;}
\PYG{n}{phase\PYGZus{}error}\PYG{+w}{ }\PYG{p}{=}\PYG{+w}{ }\PYG{n}{ang\PYGZus{}diff}\PYG{p}{(}\PYG{n}{ip\PYGZus{}estimate\PYGZus{}noncausal}\PYG{p}{,}\PYG{+w}{ }\PYG{n}{ip\PYGZus{}estimate\PYGZus{}causal}\PYG{p}{)}\PYG{p}{;}

\PYG{n+nb}{fprintf}\PYG{p}{(}\PYG{l+s}{\PYGZsq{}}\PYG{l+s}{\PYGZbs{}n\PYGZsq{}}\PYG{p}{)}

\PYG{c}{\PYGZpc{}\PYGZpc{} Visualize}
\PYG{n+nb}{figure}
\PYG{n}{ax1}\PYG{+w}{ }\PYG{p}{=}\PYG{+w}{ }\PYG{n+nb}{subplot}\PYG{p}{(}\PYG{l+m+mi}{2}\PYG{p}{,}\PYG{l+m+mi}{2}\PYG{p}{,}\PYG{l+m+mi}{1}\PYG{p}{)}\PYG{p}{;}
\PYG{n+nb}{plot}\PYG{p}{(}\PYG{n}{sc}\PYG{p}{.}\PYG{n}{Time}\PYG{o}{\PYGZhy{}}\PYG{n}{sc}\PYG{p}{.}\PYG{n}{Time}\PYG{p}{(}\PYG{l+m+mi}{1}\PYG{p}{)}\PYG{p}{,}\PYG{+w}{ }\PYG{p}{[}\PYG{n}{sc}\PYG{p}{.}\PYG{n}{Data}\PYG{p}{(}\PYG{p}{:}\PYG{p}{,}\PYG{l+m+mi}{1}\PYG{p}{)}\PYG{+w}{ }\PYG{n}{data}\PYG{p}{]}\PYG{p}{)}
\PYG{n}{ax2}\PYG{+w}{ }\PYG{p}{=}\PYG{+w}{ }\PYG{n+nb}{subplot}\PYG{p}{(}\PYG{l+m+mi}{2}\PYG{p}{,}\PYG{l+m+mi}{2}\PYG{p}{,}\PYG{l+m+mi}{2}\PYG{p}{)}\PYG{p}{;}
\PYG{n+nb}{plot}\PYG{p}{(}\PYG{n}{sc}\PYG{p}{.}\PYG{n}{Time}\PYG{o}{\PYGZhy{}}\PYG{n}{sc}\PYG{p}{.}\PYG{n}{Time}\PYG{p}{(}\PYG{l+m+mi}{1}\PYG{p}{)}\PYG{p}{,}\PYG{+w}{ }\PYG{p}{[}\PYG{n}{ip\PYGZus{}estimate\PYGZus{}causal}\PYG{+w}{ }\PYG{n}{ip\PYGZus{}estimate\PYGZus{}noncausal}\PYG{p}{]}\PYG{p}{)}
\PYG{n+nb}{linkaxes}\PYG{p}{(}\PYG{p}{[}\PYG{n}{ax1}\PYG{+w}{ }\PYG{n}{ax2}\PYG{p}{]}\PYG{p}{,}\PYG{+w}{ }\PYG{l+s}{\PYGZsq{}}\PYG{l+s}{x\PYGZsq{}}\PYG{p}{)}
\PYG{n+nb}{subplot}\PYG{p}{(}\PYG{l+m+mi}{2}\PYG{p}{,}\PYG{l+m+mi}{2}\PYG{p}{,}\PYG{l+m+mi}{4}\PYG{p}{,}\PYG{n+nb}{polaraxes}\PYG{p}{)}\PYG{p}{;}
\PYG{n+nb}{polarhistogram}\PYG{p}{(}\PYG{n}{phase\PYGZus{}error}\PYG{p}{,}\PYG{+w}{ }\PYG{l+s}{\PYGZsq{}}\PYG{l+s}{Normalization\PYGZsq{}}\PYG{p}{,}\PYG{+w}{ }\PYG{l+s}{\PYGZsq{}}\PYG{l+s}{probability\PYGZsq{}}\PYG{p}{,}\PYG{+w}{ }\PYG{l+s}{\PYGZsq{}}\PYG{l+s}{BinWidth\PYGZsq{}}\PYG{p}{,}\PYG{+w}{ }\PYG{n+nb}{pi}\PYG{o}{/}\PYG{l+m+mi}{36}\PYG{p}{)}\PYG{p}{;}
\PYG{n+nb}{title}\PYG{p}{(}\PYG{n+nb}{sprintf}\PYG{p}{(}\PYG{l+s}{\PYGZsq{}}\PYG{l+s}{circular standard deviation = \PYGZpc{}.1f°\PYGZsq{}}\PYG{p}{,}\PYG{+w}{ }\PYG{n+nb}{rad2deg}\PYG{p}{(}\PYG{n}{ang\PYGZus{}var2dev}\PYG{p}{(}\PYG{n}{ang\PYGZus{}var}\PYG{p}{(}\PYG{n}{phase\PYGZus{}error}\PYG{p}{)}\PYG{p}{)}\PYG{p}{)}\PYG{p}{)}\PYG{p}{)}

\PYG{c}{\PYGZpc{} remove the scope}
\PYG{n}{remscope}\PYG{p}{(}\PYG{n}{bd}\PYG{p}{.}\PYG{n}{tg}\PYG{p}{,}\PYG{+w}{ }\PYG{l+m+mi}{101}\PYG{p}{)}
\end{sphinxVerbatim}


\bigskip\hrule\bigskip



\section{Loop Latency Measurement}
\label{\detokenize{5_demo_scripts:loop-latency-measurement}}
\sphinxAtStartPar
This demo script uses bossdevice RESEARCH and 2 different approaches to generate jittered open loop TTL output

\sphinxAtStartPar
Resources: 1) bossdevice Switched On, 2) bossdevice Open Source MATLAB API

\begin{sphinxVerbatim}[commandchars=\\\{\}]
\PYG{n}{bd}\PYG{+w}{ }\PYG{p}{=}\PYG{+w}{ }\PYG{n}{bossdevice}\PYG{p}{;}

\PYG{c}{\PYGZpc{}\PYGZpc{} Configuring Scope}
\PYG{n}{sc}\PYG{+w}{ }\PYG{p}{=}\PYG{+w}{ }\PYG{n}{addscope}\PYG{p}{(}\PYG{n}{bd}\PYG{p}{.}\PYG{n}{tg}\PYG{p}{,}\PYG{+w}{ }\PYG{l+s}{\PYGZsq{}}\PYG{l+s}{host\PYGZsq{}}\PYG{p}{,}\PYG{+w}{ }\PYG{l+m+mi}{255}\PYG{p}{)}\PYG{p}{;}

\PYG{n}{mrk\PYGZus{}signal\PYGZus{}id}\PYG{+w}{ }\PYG{p}{=}\PYG{+w}{ }\PYG{n}{getsignalid}\PYG{p}{(}\PYG{n}{bd}\PYG{p}{.}\PYG{n}{tg}\PYG{p}{,}\PYG{+w}{ }\PYG{l+s}{\PYGZsq{}}\PYG{l+s}{UDP/raw\PYGZus{}mrk\PYGZsq{}}\PYG{p}{)}\PYG{+w}{ }\PYG{o}{+}\PYG{+w}{ }\PYG{n+nb}{int32}\PYG{p}{(}\PYG{p}{[}\PYG{l+m+mi}{0}\PYG{+w}{ }\PYG{l+m+mi}{1}\PYG{+w}{ }\PYG{l+m+mi}{2}\PYG{p}{]}\PYG{p}{)}\PYG{p}{;}

\PYG{n}{addsignal}\PYG{p}{(}\PYG{n}{sc}\PYG{p}{,}\PYG{+w}{ }\PYG{n}{mrk\PYGZus{}signal\PYGZus{}id}\PYG{p}{)}\PYG{p}{;}
\PYG{n}{sc}\PYG{p}{.}\PYG{n}{NumSamples}\PYG{+w}{ }\PYG{p}{=}\PYG{+w}{ }\PYG{l+m+mi}{100}\PYG{p}{;}
\PYG{n}{sc}\PYG{p}{.}\PYG{n}{NumPrePostSamples}\PYG{+w}{ }\PYG{p}{=}\PYG{+w}{ }\PYG{o}{\PYGZhy{}}\PYG{l+m+mi}{50}\PYG{p}{;}
\PYG{n}{sc}\PYG{p}{.}\PYG{n}{Decimation}\PYG{+w}{ }\PYG{p}{=}\PYG{+w}{ }\PYG{l+m+mi}{1}\PYG{p}{;}
\PYG{n}{sc}\PYG{p}{.}\PYG{n}{TriggerMode}\PYG{+w}{ }\PYG{p}{=}\PYG{+w}{ }\PYG{l+s}{\PYGZsq{}}\PYG{l+s}{Signal\PYGZsq{}}\PYG{p}{;}
\PYG{n}{sc}\PYG{p}{.}\PYG{n}{TriggerSignal}\PYG{+w}{ }\PYG{p}{=}\PYG{+w}{ }\PYG{n}{getsignalid}\PYG{p}{(}\PYG{n}{bd}\PYG{p}{.}\PYG{n}{tg}\PYG{p}{,}\PYG{+w}{ }\PYG{l+s}{\PYGZsq{}}\PYG{l+s}{gen\PYGZus{}running\PYGZsq{}}\PYG{p}{)}\PYG{p}{;}
\PYG{n}{sc}\PYG{p}{.}\PYG{n}{TriggerLevel}\PYG{+w}{ }\PYG{p}{=}\PYG{+w}{ }\PYG{l+m+mf}{0.5}\PYG{p}{;}
\PYG{n}{sc}\PYG{p}{.}\PYG{n}{TriggerSlope}\PYG{+w}{ }\PYG{p}{=}\PYG{+w}{ }\PYG{l+s}{\PYGZsq{}}\PYG{l+s}{Rising\PYGZsq{}}\PYG{p}{;}

\PYG{c}{\PYGZpc{}\PYGZpc{} Generating TTL Output}

\PYG{n+nb}{fprintf}\PYG{p}{(}\PYG{l+s}{\PYGZsq{}}\PYG{l+s}{\PYGZbs{}nTesting... \PYGZsq{}}\PYG{p}{)}
\PYG{n+nb}{start}\PYG{p}{(}\PYG{n}{sc}\PYG{p}{)}\PYG{p}{;}
\PYG{n+nb}{pause}\PYG{p}{(}\PYG{l+m+mf}{0.1}\PYG{p}{)}\PYG{p}{;}\PYG{+w}{ }\PYG{c}{\PYGZpc{} give the scope time to pre\PYGZhy{}aquire}
\PYG{n+nb}{assert}\PYG{p}{(}\PYG{n+nb}{strcmp}\PYG{p}{(}\PYG{n}{sc}\PYG{p}{.}\PYG{n}{Status}\PYG{p}{,}\PYG{+w}{ }\PYG{l+s}{\PYGZsq{}}\PYG{l+s}{Ready for TTL Output generation\PYGZsq{}}\PYG{p}{)}\PYG{p}{)}\PYG{p}{;}

\PYG{n}{s}\PYG{+w}{ }\PYG{p}{=}\PYG{+w}{ }\PYG{p}{[}\PYG{l+m+mi}{0}\PYG{p}{,}\PYG{+w}{ }\PYG{l+m+mi}{1}\PYG{p}{,}\PYG{+w}{ }\PYG{l+m+mi}{0}\PYG{p}{]}\PYG{p}{;}
\PYG{n}{s}\PYG{p}{(}\PYG{l+m+mi}{1000}\PYG{p}{,}\PYG{l+m+mi}{3}\PYG{p}{)}\PYG{+w}{ }\PYG{p}{=}\PYG{+w}{ }\PYG{l+m+mi}{0}\PYG{p}{;}\PYG{+w}{ }\PYG{c}{\PYGZpc{} fill with zeros (TODO: this should be done in the API)}
\PYG{n}{bd}\PYG{p}{.}\PYG{n}{generator\PYGZus{}sequence}\PYG{+w}{ }\PYG{p}{=}\PYG{+w}{ }\PYG{n}{s}\PYG{p}{;}

\PYG{n}{bd}\PYG{p}{.}\PYG{p}{;}

\PYG{n+nb}{pause}\PYG{p}{(}\PYG{l+m+mf}{0.1}\PYG{p}{)}
\PYG{n+nb}{assert}\PYG{p}{(}\PYG{n+nb}{strcmp}\PYG{p}{(}\PYG{n}{sc}\PYG{p}{.}\PYG{n}{Status}\PYG{p}{,}\PYG{+w}{ }\PYG{l+s}{\PYGZsq{}}\PYG{l+s}{Finished\PYGZsq{}}\PYG{p}{)}\PYG{p}{)}

\PYG{n+nb}{fprintf}\PYG{p}{(}\PYG{l+s}{\PYGZsq{}}\PYG{l+s}{loop delay is \PYGZpc{}2.1f ms\PYGZbs{}n\PYGZsq{}}\PYG{p}{,}\PYG{+w}{ }\PYG{p}{(}\PYG{n+nb}{find}\PYG{p}{(}\PYG{n}{sc}\PYG{p}{.}\PYG{n}{Data}\PYG{p}{(}\PYG{p}{:}\PYG{p}{,}\PYG{l+m+mi}{1}\PYG{p}{)}\PYG{p}{,}\PYG{+w}{ }\PYG{l+m+mi}{1}\PYG{p}{)}\PYG{o}{\PYGZhy{}}\PYG{l+m+mi}{50}\PYG{p}{)}\PYG{o}{/}\PYG{l+m+mi}{5}\PYG{p}{)}
\end{sphinxVerbatim}


\bigskip\hrule\bigskip



\chapter{Downloads \& Dependencies}
\label{\detokenize{6_downloads_n_dependencies:downloads-dependencies}}\label{\detokenize{6_downloads_n_dependencies::doc}}

\section{Download \& Setup API}
\label{\detokenize{6_downloads_n_dependencies:download-setup-api}}
\sphinxAtStartPar
bossdevice RESEARCH API version 2021 is an open source MATLAB repository incubated at sync2brain’s GitHub account and is linked below:

\begin{sphinxVerbatim}[commandchars=\\\{\}]
\PYG{n}{https}\PYG{p}{:}\PYG{o}{/}\PYG{o}{/}\PYG{n}{github}\PYG{p}{.}\PYG{n}{com}\PYG{o}{/}\PYG{n}{sync2brain}\PYG{o}{/}\PYG{n}{bossdevice}\PYG{o}{\PYGZhy{}}\PYG{n}{api}\PYG{o}{\PYGZhy{}}\PYG{n}{matlab}
\end{sphinxVerbatim}


\bigskip\hrule\bigskip



\section{Download bossdevice firmware}
\label{\detokenize{6_downloads_n_dependencies:download-bossdevice-firmware}}
\sphinxAtStartPar
bossdevice RESEARCH simulink based frmware is also incubated at sync2brain’s download’s area and is linked below:

\begin{sphinxVerbatim}[commandchars=\\\{\}]
\PYG{n}{https}\PYG{p}{:}\PYG{o}{/}\PYG{o}{/}\PYG{n}{sync2brain}\PYG{p}{.}\PYG{n}{com}\PYG{o}{/}\PYG{n}{downloads}
\end{sphinxVerbatim}


\bigskip\hrule\bigskip



\section{Requirements:}
\label{\detokenize{6_downloads_n_dependencies:requirements}}
\begin{sphinxadmonition}{important}{Important:}
\sphinxAtStartPar
Matlab 2021a downloaded and installed
\end{sphinxadmonition}

\begin{sphinxadmonition}{important}{Important:}
\sphinxAtStartPar
Simulink Real\sphinxhyphen{}Time and Simulink Coder, Simulink, Simulink Target Support package to be installed as an Adds\sphinxhyphen{}On from MATLAB Toolboxes
\end{sphinxadmonition}

\begin{sphinxadmonition}{important}{Important:}
\sphinxAtStartPar
Download and install the hardware I/O blockset drivers from the downloads area of sync2brain.com/downloads
\end{sphinxadmonition}

\begin{sphinxadmonition}{important}{Important:}
\sphinxAtStartPar
Download the repository from master branch
\end{sphinxadmonition}

\begin{sphinxadmonition}{important}{Important:}
\sphinxAtStartPar
Add the bossdevice RESEARCH firmware and repository to the search path of MATLAB
\end{sphinxadmonition}

\begin{sphinxadmonition}{important}{Important:}
\sphinxAtStartPar
Observe from command line that if bossdevice firmware has been loaded from intended path, otherwise many copies of API are available on your device and MATLAB is confused about choosing one therefore it is recommended to change the current directory of MATLAB to the correct path of bossdevice latest API.
\end{sphinxadmonition}


\chapter{Issues, Bugs and Requests}
\label{\detokenize{7_issues_bugs_requests:issues-bugs-and-requests}}\label{\detokenize{7_issues_bugs_requests::doc}}

\section{Issues | Bugs | Requests}
\label{\detokenize{7_issues_bugs_requests:issues-bugs-requests}}
\sphinxAtStartPar
You first have to sign up for an account or log in on GitHub, subsequently you can report the issue on the GitHub repositroty linked below:

\begin{sphinxVerbatim}[commandchars=\\\{\}]
\PYG{n}{ttps}\PYG{p}{:}\PYG{o}{/}\PYG{o}{/}\PYG{n}{github}\PYG{p}{.}\PYG{n}{com}\PYG{o}{/}\PYG{n}{sync2brain}\PYG{o}{/}\PYG{n}{bossdevice}\PYG{o}{\PYGZhy{}}\PYG{n}{api}\PYG{o}{\PYGZhy{}}\PYG{n}{matlab}\PYG{o}{/}\PYG{n}{issues}
\end{sphinxVerbatim}

\sphinxAtStartPar
We automatically will receive an email and will follow up and keep you updated; i.e., you will get an email through GitHub whenever someone works on your issue.

\begin{sphinxadmonition}{important}{Important:}
\sphinxAtStartPar
Alternatively, you can also write an email to sync2brain support team for a quick response to your questions.
\end{sphinxadmonition}


\bigskip\hrule\bigskip



\section{Issues \& Bugs Details}
\label{\detokenize{7_issues_bugs_requests:issues-bugs-details}}
\sphinxAtStartPar
The easier it is for one of the developers to reproduce your bug, the more likely it is that we’ll fix the problem. Good bug reports include a small test script and the data (i.e. mat file) required to reproduce the bug.

\sphinxAtStartPar
Please create a small test script and a piece of data that are both as small and simple as possible to reproduce the problem. For example: a .mat file containing the bossdevice RESEARCH API class object and some screenshots can lead us to solve the issue.


\bigskip\hrule\bigskip



\section{Use Case Scenarios}
\label{\detokenize{7_issues_bugs_requests:use-case-scenarios}}
\begin{sphinxadmonition}{important}{Important:}
\sphinxAtStartPar
Use cases, previous research papers that might have implemented those methods or a brief description about the feature requests can help our developers to create the smartest solution for you.
\end{sphinxadmonition}


\bigskip\hrule\bigskip



\section{Troubleshooting}
\label{\detokenize{7_issues_bugs_requests:troubleshooting}}\begin{itemize}
\item {} 
\sphinxAtStartPar
I am not receiving correct/any markers.
\sphinxhyphen{}  Step 1: If using NeurOne, please check that the protocol is set properly and “Send Markers as Channels” is selected.
\sphinxhyphen{}  Step 2: Ensure that eeg\_channels and aux\_channels property is set according to the streaming protocol i.e. if there are no aux channels streaming then they should be set to 0.

\end{itemize}



\renewcommand{\indexname}{Index}
\printindex
\end{document}